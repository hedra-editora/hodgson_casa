\textbf{William Hope Hodgson} (Essex, 1877---Ypres, 1918), marinheiro, fisioculturista, 
fotógrafo e escritor, embora pouco conhecido entre os leitores de língua portugesa, 
produziu um grande número de ensaios, contos e romances, entre os quais se destacam 
os de terror e fantasia. Com o livro \textit{The Boats of the Glen Carrig}, publicado em 1907,
alcançou a projeção que lhe convenceu a dedicar-se à carreira de escritor. 
Deixou a marinha em 1912, mas teve de se alistar no exército com a explosão da Primeira Guerra Mundial,
sendo morto com um tiro durante um confronto em Ypres, na Bélgica.
Sua obra influenciou autores como H.P. Lovecraft, que sempre explicitou sua admiração pela escrita
de Hodgson. 



\textbf{A casa do fim do mundo}, publicado originalmente em 1908, 
trata das experiências do narrador de um misterioso manuscrito, 
encontrado por dois viajantes nas ruínas de uma casa afastada, 
em um estranho vilarejo irlandês. O eremita responsável pelo registro dos últimos meses 
como morador da velha e maligna construção, sobre a qual agiam forças terríveis, transporta-nos
a uma atmosfera de terror -- criaturas híbridas e ameaçadoras que habitam um fosso sinistro 
sob as fundações da casa entre dois mundos, uma inesperada jornada astral por eras incontáveis 
até que se testemunhe o fim de nosso sistema solar -- que desafia categorias,
configurando essa obra, nas palvras de H.P. Lovecraft, como ``provavelmente a maior de Hodgson 
e algo absolutamente único na literatura universal''.



\textbf{Juliana Lemos} é tradutora literária há mais de dez anos, tendo traduzido obras de Neil Gaiman, 
Daniel Woodrell e C. S. Lewis, além de diversos trabalhos nas áreas: 
Economia, Direito Constitucional, Marketing, Jornalismo e Games.