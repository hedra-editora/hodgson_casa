%A casa do fim do mundo
%William Hope Hodgson
%Tradução: Juliana Lemos
%
%A partir do manuscrito encontrado em 1877
%pelos Srs. Tonnison e Berreggnog,
%nas ruínas ao sul do vilarejo de Kraighten,
%oeste da Irlanda.
%
%Transcrito com notas
%de William Hope Hodgson

%Introdução
%
%Quando li este livro pela primeira vez na infância, há quase cinquenta anos, eu já tinha uma prateleira cheia de
%histórias de horror --- de Stoker a Derleth --- e um gosto exigente: adorava todas. Ratos dentro de paredes, mãos decepadas
%assassinas, \textit{coisas} inomináveis oriundas de lugares inimagináveis --- tudo isso alimentava a minha imaginação,
%faminta por subsistir da despensa da vida real no centro"-oeste americano. Tudo bem que Jesus me amasse --- não era um
%fato interessante, mas tudo bem ---, mas certamente o Yog"-Sothoth de Lovecraft não me amava, e me parecia mais do que
%plausível que, se o universo tinha espaço para deuses, eles deveriam ser do tipo maligno, como os de Lovecraft.
%
%Quando me deparei com \textit{A casa entre dois mundos}, já conhecia o aterrorizante conto do autor,\textit{“The Voice
%in the Night”}. Desde então, a imagem de um pequeno e maligno ponto fungiforme na mão que cresce cada vez mais assombra
%minhas noites de maneiras bem interessantes. Que o desventurado amante tenha remado num barquinho até um navio parado
%no mar para implorar por comida para si mesmo e sua amada, depois que toda a esperança de cura se esvai, parecia me
%dizer algo sobre o papel do \textit{páthos} no amor que eu, ainda jovem, até então não suspeitava.
%
%Em 1962, a Ace Books publicou uma versão em papel jornal de \textit{A casa entre dois mundos}, e foi provavelmente aí
%que, pela primeira vez, a maioria dos leitores do pós"-guerra foram expostos a um texto mais extenso de William Hope
%Hodgson. No começo da década de 1970, a Ballantine Books publicou duas outras fantasias de Hodgson, \textit{The Boats
%of the “Glen Carrig”} e \textit{The Night Land}. As duas têm seus pontos fortes --- se você está buscando a sensação de
%uma mão úmida e pegajosa na nuca ---, mas nenhuma delas transmite ao leitor o arrepiante desespero de \textit{A casa no
%fim do mundo}. Não há um único feixe de luz esmaecida neste livro que não tenha um quê de terror. Se algum escritor
%moderno ultrapassou Hodgson nesse aspecto, eu o desconheço.
%
%\textit{A casa entre dois mundos}, publicado pela primeira vez em 1908, é mais do que uma história de horror. Quando o
%narrador não está combatendo as criaturas suínas que habitam as cavernas sob a casa isolada, ele está sendo
%transportado para um futuro desolador, cada vez imerso na escuridão. A Terra está morrendo. O Sol está morrendo. Dias e
%séculos passam voando como as imagens projetadas nas paredes por uma luminária acesa. O que nos aguarda é uma escuridão
%vazia sem fim; Hodgson compreendia que o verdadeiro horror é ter um vislumbre da dimensão cósmica.
%
%Se algumas das descrições neste livro devem parte de seu mérito a \textit{A máquina do tempo}, de H. G. Wells (publicado
%originalmente em 1895), muitos escritores posteriores devem muito mais a Hodgson. Os deuses monstruosos que observam a
%Casa entre Dois Mundos são antecedentes diretos dos deuses de Lovecraft. Para ambos os escritores, o universo, em seu
%âmago, é um lugar apavorante e totalmente avesso a nós.
%
%Uma década depois do lançamento deste livro, Hodgson morreu em combate, aos quarenta anos de idade, durante a \textsc{i} Guerra
%Mundial. 
%
%MAX GERSH


\chapter{Introdução do autor ao manuscrito}

\textsc{Muitas foram} as horas em que refleti sobre a história transcrita nas páginas a seguir.\footnoteInSection{A partir do manuscrito encontrado em 1877
pelos srs.~Tonnison e Berreggnog, nas ruínas ao sul do vilarejo de Kraighten, oeste da Irlanda. Transcrito com notas
de William Hope Hodgson.} Acredito que meus instintos estejam
corretos ao deixar o relato, com toda sua simplicidade, exatamente como me foi entregue.
                                                       
Quanto ao manuscrito em si --- o leitor deve me imaginar, quando ele primeiro caiu sob meus cuidados, revirando"-o,
curioso, e perscrutando"-o de maneira rápida e espasmódica. É um livro pequeno, mas cheio, exceto pelas últimas páginas,
de uma caligrafia antiquada, embora legível, com letras pouco espaçadas. Agora, enquanto escrevo, ainda consigo sentir
o leve e estranho cheiro de água que emana de suas páginas, e meus dedos retêm subconscientemente a memória da textura
macia e grudenta das páginas umedecidas durante anos.

Eu o li e, ao fazê"-lo, levantei as Cortinas do Impossível que cegam a mente e pude olhar diretamente para o
desconhecido. Vagueei por entre suas frases ásperas e rígidas e logo não tinha críticas a fazer a seu estilo abrupto,
pois, melhor do que minha narração ambiciosa, esta história mutilada é capaz de nos fazer compreender tudo
aquilo que o velho Eremita, na casa desaparecida, tentou nos dizer.

Sobre o relato simples e direto de assuntos estranhos e extraordinários, pouco direi. A história fala por si só. Cada
leitor deve descobrir a verdade, de acordo com seus desejos e capacidades. Mesmo que o leitor não
possa enxergar, como eu agora enxergo, a sombria imagem e o conceito daquilo a que podemos dar os nomes já aceitos de
Céu e Inferno, garanto que a leitura é capaz de proporcionar certas emoções, mesmo se a história for considerada
somente uma história.
\bigskip

{\raggedleft
William Hope Hodgson,\\
17 de dezembro de 1907.
\par}


\clearpage

\section{A meu pai}

\bgroup\centering
(\emph{cujos pés percorrem eras infinitas})
\par\smallskip\egroup

\begin{verse}
Abra a porta\\
E ouça!\\
O vento aporta\\
E balouçam\\
Lágrimas ao redor da Lua.\\
E os rastros tortos\\
Na imaginação tua\\
Seguem atrás dos Mortos.

Silêncio! Ouça\\
O triste lamentar\\
Do choro na escuridão.\\
Ouça com atenção, sem suspirar,\\
O eco de passos pela eternidade:\\
Parece à morte convidar.\\
Ouça com atenção! Ouça!\\
O rastro dos Mortos.
\end{verse}


\cleardoublepage


\chapter{A descoberta do manuscrito}

\textsc{Bem no} litoral oeste da Irlanda, há uma pequena aldeia chamada Kraighten. Solitária, ela situa"-se ao pé de um morro
baixo. Em todo o vasto arredor, o terreno é lúgubre e totalmente inóspito; aqui e acolá, a grandes intervalos de
distância, é possível deparar com as ruínas de algum casebre há muito abandonado --- grosseiro e sem palha no telhado.
Toda a área é vazia e inabitada, e a própria terra mal cobre a rocha que jaz embaixo de si, rocha que sobeja em toda a
região, por vezes erguendo"-se do solo em cumes semelhantes a ondas.

Entretanto, apesar de tamanha desolação, meu amigo Tonnison e eu escolhemos passar nossas férias ali. Ele deparara
com o local por puro acaso, no ano anterior, durante um longo passeio a pé, e descobriu que era possível pescar num rio
pequeno e sem nome que corre nos arredores do vilarejo.

Mencionei que o rio não tem nome; devo acrescentar que nenhum mapa consultado até o momento indica nem o vilarejo nem
o rio. Parecem ter escapado completamente à observação; de fato, nem sequer existem, de acordo com os guias
turísticos comuns. Isso pode ser que se deva ao fato de que a estação de trem mais próxima (em Ardrahan) fica a mais de
sessenta quilômetros.

Era de noitinha e o clima estava ameno quando eu e meu amigo chegamos a Kraighten. Havíamos alcançado Ardrahan na noite
anterior, e alugamos aposentos dos correios do vilarejo, onde dormimos, e partimos em bom horário, na manhã seguinte,
em uma carroça típica, agarrando"-nos precariamente às bordas para manter o equilíbrio.

Levamos o dia inteiro para completar a viagem, percorrendo estradas muito difíceis, e o resultado foi que ficamos
extremamente cansados e um tanto irritadiços. Contudo, era preciso erguer a barraca e guardar os mantimentos antes de
comer ou descansar. Assim, começamos a trabalhar, com a ajuda do cocheiro, e logo tínhamos a barraca montada em uma
pequena área um pouco longe do vilarejo, bem próxima ao rio.

Então, depois de guardar nossos pertences, despedimo"-nos do cocheiro, já que ele precisava voltar o mais rápido
possível, e pedimos que viesse nos buscar depois de duas semanas. Trouxéramos mantimentos em quantidade suficiente para
aquele período e poderíamos pegar água do rio. Não precisávamos de combustível, já que havíamos trazido também uma
pequena estufa, e o tempo estava agradável e quente.

Foi ideia de Tonnison acampar do lado de fora em vez de nos abrigarmos em um dos casebres. Como ele disse, não seria
divertido dormir num aposento com uma enorme família de saudáveis irlandeses num canto e um chiqueiro no outro enquanto
do teto uma desfalcada colônia de galinhas distribuía generosamente suas bênçãos, num ambiente com tanta fumaça de
turfa queimada que seria capaz de fazer alguém espirrar até explodir só de pôr a cabeça na porta.

Tonnison já havia acendido a salamandra e estava ocupado cortando fatias de bacon na frigideira; então peguei a chaleira
e fui até o rio, em busca de água. No caminho, precisei passar perto de um pequeno grupo de habitantes locais, que me
observaram curiosos, mas não com ar hostil, por mais que nenhum deles tivesse pronunciado palavra.

Quando voltei com a chaleira cheia, fui até eles e, depois de fazer um gesto amistoso com a cabeça, ao qual responderam
da mesma maneira, perguntei"-lhes casualmente sobre os peixes; mas, em vez de responderem, simplesmente menearam a
cabeça em silêncio, olhando para mim. Repeti a pergunta, dirigindo"-me especificamente a um sujeito grande e esquálido
perto de mim; mais uma vez, não recebi resposta. E então o homem virou"-se para um de seus companheiros e disse algo
rápido numa língua que não compreendi; e, no mesmo instante, todo o grupo começou a tagarelar no que, depois de alguns
momentos,	 imaginei ser o mais puro gaélico. Enquanto conversavam, lançavam olhares na minha direção. Talvez tenham
ficado um minuto conversando assim; depois, o homem a quem me dirigira virou"-se, fitou"-me e disse algo. Pela expressão
em seu rosto, imaginei que ele, por sua vez, estivesse me indagando sobre algo; mas agora era eu quem precisava
balançar a cabeça para indicar que não compreendia o que queriam saber; e lá permanecemos, olhando um para o outro, até
que ouvi Tonnison chamar, dizendo para me apressar com a chaleira. Então, sorrindo e com um gesto de cabeça, eu os
deixei, e todos no pequeno grupo sorriram e fizeram o mesmo gesto, embora suas expressões ainda indicassem que estavam
confusos.

Era evidente, refleti, enquanto caminhava rumo à barraca, que os habitantes daquelas poucas cabanas no meio do nada não
sabiam uma única palavra de inglês; quando disse isso a Tonnison, ele comentou que estava ciente e, ademais, que não
era algo de todo incomum naquela região do país, onde pessoas muitas vezes viviam e morriam em vilarejos isolados sem
jamais entrar em contato com o mundo exterior.

--- Queria que o cocheiro tivesse sido nosso intérprete antes de partir --- comentei, quando sentamos para fazer nossa
refeição. --- É estranho que as pessoas deste lugar nem sequer possam saber por que viemos para cá.

Tonnison grunhiu, concordando, e depois ficamos em silêncio durante algum tempo.

Mais tarde, depois de matar um pouco nossa fome, conversamos e fizemos planos para o dia seguinte; então, depois de
fumar, fechamos a entrada da barraca e nos preparamos para dormir. 

--- Imagino que aqueles camaradas lá fora não roubarão nada, certo? --- perguntei, enquanto nos enrolávamos nos cobertores.

Tonnison disse que achava que não, pelo menos não enquanto estivéssemos ali; e, continuou ele, poderíamos trancar tudo,
menos a barraca, no grande baú que trouxemos para guardar nossos mantimentos. Concordei, e logo adormecemos.

Na manhã seguinte, levantamos cedo e fomos nadar no rio; ao voltar, vestimo"-nos e tomamos o café da manhã. Depois,
examinamos e reparamos o nosso equipamento de pesca; quando terminamos, o café da manhã já bem digerido, deixamos tudo
seguro dentro da barraca e saímos na direção que meu amigo havia explorado em sua visita prévia.

Durante o dia, pescamos felizes, sempre seguindo pelo rio na direção contrária da corrente e, ao cair da tarde, já
tínhamos uma cesta repleta de peixes, bonita como havia muito eu não via. Ao voltarmos para a vila, fizemos um belo
banquete com a pesca do dia e então, depois de selecionar alguns dos melhores peixes para o nosso café da manhã,
presenteamos o restante ao grupo de aldeões que havia se reunido a uma distância respeitosa para observar o que
fazíamos. Pareceram ficar extremamente gratos e fizeram muitos gestos sobre nossas cabeças, que imaginei serem bênçãos
tipicamente irlandesas.

E assim passamos vários dias, divertindo"-nos a valer, com muito apetite para fazer jus ao que pescávamos. Ficamos
satisfeitos ao ver que os habitantes do lugar estavam dispostos a ser simpáticos conosco e que não havia nenhum
indício de que tentaram mexer em nossos pertences durante nossa ausência.

Foi numa terça"-feira que chegamos a Kraighten e seria no domingo seguinte que fizemos uma grande descoberta. Até então,
sempre havíamos subido o rio; entretanto, naquele dia, deixamos de lado nossas varas de pesca e, levando conosco alguns
mantimentos, saímos para passear na direção oposta. O dia estava quente e caminhávamos com dificuldade pelo terreno,
mas sem pressa, parando perto do meio"-dia para almoçar sobre uma grande rocha plana, perto da beira do rio. Depois,
ficamos algum tempo sentados, fumando, voltando a caminhar só depois de cansar de ficar parados.

Durante talvez mais uma hora continuamos andando a esmo, conversando agradavelmente em tom baixo sobre vários assuntos,
parando em diversos momentos para que meu amigo --- que é meio artista --- fizesse esboços não muito bem acabados de
detalhes mais marcantes da paisagem silvestre.

Então, sem nenhum sobreaviso, o rio que seguíamos tão confiantes chegou ao fim --- desapareceu no meio da terra.

--- Deus meu! --- eu disse. --- Quem poderia imaginar algo assim?

E fiquei olhando, aparvalhado, e voltei"-me para Tonnison. Ele olhava com expressão neutra para o ponto em que o
rio desaparecia.

Depois de um instante, propôs:

--- Vamos continuar mais um pouco, pode ser que ele reapareça. De todo modo, vale a pena investigar.

Concordei, e novamente pusemo"-nos a andar, mas agora mais ou menos a esmo, pois não tínhamos muita certeza sobre que
direção tomar. Continuamos adiante por talvez mais um quilômetro e meio; então Tonnison, que olhava ao redor com
curiosidade, parou e pôs a mão acima dos olhos para fazer sombra.

--- Olhe! --- disse ele, depois de um instante. --- Não é uma névoa ou algo parecido ali, mais à direita --- alinhada com aquela
rocha enorme? --- E apontou para o local.

Fiquei olhando e, depois de alguns instantes, achei que via algo, mas não tinha certeza, e comuniquei minha dúvida.

--- Mesmo assim, vamos até lá dar uma olhada --- respondeu meu amigo. Começou a andar na direção que havia sugerido e fui
atrás. Logo em seguida, deparamos com uma mata de arbustos e, depois de certo tempo, saímos dos arbustos sobre uma
barragem alta, cheia de pedregulhos, e lá de cima vimos uma vasta selva de arbustos e árvores.

--- Pelo jeito, encontramos um oásis neste deserto de pedra --- murmurou Tonnison, observando com interesse. E então ficou em
silêncio, o olhar num ponto fixo; e eu também olhei. Pois em algum ponto, mais ou menos no centro daquela floresta,
erguia"-se no ar silencioso uma enorme coluna de respingos finos, como se fosse uma névoa, sobre a qual incidia o sol,
gerando inúmeros arco"-íris.

--- Que bonito! --- exclamei.

--- Sim --- respondeu Tonnison, pensativo. --- Deve haver uma cachoeira ou algo assim ali. Talvez seja o nosso
rio. Vamos lá ver.

Descemos pela barragem em declive e penetramos o agrupamento de árvores e arbustos. Era um emaranhado de arbustos e
árvores altas, e o lugar era desagradavelmente escuro; mas não escuro o suficiente para deixarmos de perceber que
muitas das árvores tinham frutos e que, ali e acolá, era possível ver, sem grandes detalhes, sinais muito antigos de
cultivo. E assim me dei conta de que estávamos entrando em um enorme e antigo jardim. Mencionei o fato a Tonnison e ele
concordou que havia motivo para desconfiar disso.

Que lugar inóspito, lúgubre e sombrio! De certa forma, à medida que avançávamos, a sensação da solidão silenciosa e do
abandono daquele velho jardim começou a apoderar"-se de mim, e senti calafrios. Podia imaginar coisas ocultas por atrás
do emaranhado de arbustos e percebi algo sinistro pairando no ar. Acredito que Tonnison também provasse o mesmo, embora
nada dissesse.

De repente, paramos. Por entre as árvores, um som distante aos poucos se fazia cada vez mais alto. Tonnison inclinou"-se
para frente, tentando ouvir com atenção. Agora, eu conseguia ouvir com mais clareza; era um som contínuo e áspero --- uma
espécie de urro constante que parecia vir de bem longe. Senti um leve nervosismo, uma sensação estranha, indescritível.
Que lugar era aquele em que havíamos nos metido? Olhei para meu amigo para ver o que ele achava e em seu rosto só havia
confusão; então, enquanto eu o observava, seu rosto foi iluminado por uma expressão de compreensão e ele meneou
afirmativamente a cabeça.

--- É uma cachoeira! --- exclamou, convicto. --- Agora sei o que é este barulho. --- E começou a empurrar com força
os arbustos, rumo ao barulho.

À medida que avançávamos, o som tornava"-se mais nítido, o que indicava que estávamos na direção certa. O urro
ficava cada vez mais alto e mais próximo, até quase parecer, como eu disse a Tonnison, que vinha de debaixo de nossos
pés --- e, mesmo assim, continuávamos ainda cercados por árvores e arbustos.

--- Cuidado! --- Tonnison exclamou. --- Olhe bem onde pisa. --- Então, de repente, saímos das árvores e chegamos a um lugar enorme
e descampado, onde, a meros seis passos à nossa frente, abria"-se um gigantesco abismo, de cujas profundezas o som
parecia subir, com os respingos contínuos, semelhantes a névoa, que havíamos visto do topo da barragem
distante.

Durante quase um minuto ficamos parados, em silêncio, olhando boquiabertos aquela cena; então meu amigo foi com
cuidado até a beira do abismo. Fui atrás e, juntos, espiamos lá embaixo, através da massa de respingos, e vimos uma
gigantesca catarata de água borbulhante que explodia e jorrava da lateral do abismo, quase trinta metros abaixo de nós.

--- Meu Deus! --- disse Tonnison.

Fiquei em silêncio, bastante admirado. Era uma paisagem inesperadamente magnífica e estranha, mas só me dei conta dessa
última característica depois.

Logo em seguida olhei para o outro lado distante do abismo. Ali, vi algo que avultava em meio à névoa de respingos:
parecia uma grande ruína, e chamei a atenção de Tonnison, tocando seu ombro. Ele levou um susto, olhou em volta, e eu
apontei para a coisa. Ele seguiu meu dedo com o olhar e de repente, quando avistou o objeto, seus olhos se iluminaram
de alegria.

--- Venha comigo --- ele gritou por cima do estrondo da catarata. --- Precisamos averiguar. Há algo muito esquisito neste lugar;
estou com essa estranha sensação. --- E começamos a andar, contornando aquele abismo semelhante a uma cratera. À medida
que nos aproximávamos daquela nova coisa, vi que minha primeira impressão não estava errada. Era, sem dúvida, parte das
ruínas de alguma construção; todavia, eu agora via que não era algo que fora construído à beira do abismo, como eu
imaginara, e sim encarapitado quase no cume de uma rocha que se projetava uns quinze ou vinte metros sobre o abismo. Na
verdade, era como se aquela ruína cheia de pedras irregulares estivesse suspensa no ar. 

Chegando perto dela, rumamos para o braço da rocha que se projetava sobre o abismo, e devo confessar que senti um terror
indescritível quando olhei, atordoado, para as profundezas desconhecidas lá embaixo --- as profundezas das quais vinham o
urro da água que caía e a névoa de respingos.

Chegando às ruínas, começamos a contorná"-las e escalá"-las com cuidado, e, do outro lado, encontramos uma pilha de pedras
e escombros. A ruína em si me parecia ser, quando comecei a examiná"-la com mais atenção, parte da parede exterior de
alguma estrutura gigantesca, já que era espessa e muito bem construída, mas o que ela fazia em tal posição eu não
conseguia imaginar. Onde estava o restante da casa, ou castelo, ou o que quer que aquilo tenha sido?

Voltei para o lado externo da parede e dali fui até a beira do abismo, deixando Tonnison revirando metodicamente a pilha
de pedras e pedregulhos do lado externo da parede. E então passei a examinar a superfície do chão perto da beira do
abismo, para ver se não havia mais resquícios da construção à qual aquela ruína obviamente pertencera. Mas, por mais
que examinasse a terra com atenção, não consegui ver nenhum sinal de que já houvera uma construção erigida naquele
local. Fiquei mais confuso do que nunca.

De repente ouvi Tonnison berrar: ele gritava meu nome, animado. Imediatamente corri pela rocha escarpada até a ruína.
Fiquei pensando se ele não teria se machucado, mas depois pensei que talvez tivesse achado algo.

Cheguei à parede em ruínas e dei a volta sobre os escombros. E lá encontrei Tonnison parado perto de um pequeno buraco
que havia feito em meio ao entulho: estava tirando o pó de algo que parecia ser um livro, muito amassado e destruído, e
abria a boca, cada um ou dois segundos, para gritar meu nome. Assim que viu que eu havia chegado, entregou"-me seu
achado, dizendo que eu deveria colocar na minha bolsa para protegê"-lo da umidade enquanto ele continuava a explorar o
lugar. Obedeci, mas, primeiro, folheei as páginas: percebi que estavam todas escritas, cheias de uma caligrafia bonita,
antiquada e bastante legível, exceto pelo trecho em que muitas das páginas estavam destruídas, sujas de terra e
amassadas, como se o livro tivesse sido enterrado aberto naquele ponto. E foi exatamente assim, como me informou
Tonnison, que ele havia descoberto o livro. O dano provavelmente fora causado pela queda dos escombros sobre a parte
aberta. O curioso era que o livro estava relativamente seco, o que atribuí ao fato de ele ter ficado muito bem
enterrado em meio aos escombros.

Depois de guardar com cuidado o livro, voltei e ajudei Tonnison com a tarefa que ele havia se imposto, escavar; contudo,
embora tenhamos ficado ali mais de uma hora revirando toda a pilha de pedras e pedregulhos, não encontramos nada além
de fragmentos de madeira quebrada, talvez partes de uma mesa de jantar ou escrivaninha; e, assim, desistimos de nossa
busca e fizemos o percurso inverso pela rocha, voltando para o solo seguro.

Em seguida, contornamos totalmente o enorme abismo, o qual, como pudemos observar, era um círculo quase perfeito, exceto
pela rocha que sustentava as ruínas e desembocava sobre ele, estragando sua simetria.

O abismo era, como descreveu Tonnison, um poço ou fosso gigante que parecia alcançar as entranhas da Terra.

Durante mais algum tempo, continuamos a observar o local, e então, ao perceber que havia um espaço vazio ao norte do
abismo, fomos até lá.

Ali, distante da boca daquele fosso impressionante, a centenas de metros, deparamos com um enorme lago com águas
tranquilas --- exceto por um local, onde a água borbulhava de modo contínuo.

Distantes agora do ruído da catarata, conseguíamos ouvir a voz do outro sem precisar gritar a plenos pulmões, e
perguntei a Tonnison o que ele achava daquele lugar --- eu havia dito a ele que não gostava e que, quanto mais cedo
fôssemos embora dali, melhor eu me sentiria.

Ele concordou e lançou um olhar furtivo para a floresta atrás de nós. Perguntei se ele vira ou ouvira algo. Ele não
respondeu; permaneceu em silêncio, como se tentasse escutar algo, e eu também fiquei em silêncio.

De repente, ele disse, ríspido:

--- Ouça! --- Olhei para ele, e depois para as árvores e arbustos, instintivamente segurando a respiração. Um minuto passou"-se
em que nós dois ficamos em silêncio; contudo, eu não ouvia nada e virei"-me para Tonnison para informá"-lo do fato; e
então, bem no instante em que abri os lábios para falar, um estranho ruído, como se fosse um gemido, saiu do meio da
floresta à nossa esquerda\ldots{} Parecia flutuar por entre as árvores, e depois veio um farfalhar da folhagem, e então o
silêncio.

No mesmo instante, Tonnison falou, colocando a mão no meu ombro: --- Vamos embora daqui --- e dirigiu"-se lentamente para o
ponto em que as árvores e arbustos ao redor não pareciam tão cerrados. Enquanto eu o seguia, percebi de repente que o
Sol estava baixo e que o ar estava frio.

Tonnison nada mais disse, continuou andando. Agora já estávamos em meio às árvores, e eu olhava em volta, nervoso, mas
nada via além dos galhos, troncos e arbustos emaranhados e silenciosos. Continuamos em frente e nenhum som rompia o
silêncio além de um ou outro graveto quebrando"-se sob nossos pés. E, mesmo assim, apesar do silêncio, eu tinha a
terrível sensação de que não estávamos sozinhos; e me mantinha tão perto de Tonnison que duas vezes chutei seus
calcanhares sem querer, mas ele nada disse. Um minuto se passou, e depois mais outro, e chegamos aos confins da
floresta, finalmente saindo dela e pondo os pés no solo de rochas nuas daquela região. Só então deixei de sentir o medo
que provei enquanto estava na floresta.

Em certo momento, enquanto nos afastávamos, pensamos ter ouvido novamente aquele som de choro distante, e eu disse a mim
mesmo que era o vento --- embora aquele começo de noite não apresentasse brisa alguma.

E então Tonnison começou a falar.

--- Escute --- disse ele, em tom incisivo. --- Eu não passaria a noite naquele lugar nem se me oferecessem todo o dinheiro do
mundo. Há algo horrível\ldots{} diabólico ali. Senti isso de repente, logo depois que você falou. Era como se a
floresta estivesse repleta de coisas terríveis, entende?

--- Sim --- respondi e olhei para trás, para o local, mas ele estava oculto por uma protuberância no solo. --- O livro está
aqui ---  eu disse, colocando a mão na bolsa. --- Você guardou direito? --- indagou Tonnison, ansioso de repente.
--- Sim --- respondi.

--- Talvez possamos descobrir algo nele quando voltarmos para a barraca --- completou ele. --- Melhor nos
apressarmos. Estamos muito longe e eu não quero ficar aqui no escuro.

Só chegamos à barraca duas horas depois; e, sem demora, começamos a preparar uma refeição, pois não havíamos comido nada
desde que almoçamos ao meio"-dia.

Depois do jantar, organizamos as coisas e acendemos nossos cachimbos. E então Tonnison me pediu para tirar o manuscrito
da bolsa. Obedeci e, como não era possível que nós dois lêssemos ao mesmo tempo, ele sugeriu que eu lesse em voz alta.
--- Mas veja bem, não leia pulando partes --- ele advertiu, ciente das minhas propensões.

Se ele soubesse o que aquele livro nos reservava, saberia quanto tal conselho era desnecessário, pelo menos
naquela ocasião. E ali, sentados dentro de nossa pequena barraca, comecei a narrar a estranha história da
Casa entre dois mundos” (pois este era o título do manuscrito). A história está relatada nas páginas
seguintes. 


\clearpage

\chapter{A Planície do Silêncio}

\textsc{Sou um} homem idoso.\footnote{\versal{N.E.}: A partir deste capítulo, lemos o conteúdo do manuscrito encontrado nas ruínas de Kraighten, em notas de rodapé estão os comentários de Hodgson.} Vivo aqui, nesta casa antiga, cercada de jardins enormes e descuidados.

Os aldeões que habitam o descampado distante dizem que sou louco. Dizem isso porque não quero nada com eles. Moro aqui
sozinho com minha velha irmã, que é quem toma conta da casa. Não temos empregados --- eu os detesto. Tenho um amigo, o
meu cão; sim, prefiro meu velho Pepper a todas as outras criaturas. Ele pelo menos me entende --- e sabe que deve me
deixar em paz quando estou de mau humor.

Decidi começar uma espécie de diário; talvez assim eu possa registrar alguns dos pensamentos e sensações que não deve
expressar para mais ninguém. Mas, acima de tudo, desejo fazer algum registro das coisas bizarras que ouvi e vi durante
os muitos anos de solidão nesta casa estranha e antiga.

Durante séculos, esta casa teve certa reputação, uma reputação ruim, e permaneceu inabitada por mais de oitenta
anos, até que eu a comprasse. Assim, consegui comprá"-la por uma quantia ridiculamente baixa.

Não sou supersticioso, mas deixei de negar que coisas de fato acontecem nesta antiga casa --- coisas que não sei
explicar; portanto, preciso me tranquilizar descrevendo todas elas da melhor maneira que conseguir; todavia, se este meu
diário for encontrado depois que eu morrer, os leitores só balançarão a cabeça, incrédulos, e ficarão ainda mais
convencidos de que eu estou louco.

Como é antiga esta casa! Mas sua idade impressiona menos do que sua estrutura singular, diferente e fantástica até o
último detalhe. Predominam pequenas torres e pináculos curvados, com contornos que sugerem labaredas; o corpo da
estrutura tem formato circular.

Já ouvi falar sobre uma velha história que corre entre as gentes daqui, a qual diz que foi o diabo quem construiu este
lugar. Pode ser. Se é verdade ou não, não sei nem me importo em saber, exceto pelo fato de que isso possa
ter contribuído para seu preço irrisório, e para que eu viesse para cá.

Eu já devia morar aqui há uns dez anos quando percebi que vira o suficiente para acreditar nas histórias contadas pelos
vizinhos a respeito desta casa. É verdade que, em pelo menos umas dez ocasiões, já havia visto vagamente coisas que me
deixaram confuso, coisas que talvez eu tivesse mais sentido do que visto. Então, à medida que os anos foram passando e
fui envelhecendo, muitas vezes percebia algo invisível, mas inegavelmente presente, nos aposentos e corredores vazios.
Mas, como afirmei, só depois de muitos anos vi manifestações reais das coisas chamadas sobrenaturais.

Não é Halloween. Se eu estivesse contando uma história somente para entreter, eu a narraria na noite de Halloween; na
verdade, este é um relato fiel das minhas experiências, e eu jamais escreveria só para entreter os outros.
Não. Era depois de meia"-noite, na madrugada do dia vinte e um de janeiro. Eu estava sentado, lendo, como costumo fazer,
no meu estúdio. Pepper dormia perto da minha cadeira.

De repente, as chamas de duas velas ficaram mais fracas e começaram a brilhar com uma luz verde e terrível. Levantei
rapidamente os olhos e, quando o fiz, vi que as luzes adotaram um tom vermelho opaco, fazendo o aposento brilhar como a
luz carmesim do poente, estranha e pesada, projetando, atrás das cadeiras e mesas, sombras que eram mais escuras que a
própria escuridão; em todos os pontos atingidos pela luz, era como se sangue luminoso tivesse sido espalhado pelo
aposento.

Ouvi um gemido baixo de medo vindo do chão e senti algo se enfiando entre meus dois pés. Era Pepper, escondendo"-se
embaixo de meu roupão. Pepper, que geralmente é corajoso feito um leão!

Acho que foi essa atitude do cão que me fez sentir a primeira pontada de medo. Fiquei bastante espantado quando as
luzes ficaram verdes e depois vermelhas, mas atribuí essa mudança à possível entrada momentânea de alguma brisa no
aposento. Todavia, agora percebia que não era esse motivo, pois as chamas ardiam de modo constante, sem sinal de que se
apagariam, como seria caso houvesse alguma mudança na atmosfera.

Fiquei imóvel. Agora me sentia bastante aterrorizado, mas não conseguia pensar em nada melhor a fazer do que esperar.
Durante talvez um minuto, fiquei observando o aposento, ansioso. Então percebi que as chamas começaram a
diminuir, bem lentamente, até virarem pequeninos pontos de fogo vermelho na escuridão, como o brilho de um rubi. Mesmo
assim, continuei sentado, observando; enquanto isso, uma espécie de torpor onírico parecia tomar conta de mim,
eliminando por completo o medo que eu sentira.

Percebi, no outro extremo daquele estúdio antigo e enorme, um leve brilho. O brilho foi crescendo aos poucos, enchendo
o aposento de lampejos de uma luz verde trêmula, e esses lampejos logo morriam e passavam a exibir --- assim como as
chamas das velas --- um tom vermelho profundo que ficava mais forte e iluminava todo o ambiente, inundando"-o de luz de
uma maneira gloriosa e terrível.

A luz vinha da parede oposta e foi ficando cada vez mais forte até que seu fulgor intolerável doeu"-me os olhos, e eu
involuntariamente os fechei. Alguns segundos devem ter se passado até que eu pudesse abri"-los de novo. A primeira
coisa que percebi foi que a luz havia diminuído bastante, a ponto de não me doerem mais os olhos. Assim, à medida que
ela ia ficando ainda mais fraca, percebi de repente que, em vez de estar olhando para o brilho vermelho, eu na
verdade olhava através dele, através da parede oposta.

Gradualmente, à medida que me acostumava à ideia, percebi que estava olhando para uma vasta planície, iluminada pela
mesma luz difusa de poente que iluminava todo o aposento. É quase impossível descrever a imensidão daquela planície. Eu
não conseguia imaginar onde ela terminava. Parecia estender"-se com tamanha infinidade que o olho humano era incapaz de
perceber seus limites. Devagar, os detalhes das áreas mais próximas começaram a ficar mais nítidos, então, quase
num instante, a luz cessou e a visão --- se é que aquilo era uma visão --- desvaneceu e sumiu.

De repente, percebi que eu não estava mais na cadeira. Em vez disso, parecia estar flutuando acima dela, olhando para
baixo, para algo indistinto, debruçado em silêncio. Depois de uma breve pausa, senti um vento gélido e vi que estava do
lado de fora, flutuando na noite como se estivesse numa bolha, subindo na escuridão. Enquanto me movia, uma sensação
gélida pareceu tomar conta de mim e estremeci.

Depois de algum tempo, olhei para um lado e para o outro e só via a escuridão intolerável da noite, interrompida por um
ou outro fulgor de fogo. Continuei sendo levado para cima, para fora. Em determinado momento, olhei para trás e vi a
Terra, uma pequena meia"-lua de luz azul, diminuindo à minha esquerda. Na distância, o Sol, um brilho de chamas brancas,
ardia vividamente contra a escuridão.

Um período indefinido passou"-se. E então, pela última vez, vi a Terra --- um pequeno globo azul radiante, pairando na
eternidade do éter. E ali estava eu, uma frágil partícula de alma, tremulando silencioso contra o vazio, saindo do
azul distante para a vastidão do desconhecido.

Depois de um grande intervalo, eu não via nada em lugar algum. Já havia ultrapassado as estrelas e mergulhava na
enorme escuridão além delas. Durante todo esse tempo, não provava nada além da sensação de leveza e o desconforto do
frio. Agora, porém, a abominável escuridão parecia tomar conta de minha alma e fui pego pelo medo e desespero. O que
aconteceria comigo? Para onde estava indo? Bem no momento em que esses meus pensamentos se formavam, surgiu, contra a
escuridão impalpável que me rodeava, uma bruma com leve tom de sangue. Parecia absurdamente remota, como uma névoa;
todavia, naquele mesmo instante a sensação de opressão diminuiu e eu não me senti mais desesperado.

Aos poucos, essa vermelhidão distante foi ficando mais evidente e maior, até que, à medida que me aproximava dela, ela
se transformava num fulgor gigantesco e sombrio --- opaco, enorme. Eu continuava pairando e logo estava tão perto
dela que ela parecia se estender abaixo de mim, como um grande oceano vermelho"-escuro. Não podia ver muita coisa,
exceto que ela parecia se espalhar infinitamente em todas as direções.

Algum tempo depois, descobri que estava descendo sobre ela; logo mergulhei num grande mar de nuvens escuras, de um tom
avermelhado. Saí devagar delas e avistei, lá embaixo, aquela planície gigantesca, a mesma que vi no meu estúdio,
nesta casa que se encontra de pé à beira do Silêncio.

Logo depois, pousei e fiquei em pé, cercado pela grande vastidão da solidão. O lugar era iluminado por uma luz
poente que lhe dava um ar de desolação indescritível.

À minha direita, no céu, ardia um anel gigante de fogo vermelho, de suas extremidades projetavam"-se labaredas enormes
que dançavam, pulsantes e irregulares. O interior desse anel era negro, negro como as trevas da noite. Compreendi
imediatamente que aquele Sol extraordinário é que causava a iluminação melancólica do local.

Desviando os olhos daquela estranha fonte de luz, olhei mais uma vez para meus arredores. Para onde quer que mirasse,
não via nada além da vastidão monótona, plana e interminável. Não havia sinal de vida em lugar algum, nem mesmo ruínas
de alguma civilização antiga.

Aos poucos, percebi que estava sendo levado adiante, levitando sobre a vasta planície. Continuei a ser impelido durante
o que pareceu ser uma eternidade. Não sentia nenhuma grande impaciência, só curiosidade e muito assombro. A
todo instante percebia ao meu redor a amplitude da enorme planície, e o tempo todo buscava algo que quebrasse sua monotonia, mas
não havia nenhuma mudança na paisagem --- somente a solidão, o silêncio e o deserto.

E então, quase de maneira semiconsciente, percebi que havia uma leve névoa úmida, de tom avermelhado, sobre sua
superfície. Embora tentasse observar com mais cuidado, não soube dizer se era de fato uma névoa, pois ela parecia
mesclar"-se à planície, dando"-lhe o aspecto peculiar de algo irreal, sem substância.

Aos poucos fui me cansando da mesmice de tudo aquilo. Entretanto, só depois de muito tempo é que percebi para onde
estava sendo levado.

Primeiro eu o vi à distância, como se fosse um pequeno morro na superfície da planície. Então, à medida que ia me
aproximando, percebi que havia me enganado: pois agora, em vez de um morro, o que eu via era uma cadeia de montanhas
enormes, cujos picos distantes avultavam na escuridão avermelhada até quase se perderem de vista.


\clearpage

\chapter{A casa na arena}

\textsc{Assim, depois} de algum tempo, cheguei às montanhas. O percurso de minha jornada foi alterado e comecei a me
mover ao longo das bases das montanhas até que, de repente, vi que fora conduzido até uma vasta brecha entre elas.
Continuei a ser conduzido por ali, em velocidade não muito alta. Dos dois lados, erguiam"-se, imponentes, muralhas
enormes e escarpadas, feitas de uma substância semelhante a rocha. Bem mais à frente, discerni um fino veio vermelho,
onde a boca da brecha abria"-se entre os picos inacessíveis. Dentro dele, só a escuridão, profunda e sombria, e um
silêncio de gelar os ossos. Durante algum tempo, continuei a ser impelido para frente e, então vi mais adiante,
finalmente, um brilho de intenso rubor, o que indicava que eu estava próximo da abertura do desfiladeiro.

Um minuto se passou e eu estava já na saída da brecha, olhando para um enorme anfiteatro de montanhas. Todavia, as
montanhas e o esplendor terrível do lugar não me prendiam a atenção, pois fiquei muito desconcertado e maravilhado ao
ver, a uma distância de vários quilômetros, ocupando o centro da arena, uma estrutura gigantesca, aparentemente feita
de jade. Mas não era a construção em si que tanto me deixava admirado, e sim o fato, cada vez mais evidente, de que
aquela estrutura não era em nada diferente, com exceção da cor e do enorme tamanho, da casa onde resido.

Durante algum tempo, fiquei olhando fixamente para ela. Mesmo assim, mal conseguia acreditar no que via. Em minha
mente, uma dúvida surgia, repetindo"-se sem cessar: ``O que isso significa? O que isso significa?'' Eu não conseguia achar
uma resposta, mesmo apelando para a imaginação. Só conseguia sentir assombro e medo. Durante mais algum tempo,
continuei a olhar, percebendo novas características semelhantes que chamavam minha atenção. Por fim, cansado e
extremamente confuso, virei"-me para observar o restante daquele estranho lugar.

Até então eu estivera tão absorto em minha análise da Casa que só havia olhado apressadamente em volta. Agora,
enquanto observava, comecei a perceber que local era aquele. A arena, pois era assim que eu a chamava, parecia ser um
círculo perfeito de cerca de quinze ou vinte quilômetros de diâmetro, e a Casa, como mencionei, ficava no centro.
A superfície do lugar, como a da planície, também tinha uma aparência enevoada, embora não houvesse névoa.

Investigando rapidamente a paisagem, meus olhos percorreram os declives das montanhas em volta. Como eram silenciosas!
Acho que esse silêncio abominável era mais insuportável para mim do que tudo que eu vira ou imaginara até então. Agora,
olhava para os enormes penhascos, imponentes e altivos. Lá no alto, a vermelhidão impalpável dava a tudo uma aparência
disforme.

E então, enquanto observava curioso, um novo terror apoderou"-se de mim, pois, bem ao longe, entre os picos indistintos
à minha direita, vislumbrei um gigantesco vulto negro, como se fosse um gigante. Ele foi ficando maior enquanto eu o
observava. Tinha uma cabeça equina, enorme, com orelhas gigantescas, e parecia fitar constantemente a arena. Algo em
seu porte me deu a impressão de que ele estava ali observando por toda a eternidade --- de que durante eras desconhecidas 
fora o guardião daquele lugar deplorável. Lentamente, os contornos do monstro ficaram mais nítidos e então, de
repente, meu olhar saltou dele para algo mais distante e mais alto entre os penhascos. Durante um minuto, que pareceu
infinito, fiquei olhando fixamente para aquilo, com medo. Era estranho, mas eu estava consciente de algo que não me era
totalmente desconhecido --- como se estivesse me lembrando de algo. A coisa era negra e tinha quatro braços grotescos.
Não conseguia ver suas feições. Em volta de seu pescoço, pude enxergar vários objetos de cor clara. Lentamente,
fui abarcando os detalhes e gelei ao me dar conta de que eram crânios. Mais para baixo, havia outro cinturão que dava
voltas em seu corpo, menos escuro que o tronco. E então, enquanto lutava com minha mente para lembrar o que era aquilo,
a memória veio e eu soube imediatamente que estava olhando para uma monstruosa imagem de Kali, a deusa hindu da morte.

Outras lembranças surgiram de meus dias de estudante. Meu olhar voltou"-se para a enorme Coisa com cabeça de animal. No
mesmo instante, percebi que era o antigo deus egípcio Set, ou Seth, o Destruidor de Almas. Ao perceber isso, logo
comecei a ter vários questionamentos --- ``Dois dos\ldots{}!'' Parei de repente, tentando raciocinar. Coisas além da imaginação
ameaçavam tomar conta de minha mente assustada. Eu não conseguia enxergar muito bem. ``Os antigos deuses da mitologia!''
Tentei compreender o que aquilo significava. Meu olhar saltava de um para o outro. ``Se\ldots{}''

Uma ideia surgiu de repente e me virei, olhando rápido para cima, perscrutando os penhascos sombrios à esquerda. Algo
avultava sob um pico enorme, uma forma cinzenta. Não entendi por que não a vira antes, e então me lembrei de que ainda
não havia olhado para aquela parte. Agora, podia ver com mais clareza. Era, como disse, uma forma cinzenta. Tinha uma
cabeça gigantesca, mas sem olhos. Essa parte de seu rosto não tinha nada.

Agora, percebia que havia outras coisas entre as montanhas. Mais adiante, reclinada sobre uma saliência bem alta,
vislumbrei uma massa lívida, irregular, mórbida. Parecia não ter forma exceto por um rosto meio animalesco e diabólico
que observava o mundo, de maneira repugnante, a partir de algum ponto no meio da massa amorfa. Então vi os outros ---
havia centenas deles. Pareciam emergir das sombras. Muitos reconheci quase que imediatamente como divindades
mitológicas; outros me eram desconhecidos, completamente estranhos, muito além do poder da imaginação humana.

Dos dois lados, eles apareciam cada vez mais. As montanhas estavam repletas de coisas estranhas --- Deuses"-Feras e
Horrores Sobrenaturais, tão terríveis e brutais que está além da possibilidade e da decência descrevê"-los. Eu me
sentia tomado por uma sensação terrível e avassaladora de horror, medo e repugnância; ainda assim, continuava a ter
dúvidas incessantes. Será que havia, afinal de contas, algo de verdadeiro na antiga veneração pagã, algo além da mera
deificação dos homens, dos animais e dos elementos? Essa dúvida tomou conta de mim. Será?

Depois, outra dúvida surgiu, repetindo"-se incessantemente. O que seriam esses Deuses"-Feras e os outros? No começo, só
me pareciam ser Monstros esculpidos, colocados indiscriminadamente entre os cumes inacessíveis e precipícios das
montanhas ao redor. Agora, enquanto eu os observava com mais atenção, minha mente chegava a novas conclusões. Havia
algo neles, uma vitalidade indescritível e silenciosa, que sugeria, para minha consciência cada vez mais desperta, um
estado de vida"-em"-morte --- algo que não era absolutamente a vida como a compreendemos, e sim uma forma não humana de
existência, que pode muito bem ser comparada a um transe imortal, condição na qual era possível imaginar que existissem
por toda a eternidade. ``Imortais!'', foi a palavra que surgiu em meus pensamentos, espontaneamente, e logo comecei a
pensar se era aquela a imortalidade dos deuses.

Então, em meio a minhas dúvidas e pensamentos, algo aconteceu. Até então, eu estivera sob a sombra da saída da grande
fenda entre as montanhas. Agora, contra minha vontade, estava sendo deslocado da semiescuridão e começava a atravessar
lentamente a arena --- indo rumo à Casa. Com isso, todos os pensamentos sobre as Formas extraordinárias acima de mim
desapareceram --- eu só conseguia olhar fixo, com medo, para a estrutura gigantesca para a qual estava sendo
implacavelmente conduzido. Todavia, por mais que eu a perscrutasse com avidez, não descobri nada que não tivesse
visto, então, aos poucos fui me acalmando.

Agora, eu estava quase no meio do caminho entre a Casa e o abismo. Ao meu redor, só havia a brutal solidão do lugar e o
silêncio contínuo. Fui me aproximando da casa em velocidade constante. De repente, algo chamou minha atenção,
algo que surgiu por trás de um dos enormes esteios da Casa e ocupou totalmente meu campo de visão. Era algo gigantesco,
que se movimentava de maneira pesada e estranha, em posição ereta, quase como se fosse um ser humano. Estava quase nu e
tinha um aspecto consideravelmente luminoso. Mas era o rosto que mais atraía minha atenção e me assustava. Era um rosto
com feições suínas.

Em silêncio, fiquei observando atentamente aquela criatura horrível, esquecendo por instantes o meu medo, interessado
em seus movimentos. Ponderoso, ele rodeava a construção, parando em cada janela para espiar lá dentro e
chacoalhar as grades que as protegiam --- assim como protegem as janelas desta casa; sempre que chegava a uma porta,
ele tentava empurrá"-la, manuseando sorrateiramente o trinco. Era evidente que buscava uma maneira de entrar na Casa.

Eu agora estava a menos de quinhentos metros da grande estrutura e ainda assim continuava a ser impelido para frente.
Abruptamente, a Coisa deu meia"-volta e ficou olhando, de maneira repugnante, na minha direção. Abriu a boca e, pela
primeira vez, o silêncio daquele lugar abominável foi rompido por um urro grave e muito alto que me fez sentir um
arrepio ainda maior de medo. Na hora percebi que ele vinha na minha direção, rápido e silencioso. Em
poucos segundos, havia percorrido metade da distância que nos separava. E, ainda assim, eu continuava a ser impelido
em sua direção. Agora, restavam somente uns cem metros entre nós, e a ferocidade brutal naquele rosto gigante deixou"-me
petrificado com a sensação do mais absoluto horror. Talvez eu tenha gritado, tamanho era o meu medo, mas então, bem no
auge do meu desespero, percebi que estava olhando para baixo, para a arena, de uma grande altura, que aumentava
rapidamente. Eu estava subindo e subindo. Em um intervalo de tempo bem curto, já havia alcançado a altitude
de centenas de metros. Lá embaixo, no ponto onde eu a abandonara, estava a repelente criatura de feições suínas. Estava de
quatro, farejando e esfregando o focinho no chão da arena, exatamente como um porco. Em um instante ele ficou de pé,
agarrando o ar com uma expressão de desejo que eu nunca vira na vida.

Continuei a subir. Depois de algum tempo, parecia já estar acima dos picos das montanhas --- flutuando solitário e longe na
atmosfera vermelha. Lá embaixo, a uma distância enorme, estava a arena, que eu mal conseguia enxergar; a
imensa Casa era somente um pequeno ponto verde. A Coisa suína não estava mais visível.

Logo depois, passei por cima das montanhas, sobre a imensa amplidão da planície. Bem ao longe, sobre sua superfície, na
direção do Sol em formato de anel, havia um borrão indistinto. Olhei para ele, indiferente. De certa forma, ele
lembrava a primeira visão que eu tivera do anfiteatro de montanhas.

Com certo receio, voltei os olhos para cima, para o imenso anel de fogo. Como era estranho! Enquanto olhava,
bem do centro escuro saiu uma repentina labareda de fogo, bem vívida. Se comparada ao tamanho do interior
negro, não era nada, mas era, por si só, monstruosa. Com renovado interesse, continuei a observar com atenção, notando
o seu estranho movimento e brilho. Então, num instante, ela toda ficou mais escura e abstrata e desapareceu. Deveras
estupefato, voltei a observar a planície da qual estava sendo erguido e tive uma nova surpresa. A planície,
\textit{tudo} havia desaparecido, e somente um mar de névoa avermelhada estendia"-se a distância, lá embaixo.
Aos poucos, enquanto eu olhava, tudo isso ficou ainda mais remoto e transformou"-se numa misteriosa névoa avermelhada,
indistinta e distante, contra a noite insondável. Depois de algum tempo, até mesmo isso havia desaparecido e, ao meu
redor, só havia as trevas impalpáveis. 


\clearpage

\chapter{A Terra}

\textsc{Assim eu} me encontrava, e somente a memória de que eu sobrevivera na escuridão uma vez antes servia para me manter são.
Um grande tempo se passou --- eras. Então uma única estrela emergiu na escuridão. Era o primeiro dos agrupamentos
remotos do Universo. Logo depois, ela ficou lá atrás, distante, e ao redor de mim brilhava o esplendor das incontáveis
estrelas. Aparentemente, anos se passaram, e só então vi o Sol, um coágulo de chamas. Ao redor dele, pude distinguir
diversos pontos remotos de luz --- os planetas do Sistema Solar. E, assim, vi a Terra de novo, azul e incrivelmente
diminuta. Ela foi ficando maior e seus contornos mais definidos.

Um grande intervalo passou"-se, e então, finalmente, penetrei as sombras do nosso mundo --- mergulhando de cabeça na
escuridão da mais pura noite terrestre. Lá em cima, as velhas constelações e uma lua crescente. Quando me aproximei da
terra, uma penumbra envolveu"-me como se eu estivesse mergulhando numa névoa negra.

Durante algum tempo, de nada sabia. Estava inconsciente. Aos poucos, comecei a perceber um gemido fraco, distante, que
foi ficando mais audível. Uma sensação desesperada de agonia tomou conta de mim. Debati"-me como louco em busca de ar e
tentei gritar. Após alguns instantes, consegui respirar mais facilmente. Percebi que algo lambia a minha mão. Sentia
algo úmido passando sobre meu rosto. Ouvi um arfar e de novo os gemidos. Agora meus ouvidos percebiam que era um
ruído conhecido, e então abri os olhos. Estava escuro, mas a sensação de opressão desaparecera. Eu estava sentado e
algo gemia de um jeito que dava dó e me lambia. Sentia"-me estranho, confuso e, por instinto, tentei afastar a coisa
que me lambia. Minha cabeça estava curiosamente vazia de pensamentos e, naquele momento, eu parecia impossibilitado de
pensar ou agir. E então comecei a me lembrar das coisas, disse “Pepper” em tom baixo.
Recebi como resposta um latido alegre e carícias redobradas e desesperadas.

Após um breve período, senti"-me mais forte e estiquei a mão para pegar os fósforos. Tateei às cegas durante alguns
segundos; minhas mãos encontraram"-nos, acendi um fósforo e olhei em volta, confuso. Ao redor, vi todas as minhas velhas
e conhecidas coisas. E ali fiquei, sentado, cheio de assombro, atordoado, até a chama do fósforo queimar"-me o dedo e eu
deixá"-lo cair; neste momento, uma súbita exclamação de dor e raiva escapou"-me dos lábios e surpreendi"-me com o som de
minha voz.

Depois de alguns instantes, acendi outro fósforo e, tropeçando pelo aposento, acendi as velas. Enquanto eu as acendia,
percebi que não haviam queimado até o fim, e sim que foram apagadas.

Enquanto as chamas cresciam, virei"-me e fiquei observando o estúdio; todavia, não havia nada de incomum; de repente,
senti uma grande irritação. O que havia acontecido? Coloquei as duas mãos na cabeça e tentei lembrar. Ah! A enorme e
silenciosa planície e o Sol de fogo vermelho em formato de anel. Onde estavam? Onde eu os vira? Havia quanto tempo?
Senti"-me atordoado, confuso. Uma ou duas vezes, andei para lá e para cá pelo aposento, com passos incertos. As
lembranças pareciam remotas e só conseguia recordar o que havia testemunhado se fizesse força.

Lembro"-me de praguejar com raiva por me sentir tão perplexo. De repente, fiquei zonzo, cambaleante, e tive de
agarrar"-me à mesa. Durante alguns instantes, apoiei"-me ali, fraco; depois consegui mover"-me de lado até uma cadeira.
Após um breve período, senti"-me um pouco melhor e tentei alcançar o armário onde geralmente deixo conhaque e biscoitos.
Servi"-me um pouco da estimulante bebida e bebi tudo. Depois de pegar um punhado de biscoitos, voltei à minha cadeira e
comecei a devorá"-los com voracidade. Estava vagamente surpreso com minha fome. Era como se eu nada houvesse
comido durante um longo período.

Enquanto comia, meu olhar varria o aposento, absorvendo seus diversos detalhes, ainda buscando, embora quase
inconscientemente, algo tangível que pudesse captar entre os mistérios invisíveis que me rodeavam. Pensei: ``Sem
dúvida deve haver algo\ldots{}'' No mesmo instante, meu olhar parou no relógio do outro lado do estúdio. Parei de comer e
fiquei só olhando. Pois, embora o tique"-taque certamente fosse prova de que o relógio ainda estava funcionando, os
ponteiros indicavam uma hora pouco antes da meia"-noite, mas, como eu bem sabia, já era bem depois de meia"-noite quando
eu testemunhara os primeiros estranhos acontecimentos que acabo de descrever.

Durante talvez um breve momento senti assombro e confusão. Se o horário fosse o mesmo de quando eu havia visto o
relógio, eu teria concluído que os ponteiros pararam mesmo que o mecanismo interno tivesse continuado funcionando, mas
isso, em hipótese alguma, explicaria os ponteiros terem andado para trás. Assim, enquanto revirava o assunto em minha
mente exausta, um pensamento surgiu de chofre: que agora se aproximava a manhã do dia vinte e dois e que eu estivera
inconsciente para o mundo visível durante boa parte das últimas vinte e quatro horas. Esse pensamento ocupou minha
mente durante todo um minuto; depois, voltei a comer. Ainda sentia muita fome.

Durante o café da manhã, na manhã seguinte, casualmente perguntei à minha irmã que dia era e descobri que
minha estimativa estava correta. De fato eu estivera ausente --- pelo menos em espírito --- durante quase um dia e uma
noite.

Minha irmã não fez perguntas, pois aquela não era absolutamente a primeira vez que eu permanecia um dia inteiro, às vezes
dois, trancado em meu estúdio, sempre que ficava bastante absorto em meus livros ou no trabalho.

E desde então os dias se passaram; ainda me pergunto qual o sentido de tudo aquilo que vi naquela noite memorável.
Contudo, sei muito bem que é pouco provável que minha curiosidade seja saciada.


\clearpage

\chapter{A coisa no fosso}

\textsc{Esta casa} é, como eu disse, cercada por um grande terreno de jardins silvestres não cultivados.

Nos fundos, a uma distância de talvez trezentos metros, há uma ravina profunda e escura --- chamada de
``Fosso'' pelos aldeões. No fundo dela corre um riacho de águas lentas, tão encoberto pelas
árvores que mal pode ser visto de cima.

Devo aqui explicar que esse rio tem origem subterrânea e surge de repente do lado leste da ravina, desaparecendo, de
maneira igualmente abrupta, por baixo dos penhascos de sua extremidade, no lado oeste.

Foi alguns meses depois da minha visão da grande planície (se é que fora uma visão) que o Fosso atraiu minha atenção.

Certo dia, eu caminhava pela beira do Fosso ao sul quando, de repente, pedras e xisto desprenderam"-se da face do
penhasco imediatamente abaixo de mim e caíram, com um baque surdo, por entre as árvores. Pude ouvir o som daquilo
caindo nas águas do rio, lá no fundo; depois, o silêncio. Eu teria ignorado por completo esse incidente se Pepper não
tivesse começado a latir desesperadamente no mesmo instante, além de não me obedecer quando eu ordenava que ficasse em
silêncio, o que é algo bastante incomum.

Acreditando que havia alguém ou algo no Fosso, voltei rapidamente para casa em busca de um bastão que me ajudasse na
descida. Quando voltei, Pepper havia parado de latir e estava rosnando e farejando, preocupado, na beirada.

Assobiando para que ele me seguisse, comecei a descer com cuidado. O Fosso deve ter cerca de quarenta metros de
profundidade, e levei um bom tempo, além de precisar estar bastante atento, até chegar ao fundo com segurança.

Já lá embaixo, Pepper e eu começamos a explorar as margens do rio. Estava muito escuro devido às copas das árvores, e
eu me movia atento, olhando sempre ao redor, com o bastão em riste.

Pepper agora estava em silêncio e se mantinha perto de mim o tempo todo. Assim, vasculhamos de um dos lados do rio, sem
nada ver ou ouvir. Depois, atravessamos o rio --- bastava pular --- e retornamos pelo sentido inverso, fustigando o
matagal. 

Talvez estivéssemos na metade do caminho quando, de novo, ouvi o som de pedras caindo do outro lado --- o lado do qual
acabávamos de vir. Uma pedra enorme atravessou com grande estrondo a copa das árvores, atingiu a margem oposta e
rolou para dentro do rio, espirrando enorme quantidade de água em nós. Com isso, Pepper soltou um rosnado grave,
depois parou e ficou de orelha em pé, atento. Também prestei atenção para ver se ouvia algo.

Um segundo depois, um grito muito alto, metade humano, metade suíno, soou por entre as árvores, parecendo vir do
meio do penhasco ao sul. Veio um som semelhante como resposta do fundo do Fosso. Com isso, Pepper deu um latido curto e
agudo e, saltando o pequeno rio, desapareceu em meio aos arbustos.

Logo em seguida, seus latidos ficaram mais graves e numerosos e, entremeados com eles, ouvi ruídos desconexos e
confusos. Tudo isso cessou e, no silêncio que se seguiu, veio um berro de agonia semi"-humano. Quase imediatamente,
Pepper soltou um comprido uivo de dor, os arbustos agitaram"-se com violência e ele surgiu correndo, com o rabo entre as
pernas, enquanto corria, olhava de vez em quando para trás. Quando chegou perto de mim, vi que estava sangrando:
havia em seu flanco o que parecia ser um grande ferimento feito por uma garra, deixando quase visíveis suas
costelas.

Ao ver Pepper tão ferido, fui tomado por profunda raiva e, girando meu cajado, comecei a correr para os arbustos de
onde Pepper saíra. Assim que penetrei a mata, acredito ter ouvido um ruído de respiração. Logo em seguida, cheguei a uma
pequena clareira a tempo de ver algo de cor pálida e lívida desaparecer por entre os arbustos do outro lado. Com um
grito, pus"-me a correr atrás dela, mas, por mais que fustigasse os arbustos e os investigasse com o bastão, não vi e nem
ouvi mais nada; assim sendo, voltei para Pepper. Ali, depois de lavar seu ferimento no rio, amarrei um lenço molhado ao
redor de seu corpo; depois disso, subimos pela parede da ravina e voltamos à luz do dia.

Quando cheguei à casa, minha irmã perguntou o que acontecera com Pepper, e eu lhe disse que ele havia brigado com um
gato selvagem; eu ouvira dizer que havia muitos deles por ali.

Achei melhor não contar a ela o que havia de fato se sucedido; todavia, eu mesmo mal sabia, na verdade. Mas de algo
tinha certeza: o que vira correr para os arbustos não era nenhum gato selvagem. Era grande demais e tinha,
pelo que pude observar, pele semelhante à de um porco selvagem, só que de um tom esbranquiçado, morto, enfermiço. E a
coisa havia corrido de pé, ou quase, sobre as patas traseiras, com movimentos semelhantes ao de um ser humano. Tudo
isso percebi nos breves instantes em que a vi e, para ser sincero, agora sentia grande temor, além de curiosidade,
enquanto pensava no assunto.

Foi de manhã que tal evento ocorrera.

Então se sucedeu que, depois do jantar, enquanto eu estava sentado lendo, ergui o olhar de repente e vi algo me
espiando sobre o parapeito da janela, só os olhos e orelhas visíveis.

``Meu Deus, um porco!'', eu disse e fiquei de pé. Assim, podia ver a coisa melhor, mas não era um porco --- sabe Deus o
que era aquilo. Lembrava pouco a horripilante Coisa que havia me perseguido da enorme arena. Tinha boca e mandíbula
grotescamente humanos, mas sem um queixo definido. O nariz era alongado como um focinho; era isso que,
com aqueles olhos pequenos e as orelhas estranhas, dava"-lhe uma aparência bem parecida com a de um
porco. Quase não tinha testa, e todo o rosto apresentava uma cor esbranquiçada e doentia.

Durante talvez um minuto fiquei ali parado, fitando aquela coisa, com uma sensação cada vez maior de repulsa e um pouco
de medo. A boca se remexia, fazendo sons sem sentido, e, em determinado momento, soltou um grunhido meio suíno.
Acho que eram os olhos que prendiam minha atenção; por vezes, pareciam brilhar com uma inteligência terrivelmente
humana e desviavam"-se do meu rosto para perscrutar os detalhes do aposento, como se meu olhar o perturbasse.

Parecia estar apoiando as duas mãos, semelhantes a garras, sobre o parapeito. Essas garras, ao contrário do rosto, eram
de um tom marrom"-terroso e lembravam vagamente mãos humanas, já que tinham quatro dedos e um polegar, mas todos tinham
membranas até a primeira junta, semelhantes aos pés de um pato. Também tinha unhas, mas tão compridas e potentes que
mais pareciam as garras de uma águia.

Como eu disse, senti um pouco de medo, mas era uma sensação quase impessoal. Talvez consiga explicar melhor se
disser que era mais uma sensação de repulsa, como a que alguém sente ao entrar em contato com algo vil e sobre"-humano,
algo diabólico, pertencente a um patamar de existência até então nunca imaginado.

Não posso afirmar com certeza que percebi todos esses diversos detalhes daquele ser selvagem naquela ocasião. Acho que
me lembrei deles depois, como se tivessem ficado marcados em minha mente. Imaginei mais do que vi, enquanto olhava para
a coisa, e os detalhes materiais ficaram mais claros depois.

Durante talvez um minuto fiquei ali, fitando a criatura, então, à medida que fui me acalmando, resisti à vaga
sensação de espanto que me assolava e dei um passo em direção à janela. Assim que fiz isso, a coisa esquivou"-se e
sumiu. Corri apressado até a porta, olhei lá fora, mas só via o mato e os arbustos emaranhados. 

Voltei correndo para casa e, pegando minha arma, saí em busca da coisa pelos jardins. Enquanto fazia isso, pensei se a
coisa que tinha acabado de ver era a mesma que vira de relance pela manhã. Estava inclinado a achar que sim.

Teria levado Pepper comigo, mas julguei ser melhor dar"-lhe tempo para se curar. Além disso, se a criatura que eu
acabara de ver era, como eu imaginava, o seu antagonista da manhã, era pouco provável que Pepper fosse de muita
utilidade.

Comecei uma busca metódica. Estava determinado, se possível, a achar e dar cabo do ser suíno. Aquele sim era, pelo
menos, um Horror material!

No começo, procurei, cauteloso, pensando no ferimento de Pepper, mas, à medida que as horas se passavam e não havia
sinal de vida naqueles jardins enormes e solitários, fiquei menos apreensivo. Quase achava que me sentiria aliviado se
fosse ver a coisa. Qualquer coisa era melhor do que aquele silêncio e da sensação sempre constante de que a criatura
podia estar à espreita em cada arbusto por onde eu passava. Depois, deixei de pensar no perigo a ponto de entrar de uma
vez nos arbustos, vasculhando"-os com o cano da arma enquanto caminhava.

Às vezes, eu gritava, mas só os ecos respondiam. Pensei que isso talvez fosse assustar a criatura, ou fazer com que 
ela aparecesse, mas só consegui que minha irmã Mary saísse e perguntasse qual era o problema. Eu lhe disse
que vira o gato selvagem que havia machucado Pepper e que estava tentando fazê"-lo sair de seu esconderijo no meio do
mato. Ela não pareceu totalmente satisfeita com a resposta e voltou para dentro de casa, com uma expressão de dúvida no
rosto. Fiquei pensando se ela não teria visto ou desconfiado de algo. Durante todo o resto da tarde, continuei minha
busca, ansioso. Achava que não conseguiria dormir com aquele ser animalesco à espreita, no meio da mata, e, todavia,
quando a noite caiu, eu ainda não tinha visto nada. Então, ao dar meia"-volta, rumando para casa, ouvi um som
ininteligível e curto entre os arbustos à minha direita. Virei no mesmo instante e, mirando rápido, disparei na direção
do som. Logo em seguida, ouvi algo saindo correndo por entre os arbustos. Movia"-se rapidamente e, num instante, já
estava fora do meu raio de audição. Depois de alguns passos, deixei de ir atrás dela, percebendo o quanto seria fútil
naquela penumbra cada vez mais profunda; sentindo uma estranha tristeza, entrei em casa.

Naquela noite, depois que minha irmã foi dormir, percorri a casa verificando todas as janelas e portas do térreo para
ver se estavam bem trancadas. Tal precaução não era muito necessária no caso das janelas, já que todas as do andar de
baixo tinham grades fortes; mas no caso das portas --- e havia cinco delas --- foi uma sábia decisão, uma vez que nenhuma estava
trancada.

Depois de trancá"-las, fui para meu estúdio; mas, por algum motivo, daquela vez, o lugar me deixou apreensivo; parecia
tão grande e cheio de ecos\ldots{} Durante algum tempo, tentei ler, mas, finalmente, achando a tarefa impossível, levei meu
livro até a cozinha, onde um grande fogo ardia na lareira, e fiquei ali sentado.

Acredito que estava lendo havia umas duas horas quando, de repente, ouvi um som que me fez abaixar o livro e escutar com
atenção. Era o som de algo se esfregando e remexendo na porta dos fundos. Em determinado momento, a porta rangeu alto,
como se algo estivesse exercendo força sobre ela. Durante esses breves momentos, senti um terror indescritível como
jamais imaginara ser possível. Minhas mãos tremiam; um suor frio tomou conta de meu corpo e estremeci com violência.

Aos poucos me acalmei. Os movimentos furtivos do lado de fora cessaram.

Durante uma hora, fiquei ali sentado, em silêncio, alerta. De repente, a sensação de medo de novo tomou
conta de mim. Senti"-me como imagino que um animal se sinta sob o olhar de uma cobra. No entanto, não ouvia nada. Mesmo
assim, não havia dúvida de que algum poder inexplicável estava em ação.

Aos poucos, quase imperceptivelmente, ouvi algo --- um som que se transformou num leve murmúrio. Cresceu rápido e
virou um coro abafado e terrível de gritos bestiais. Pareciam vir das profundezas da terra.

Ouvi um baque surdo e percebi, de maneira um tanto perplexa e lenta, que havia deixado cair o livro. Depois disso,
fiquei apenas sentado, e foi assim que a luz do dia me encontrou, quando se insinuou fracamente através das janelas
altas e gradeadas da enorme cozinha.

Com a luz da alvorada, a sensação de torpor e medo me abandonaram, passei a ter mais controle de meus sentidos.

Em seguida, peguei meu livro e fui em silêncio até a porta, atento aos ruídos. Nenhum som interrompia a quietude
indiferente. Durante alguns minutos, fiquei ali em pé; então, com cuidado, puxei o ferrolho aos poucos e, com a
porta aberta, espiei lá fora.                                                                                          

Minha cautela não era necessária. Não havia nada para se ver ali, exceto a paisagem cinzenta das horríveis árvores e 
arbustos emaranhados que se estendia até a plantação distante.

Com um arrepio, fechei a porta e rumei, em silêncio, para a cama. 


\clearpage

\chapter{As Coisas suínas}

\textsc{Era o} começo da noite, uma semana depois. Minha irmã estava sentada no jardim, tricotando. Eu andava para lá e para cá,
lendo. Minha arma estava apoiada numa parede da casa, pois, desde a aparição daquela coisa estranha nos jardins, achei
melhor precaver"-me. Todavia, durante toda a semana, nada ocorreu de alarmante, fosse visão ou ruído; assim, pude
recordar calmamente o incidente, ainda que com uma sensação de estranhamento e total assombro.

Eu estava, como disse, andando para lá e para cá, bastante absorto em meu livro. De repente, ouvi um baque a distância,
da direção do Fosso. Virei com um movimento rápido e vi uma imensa coluna de poeira erguendo"-se no ar da noite.

Minha irmã estava de pé, depois de dar um grito agudo de surpresa e medo.

Disse"-lhe para ficar onde estava, peguei minha arma e corri até o Fosso. Quando me aproximei, ouvi um som
surdo, como um ronco, que logo cresceu e transformou"-se num urro, interrompido por baques mais graves, e do Fosso
surgiu mais uma nuvem de poeira.

O ruído cessou, mas ainda assim a poeira subia numa nuvem turbulenta.

Aproximei"-me da beirada e olhei para baixo, mas não conseguia ver nada além de nuvens de poeira girando para lá e para
cá. O ar ficou tão cheio de partículas de pó que elas me cegavam e sufocavam; finalmente, precisei fugir da poeira para
poder respirar.

Gradualmente, a matéria suspensa começou a cair, pairando como uma camada sobre a boca do Fosso. Eu só podia suspeitar
o que acontecera.

Que ocorrera algum tipo de deslizamento de terra, não havia muita dúvida, mas a causa eu não conseguia imaginar e,
mesmo assim, até mesmo naquela ocasião, eu imaginava certas coisas, pois me recordei das pedras caindo e daquela Coisa
no fundo do Fosso, mas, nos primeiros minutos de confusão, não consegui chegar a uma conclusão sobre a causa natural
da catástrofe.

Aos poucos, a poeira dissipou"-se até que, por fim, pude aproximar"-me da beirada e olhar para baixo.

Durante algum tempo, fiquei olhando, sem poder fazer nada além disso, tentando enxergar através da névoa. A princípio,
não conseguia ver nada. Depois, enquanto perscrutava, notei algo lá embaixo, à minha esquerda, movimentando"-se.
Fiquei olhando com atenção e, logo depois, discerni outro, e depois mais outro --- três formas indistintas que pareciam
estar escalando as paredes do Fosso. Eu só podia vê"-las indistintamente. Enquanto olhava e tentava compreender o que
eram, ouvi uma sequência de pedras caindo em algum ponto à minha direita. Olhei para o outro lado, mas nada vi. Inclinei"-me
para frente e fiquei observando o Fosso logo abaixo de onde eu estava; e o que vi foi uma horrível cara branca e suína
a uma distância de poucos metros de meus pés. Abaixo dela, pude discernir várias outras. Quando a Coisa me viu, deu um
grito repentino, grosseiro, que recebeu respostas de todos os cantos do Fosso. Com isso, fui tomado pelo horror e pelo
medo e, inclinando"-me, disparei a arma bem na cara da coisa. Ela imediatamente sumiu, com um ruído de terra e pedras
caindo.

Houve um silêncio momentâneo ao qual eu, provavelmente, devo minha vida, pois, durante esse intervalo, ouvi o ruído de
muitos pés e, virando"-me de repente, vi uma tropa de criaturas correndo na minha direção. Instantaneamente, ergui minha
arma e disparei na direção da que estava mais à frente, que caiu de cabeça, com um uivo terrível. Então me virei para
correr. A mais de metade da distância entre o Fosso e a casa, vi minha irmã --- ela vinha na minha direção. Não conseguia
ver claramente o seu rosto, já que a noite havia caído, mas ouvi o medo em sua voz quando ela berrou, perguntando por
que eu estava atirando.

``Corra!'', gritei para ela. ``Corra, se quiser continuar viva!''

Sem demora, ela deu meia"-volta e fugiu --- levantando a saia do vestido com as duas mãos. Enquanto eu corria atrás dela,
olhei de soslaio para trás. As bestas selvagens corriam nas patas traseiras --- às vezes, de quatro.

Acho que foi o terror em minha voz que fez com que Mary corresse tanto, pois estou convencido de que ela ainda não
havia visto as criaturas infernais que nos perseguiam.

E continuamos a correr, minha irmã à frente.

A cada instante, o som dos passos cada vez mais próximos indicava que as bestas logo nos alcançariam. Felizmente,
tenho, de certa maneira, uma vida ativa. Mas, parece, o esforço despendido com aquela corrida começava a pesar
muito.

Podia ver lá na frente a porta dos fundos --- por sorte, estava aberta. Eu estava a uns cinco metros de Mary agora,
respirando com dificuldade. Então algo tocou meu ombro. Virei a cabeça com violência, rápido, e vi um daqueles rostos
monstruosos e pálidos perto do meu. Uma das criaturas conseguiu ultrapassar as outras e quase havia me pegado. Até no
mesmo instante em que me virava, ela tentou me agarrar de novo. Fazendo um derradeiro esforço, pulei para o lado e,
girando a arma pelo cano, atingi a cabeça da horrível criatura com toda a força. A Coisa caiu com um gemido quase
humano.

Até mesmo esta breve demora quase foi suficiente para que o resto daquelas feras me alcançasse; assim, sem me demorar
nem mais um instante, virei e corri em direção à porta.

Chegando lá, joguei"-me dentro de casa; virei rápido, bati a porta e fechei o ferrolho bem no instante em que uma das
primeiras criaturas chocou"-se de repente contra ela.

Minha irmã estava sentada, resfolegando, numa cadeira. Parecia prestes a desmaiar, mas eu não tinha tempo para
acudi"-la. Precisava ter certeza de que todas as portas estavam trancadas. Ainda bem que estavam. A que saía do meu
estúdio para o jardim foi a última que verifiquei. Eu acabara de me certificar que estava trancada quando pensei ter
ouvido um barulho do lado de fora. Fiquei em completo em silêncio, ouvindo. Sim! Agora podia ouvir distintamente o
som de sussurros, algo rascante sobre as tábuas da porta, o ruído de algo raspando, arranhando. Era evidente que algumas
das feras tateavam a porta com suas garras para descobrir se havia algum meio de entrar na casa.

Que as criaturas logo tivessem encontrado a porta era, para mim, prova de sua capacidade de raciocínio. Com isso tive
certeza de que não deveria considerá"-las, em hipótese alguma, meros animais. Já sentira algo parecido, quando aquela
primeira Coisa ficou espiando pela minha janela. Eu a descrevera como sobre"-humana, sabendo de maneira quase instintiva
que a criatura era algo diferente de uma fera qualquer. Era algo além do humano, não num bom sentido, e sim vil,
hostil a tudo que há de grandeza e bondade no ser humano. Numa descrição sucinta, algo inteligente, mas desumano. O mero
ato de pensar nessas criaturas me enchia de repulsa.

Agora, lembrava"-me de minha irmã; fui até o armário, peguei uma garrafa de conhaque e uma taça. Desci para a cozinha
com os dois, levando também uma vela acesa. Ela não estava sentada na cadeira: havia caído de bruços no chão.

Com muito cuidado, virei"-a e ergui um pouco sua cabeça. Verti um pouco do conhaque entre seus lábios. Após algum tempo,
ela se mexeu de leve. Depois, engasgou e tossiu diversas vezes e abriu os olhos. De um modo vago, como se não percebesse
nada, ela olhou para mim. Então seus olhos se fecharam, lentamente, e eu lhe dei um pouco mais de conhaque. Durante
talvez mais um minuto, ela ficou ali deitada, em silêncio, respirando rápido. De repente, seus olhos abriram"-se e,
pelo que pude ver, suas pupilas estavam dilatadas, como se o medo tivesse voltado com sua consciência.
Com um movimento tão inesperado que me fez levar um susto e saltar para trás, ela se sentou. Percebendo que ela
parecia atordoada, estiquei o braço para equilibrá"-la. Com isso, ela soltou um grito bem alto pôs"-se rápida de
pé saiu correndo.

Durante alguns instantes, fiquei ali --- ajoelhado, segurando a garrafa de conhaque. Estava totalmente
confuso, estupefato.

Será que ela estava com medo de mim? Não! Por que estaria? Só podia concluir que estava muito nervosa, num momento de
insanidade. Lá de cima, ouvi o barulho alto de uma porta batendo, então soube que ela buscava refúgio em seu quarto.
Coloquei a garrafa sobre a mesa. Um ruído que vinha da porta dos fundos chamou minha atenção. Fui até lá e fiquei
ouvindo. A porta parecia estremecer, como se algumas das criaturas lutassem contra ela em silêncio, mas era uma
porta muito robusta e pesada para ser movida facilmente.

Dos jardins, vinha um som contínuo, cada vez mais alto. Quem ouvisse aquilo por acaso poderia pensar que eram os
grunhidos e gritos de um rebanho de porcos. Mas, ali de pé, percebi que havia sentido em todos aqueles ruídos suínos.
Aos poucos, fui conseguindo traçar uma semelhança com a fala de seres humanos --- incoerente e arrastada, como se cada
articulação fosse difícil; todavia, estava convencido de que não eram apenas ruídos aleatórios, e sim uma troca rápida
de ideias.

A essa hora, os corredores da casa já estavam bastante escuros e deles vinham os ruídos e gemidos tão comuns numa casa
velha ao anoitecer. Sem dúvida, isso ocorre porque tudo está mais silencioso e temos mais tempo livre para ouvir.
Também pode haver algo de verdadeiro na teoria de que a rápida mudança de temperatura, quando o Sol se põe, afeta a
estrutura da casa de alguma maneira --- fazendo com que ela contraia e se acomode, por assim dizer, para dormir. Pode ser
que isso seja verdade, mas, naquela noite específica, ficaria feliz se não ouvisse tantos ruídos estranhos. A mim
parecia que cada estalo e rangido era uma daquelas Coisas aproximando"-se pelos corredores escuros; mesmo que eu
soubesse, lá no fundo, que isso não era possível, pois eu mesmo havia verificado se todas as portas estavam trancadas.

No entanto, gradualmente, esses ruídos passaram a me irritar de tal maneira que, mesmo que fosse só para punir minha
covardia, achei que era melhor fazer de novo uma ronda pelo porão e, caso houvesse algo ali, enfrentá"-lo. Depois eu
subiria para o meu estúdio, pois eu sabia que dormir estava fora de questão, com a casa cercada de criaturas metade
feras e metade outra coisa, totalmente diabólicas.

Peguei a lamparina da cozinha do gancho e fui descendo de porão em porão, de aposento em aposento; através de
despensas e salas de carvão, entrando nos mil e um becos e esconderijos que formam o subsolo da velha casa. Então,
ciente de que eu já tinha estado em cada canto e fenda grande o suficiente para ocultar algo que tivesse tamanho
considerável, dirigi"-me às escadas.

Com o pé no primeiro degrau, parei de repente. Pareceu"-me ter ouvido algo, aparentemente vindo da despensa, que fica à
esquerda da escada. Foi um dos primeiros lugares que vasculhei; contudo, tinha certeza de que meus sentidos não me
enganavam. Agora, meus nervos estavam à flor da pele e, sem a menor hesitação, fui até a porta, erguendo a lamparina
acima da cabeça. Olhando rápido, vi que não havia nada no lugar, exceto as pesadas lajes de pedra, apoiadas em pilares
feitos de tijolos, e já estava saindo dali, convencido de que havia me enganado, quando, ao virar, minha luz foi
refletida por dois pontos luminosos do lado de fora da janela, mais para cima. Durante alguns instantes, fiquei ali
parado, olhando. E aí eles se moveram --- deslocando"-se lentos, emanando alternadamente um brilho verde e vermelho;
ou pelo menos assim me parecia. E aí percebi que eram olhos.

Aos poucos, discerni o contorno obscuro de uma das Coisas. Ela parecia agarrada às grades da janela e sua
pose sugeria que ela a estivesse escalando. Cheguei perto da janela e ergui a lamparina mais alto. Não havia por que ter medo da
criatura; as grades eram fortes e havia pouco risco de ela conseguir movê"-las. Contudo, de repente, apesar de saber que
a fera selvagem não conseguiria me fazer mal, senti de novo a horrível sensação de medo que havia me assaltado
naquela noite, uma semana antes. Era a mesma sensação de impotência, de extremo pavor. Percebi, vagamente, que os olhos
da criatura fitavam os meus com um olhar fixo, penetrante. Tentei dar meia"-volta e ir embora, mas não consegui. Agora,
era como se eu visse a janela através de uma névoa. Depois, achei que outros olhos apareceram para espiar, e depois
mais outros; até que uma galáxia inteira de órbitas perversas me observava atenta, parecendo me hipnotizar.

Comecei a me sentir zonzo, minha cabeça latejava com força. Senti uma dor intensa na mão esquerda. A dor na mão foi
ficando mais forte e forçou"-me a voltar a atenção para ela. Fazendo um esforço tremendo, olhei para baixo e, com isso,
o encantamento em que eu me encontrava se dissipou. Percebi que eu havia, durante o meu transe, colocado,
inconsciente, a mão no vidro quente da lamparina e que queimei severamente a mão. De novo olhei para a janela. A
aparência de névoa havia desaparecido e, agora, via que à janela estavam dezenas de rostos bestiais. Com um repentino
acesso de fúria, ergui a lamparina e acertei"-a em cheio na janela. Ela atingiu o vidro, quebrando uma vidraça, e passou
entre duas barras da grade, caindo no jardim e espalhando óleo quente. Ouvi vários gritos de dor bem altos e, enquanto
meus olhos se acostumaram à escuridão, descobri que as criaturas haviam saído da janela.

Recompondo"-me, tateei em busca da porta e, tendo"-a encontrado, subi a escada, tropeçando em cada degrau. Sentia"-me
tonto, como se tivesse recebido uma pancada na cabeça. Ao mesmo tempo, minha mão ardia muito, e eu sentia uma raiva
surda e intensa contra aquelas Coisas.

Ao chegar ao meu estúdio, acendi as velas. Enquanto queimavam, seus raios de luz refletiram"-se no cavalete de armas na
parede. Ao ver aquilo, lembrei"-me de que tinha um poder, o qual, como eu havia provado antes, parecia tão fatal para
aqueles monstros quanto para animais mais ordinários, e decidi tomar a ofensiva. 

Primeiro, fiz uma atadura na mão, pois a dor rapidamente ficou intolerável. Depois disso as coisas pareceram mais
fáceis e fui até o outro lado do estúdio, até o cavalete de rifles. Ali, selecionei um rifle pesado --- uma arma antiga,
com a qual já tinha experiência; depois de pegar munição, subi para uma das pequenas torres que coroam a casa.

Dali, dei"-me conta de que não conseguia enxergar nada. Nos jardins, eu via um borrão escuro de sombras --- talvez um
pouco mais negras no ponto onde ficavam as árvores. E isso era tudo, e eu sabia que era inútil atirar naquela
escuridão. A única coisa a fazer era esperar a Lua surgir, e só então eu poderia executar algumas daquelas feras.

Enquanto isso, fiquei sentado, em silêncio, com os ouvidos atentos. Os jardins agora estavam relativamente em silêncio,
e eu só ouvia um ou outro grunhido ou grito. Não gostava daquele silêncio; obrigava"-me a imaginar que tipo de diabrura as
criaturas estariam aprontando. Duas vezes deixei a torre e passeei pela casa, mas tudo estava em silêncio.

Uma vez, ouvi um ruído vindo do Fosso, como se mais terra tivesse caído. Depois disso, houve uma comoção entre os que
estavam nos jardins, o que durou uns quinze minutos. Depois, a algazarra arrefeceu e tudo ficou em silêncio de novo.

Cerca de uma hora depois, a luz da Lua surgia acima do horizonte distante. De onde eu estava sentado, podia vê"-la por
entre as árvores, mas foi só quando ela se ergueu bem acima delas que fui capaz de discernir detalhes nos jardins lá
embaixo. Mesmo então, não consegui enxergar nenhuma daquelas feras até que, ao me inclinar para frente, vi várias
delas deitadas contra a parede da casa. Não conseguia ver o que estavam fazendo. Mas era uma chance boa demais para ser
ignorada; mirei e atirei direto em um deles lá embaixo. Ouvi um grito agudo e, quando a fumaça da pólvora se
dissipou, vi que ele agora estava de costas no chão, mexendo"-se com dificuldade. Depois, ficou parado. Os outros haviam
desaparecido.

Logo depois, ouvi um grito alto e agudo, vindo do Fosso. Umas cem vozes responderam de todos os cantos do
jardim. Isso me deu uma noção do número de criaturas e comecei a desconfiar que a situação toda estava ficando ainda
mais séria do que eu imaginara.

Sentado ali, em silêncio, alerta, veio"-me um pensamento: por que tudo isso? O que eram essas coisas? O que aquilo
significava? E então meus pensamentos voltaram"-se para aquela visão (embora mesmo agora eu duvide que era de fato uma
visão) da Planície do Silêncio. O que aquilo significava?, pensei. E aquela Coisa na arena? Argh! Por fim, pensei na
casa que havia visto naquele lugar tão distante. Aquela casa, tão similar à minha em cada detalhe de sua estrutura
externa que parecia ser uma cópia desta, ou então esta daquela. Eu nunca havia parado para pensar nisso\ldots{}

Bem naquele instante, ouvi outro guincho comprido, vindo do Fosso, seguido, um segundo depois, por alguns mais curtos.
No mesmo momento, o jardim foi tomado de gritos em resposta. Fiquei em pé e espiei por cima do parapeito. À
luz da Lua, parecia que os arbustos estavam vivos. Agitavam"-se para lá e para cá, como se fossem castigados por um
vento forte e irregular, ao mesmo tempo, o som do farfalhar contínuo e o ruído de pisadas chegavam até mim. Diversas
vezes, vi a luz da Lua brilhar sobre vultos brancos que corriam por entre os arbustos, e atirei duas vezes. Na segunda
vez, em resposta ao meu tiro, ouvi um breve grito de dor.

Um minuto depois, os jardins ficaram em silêncio. Do Fosso, vinha uma arenga rouca e grave de conversa suína. Em certos
momentos, gritos raivosos cortavam o ar, respondidos por inúmeros grunhidos. Ocorreu"-me que talvez estivessem em algum
tipo de reunião, talvez para discutir o problema de entrar na casa. Também achei que pareciam muito irritados, talvez
por causa das mortes que eu causara.

Ocorreu"-me que agora seria uma boa hora para fazer uma vistoria final de nossas defesas. Rápido, ocupei"-me da
tarefa, visitando todo o porão e examinando cada uma das portas. Felizmente, todas eram, como a dos fundos, feitas de
carvalho sólido, com pregos de ferro. Depois, subi para o meu estúdio. Era a porta dali que me preocupava. Ela é
de um feitio visivelmente mais moderno que as outras e, por mais que seja uma peça robusta, não tem o peso e a força
das outras.

Aqui, devo explicar que existe um pequeno gramado suspenso deste lado da casa, sobre o qual a porta se abre --- e as
janelas do estúdio ficam ocultas devido a isso. Todas as outras entradas --- com exceção do grande portão, que nunca é
aberto --- ficam no andar mais baixo.


\clearpage

\chapter{O ataque}

\textsc{Fiquei algum} tempo pensando como reforçar a porta do estúdio. Finalmente, desci até a cozinha e, com certa dificuldade,
levei para o estúdio vários pedaços pesados de madeira. Apoiei"-os inclinados na porta, contra o chão, e os prendi com
pregos na parte de cima e de baixo. Trabalhei muito durante meia hora, até me dar por satisfeito.

Sentindo"-me mais tranquilo, peguei meu casaco, que havia deixado de lado, e passei a cuidar de alguns assuntos
antes de voltar à torre. Enquanto estava ocupado, ouvi o som de algo remexendo na porta, experimentando o trinco. Em
silêncio, esperei. Logo ouvi várias das criaturas lá fora. Grunhiam umas para as outras, baixinho. Depois, durante um
minuto, houve silêncio. De repente, ouvi um grunhido baixo e rápido e a porta rangeu sob uma pressão tremenda. Ela
teria se despedaçado se não fossem os suportes que eu havia colocado. A força cessou tão rápido quanto iniciou e houve
mais conversa.

Logo depois, uma das Coisas guinchou baixinho e ouvi o som de outros se aproximando. Houve uma pequena confabulação;
de novo, silêncio; percebi que haviam chamado vários outros para ajudar. Sentindo que agora era o momento ideal,
fiquei preparado, mirando o rifle. Se a porta cedesse, eu pelo menos mataria o máximo deles que pudesse.

Outra vez ouvi o sinal em tom baixo e, mais uma vez, a porta rangeu sob uma força imensa. Durante talvez um minuto,
eles continuaram a pressão; fiquei aguardando, ansioso, o tempo todo pensando que a porta seria derrubada. Mas não: os
apoios eram firmes e a tentativa provou"-se infrutífera. De novo ouvi aquela conversa horrível de grunhidos e,
enquanto ela durou, acredito que ouvi mais feras chegando.

Depois de uma longa discussão, durante a qual a porta foi sacudida várias vezes, eles voltaram a ficar em silêncio, e
eu sabia que tentariam arrebentá"-la uma terceira vez. Eu estava quase desesperado. Os apoios aguentaram tudo o que
podiam nos dois ataques anteriores, mas eu morria de medo de que uma terceira tentativa fosse demais para eles.

Naquele momento, como uma inspiração, um pensamento surgiu em minha mente inquieta. Instantaneamente, pois não havia
tempo a perder, saí correndo do aposento e subi degrau após degrau. Desta vez, não fui para uma das torres, e sim para
o telhado plano. Ali, corri até o parapeito que o cerca e olhei para baixo. Assim que fiz isso, ouvi o grunhido curto
que era o sinal e, mesmo dali de cima, pude ouvir o ranger da porta sob o ataque.

Não havia um instante a perder e, debruçando"-me, mirei rápido e atirei. O som do tiro foi alto e, quase mesclado a
ele, veio o som abafado da bala atingindo o alvo. Lá de baixo, ouvi um berro agudo e a porta parou de ranger.
Enquanto deixava de apoiar meu peso sobre o parapeito, um enorme pedaço de pedra da beirada caiu com um baque enorme
entre a desorganizada turba de feras. Vários gritos horríveis romperam o ar noturno e então ouvi o som de
pés correndo. Com cuidado espiei lá embaixo. Sob o luar, podia ver a grande pedra do parapeito caída bem em frente
à porta. Pensei ter visto algo embaixo dela --- várias coisas brancas, mas não tinha certeza.

Passaram"-se alguns minutos.

Enquanto olhava, vi algo surgir da sombra da casa. Era uma das Coisas. Foi até a pedra, em silêncio, e agachou"-se.
Eu não conseguia ver o que fazia. Depois de alguns instantes, ficou de pé. Tinha algo entre as garras, que colocou na
boca e dela arrancava pedaços\ldots{}

Durante alguns instantes, não me dei conta do que ela fazia. Então, aos poucos, compreendi. A Coisa agora agachava"-se
de novo. Era horrível. Comecei a carregar o rifle. Quando olhei outra vez, o monstro estava empurrando a
pedra --- movendo"-a para o lado. Apoiei o rifle no parapeito e apertei o gatilho. A fera caiu de cara no chão e ficou
chutando o ar debilmente. 

Quase que ao mesmo tempo, com o tiro, ouvi outro som --- de vidro quebrando. Demorando"-me só o tempo necessário
para recarregar a arma, saí correndo do telhado e desci os dois primeiros lances de escada.

Ali, fiz uma pausa para tentar ouvir algo. E ouvi outro ruído de vidro caindo. Parecia vir do andar de baixo. Desci,
ansioso, os degraus e, guiado pelo barulho de uma das janelas de correr, cheguei à porta de um dos quartos vazios
nos fundos da casa. Abri"-a de uma só vez. O quarto quase não estava iluminado pelo luar; grande parte da claridade era
obscurecida por vultos que se moviam na janela. Mesmo enquanto eu estava ali de pé, um dos vultos conseguiu pular a
janela para dentro do quarto. Erguendo a arma, atirei sem mirar --- enchendo o quarto com um estrondo ensurdecedor.
Quando a fumaça dissipou"-se, vi que o quarto estava vazio e a janela sem nada. A luz agora penetrava melhor. O ar da
noite entrou, frio, por entre as vidraças quebradas. Lá embaixo, na noite, podia ouvir um gemido baixinho e um
burburinho confuso de vozes suínas.

Fiquei contra a parede de um dos lados da janela, recarreguei e fiquei esperando. Logo depois, ouvi um som rascante. De
onde estava, à sombra, podia ver sem ser visto.

O som estava cada vez mais próximo, então vi algo subir o peitoril e agarrar"-se ao caixilho da janela. Agarrou um
pedaço da madeira e, agora, eu podia ver que era uma mão e um braço. Segundos depois, o rosto de uma das criaturas
suínas despontou. Antes mesmo que eu pudesse usar o meu rifle ou fazer algo, ouvi o som agudo de algo se
partindo, e o caixilho cedeu sob o peso da Coisa. No segundo seguinte, o baque surdo da coisa esborrachando"-se e a
gritaria indicavam que ela havia caído no chão. Esperando com fervor que ela tivesse morrido, fui até a janela. A
Lua estava oculta sob uma nuvem, então eu não conseguia enxergar nada, mas o murmúrio constante e sem sentido, logo
abaixo de onde eu estava, indicava que havia ainda muito mais feras por perto.

Ali de pé, olhando para baixo, fiquei imaginando como seria possível que as criaturas pudessem ter escalado tão alto;
afinal, a parede era relativamente lisa e a distância até o chão deveria ser de pelo menos uns vinte e cinco metros.

Enquanto estava inclinado na janela, olhando, vi que algo indistinto cruzava a grande sombra cinzenta da casa,
como se fosse uma linha negra. Passava pela janela à esquerda, a uma distância de meio metro. Lembrei: era uma calha
que havia sido colocada ali alguns anos antes para levar embora a água da chuva. Eu havia me esquecido dela. Agora
entendia como as criaturas conseguiram chegar até a janela. Enquanto pensava na solução, ouvi um leve som rascante de
arranhões, então soube que outra fera estava subindo. Esperei alguns instantes, depois me debrucei na janela e tateei
a calha. Para minha agradável surpresa, descobri que estava bem frouxa, e consegui, usando o cano do rifle como um
pé de cabra, levantá"-la da parede. Fiz isso rapidamente. Segurando"-a com as duas mãos, arranquei toda a
calha e a atirei lá embaixo, no jardim --- com a Coisa ainda agarrando"-se a ela.

Durante mais alguns minutos, fiquei ali esperando, ouvindo, mas, depois do primeiro grito, não ouvi nada. Agora sabia
que não havia mais motivo para temer um ataque deste lado da casa. Eu havia eliminado o único meio de alcançar a janela
e, como nenhuma das outras janelas tinha calhas adjacentes para incitar os monstros a escalá"-las, comecei a me sentir mais
confiante para escapar deles.

Saindo do quarto, dirigi"-me ao estúdio. Estava ansioso para ver como a porta havia suportado o ataque. Ao entrar,
acendi duas das velas e olhei para a porta. Um dos grandes apoios de madeira havia saído do lugar e, daquele lado, a
porta estava empenada para dentro uns quinze centímetros.

Foi bastante oportuno ter conseguido expulsar as feras bem naquele momento! E a pedra do parapeito! Tentei
imaginar como poderia tê"-la deslocado. Eu não havia percebido que ela estava solta quando atirei; quando me ergui,
ela havia escorregado embaixo de mim\ldots{} Refleti que devia a expulsão das feras mais à sua oportuna queda do que a meu
rifle. Era evidente que as criaturas não haviam retornado desde a queda da pedra, mas quem saberia dizer quanto tempo
ficariam longe?

Imediatamente, pus"-me a consertar a porta --- trabalhando muito, sem descanso. Primeiro, desci até o porão e,
vasculhando, achei diversas tábuas pesadas de carvalho. Voltei com elas para o estúdio e, depois de remover os apoios
de madeira, encostei as tábuas contra a porta. Bati pregos bem fundo nela.

Assim, a deixei mais resistente do que nunca, pois agora ela estava firme com o reforço das tábuas, e eu estava
convencido de que seria capaz de resistir a uma pressão ainda mais forte, sem ceder.

Depois disso, acendi a lamparina que havia trazido da cozinha e desci para dar uma olhada nas janelas mais baixas.

Agora que eu tivera um exemplo da força que as criaturas possuíam, sentia"-me bastante ansioso em relação às
janelas do andar térreo --- apesar de suas fortes grades.

Fui primeiro para a despensa, com uma lembrança vívida da minha primeira aventura ali. O lugar estava frio e o vento
que passava pelo vidro quebrado cantava de maneira lúgubre. Com exceção do ar generalizado de desorganização, o lugar
estava exatamente como eu o havia deixado na noite anterior. Indo até a janela, examinei as grades com atenção,
percebendo que eram bem grossas, o que me reconfortava. Mesmo assim, ao observar com mais precisão, pareceu"-me que a
barra do meio estava ligeiramente torta, mas era um desvio insignificante, e pode ser que estivesse assim havia anos. Eu
nunca prestara muita atenção nelas até então.

Enfiei a mão na janela quebrada e sacudi a barra. Firme como uma rocha. Talvez as criaturas tivessem tentado entortá"-la
e, ao descobrir que não tinham força suficiente, desistiram. Depois disso, fui até cada uma das janelas, examinando"-as
atentamente, mas não vi nada que indicasse tentativa de invasão. Feita minha vistoria, voltei para o estúdio e me servi
de um pouco de conhaque. Por fim fui até a torre, para ficar de sentinela.


\clearpage

\chapter{Depois do ataque}

\textsc{Agora era} por volta de três da manhã e o céu a leste começava a empalidecer com o nascer do dia, que raiou
lentamente; sob sua luz, varri com o olhar os jardins, ansioso, mas não havia sinal algum das feras. Debrucei"-me e
olhei para o pé da parede da casa, para ver se o corpo da Coisa em que eu havia atirado na noite anterior ainda estava
lá. Não estava. Imaginei que os outros monstros o tivessem tirado dali durante a noite.

Depois, desci para o telhado e fui até o vazio onde antes estivera a pedra. Olhei para baixo. Sim, lá estava a pedra,
exatamente como eu a havia visto da última vez, mas não havia sinal de nada embaixo dela, e também não conseguia ver as
criaturas que ela matara com sua queda. Era evidente que elas também tinham sido levadas embora. Dei meia"-volta e desci para o
estúdio. Chegando lá, sentei, sentindo"-me muito cansado. Totalmente exausto. Agora o dia estava bastante claro, muito
embora os raios de sol ainda não estivessem quentes. Um relógio anunciou as quatro da manhã.

Acordei assustado e olhei apressado ao redor. O relógio no canto indicava que eram três horas. Já era de tarde. Devia
ter dormido durante quase onze horas.

Com um movimento súbito, inclinei"-me para frente, sentado na cadeira, e prestei atenção. A casa estava no mais absoluto
silêncio. Fiquei de pé e bocejei. Eu ainda me sentia extremamente cansado e sentei de novo, tentando imaginar o que
teria me acordado.

Devem ter sido as batidas do relógio, concluí logo depois; eu voltava a adormecer quando um ruído repentino me trouxe outra
vez à vida. Era o som de uma passada, como se uma pessoa estivesse com cuidado vindo pelo corredor na direção do
meu estúdio. Num átimo eu já estava de pé, agarrando o rifle. Sem fazer ruído, esperei. Será que as criaturas haviam
conseguido entrar durante meu sono? Enquanto me fazia essa pergunta, os passos chegaram até a minha porta, ficaram em
silêncio por alguns instantes e continuaram a andar pelo corredor. Fui em silêncio até a porta, na ponta dos
pés, e espiei. Senti enorme alívio, igual ao de um criminoso poupado da pena de morte: era minha irmã. Ela estava indo
em direção à escada.

Saí para o corredor e estava prestes a chamá"-la quando me dei conta de que era muito estranho ela passar
pela minha porta de maneira tão silenciosa. Fiquei intrigado e, por breve instante, veio"-me o pensamento de que
não era ela, e sim uma nova assombração da casa. Mas vi de relance sua velha anágua e o pensamento foi"-se tão rápido
quanto havia chegado, e quase ri de mim mesmo. Não havia como não reconhecer aquela velha peça de roupa. Todavia,
tentei imaginar o que ela estaria fazendo; lembrando"-me do seu estado mental no dia anterior, achei melhor segui"-la em
silêncio --- tomando cuidado para não assustá"-la --- para ver o que ela faria. Se ela se comportasse de maneira racional,
tudo estava bem; se não, deveria tomar medidas para contê"-la. Não poderia correr riscos desnecessários com o perigo que
nos ameaçava.

Rapidamente cheguei ao fim da escada e esperei alguns instantes. Então ouvi um som que me deu um sobressalto
absurdo: ferrolhos sendo abertos. A minha insensata irmã estava de fato destrancando a porta dos fundos.

Alcancei"-a quando sua mão estava no último ferrolho. Ela não havia me visto e, quando deu por si, eu já segurava seu
braço. Ela ergueu os olhos rapidamente, tal qual um animal assustado, e gritou alto.

``Ora, Mary!'', eu disse, com firmeza. “O que significa esse absurdo todo? Não me diga que não compreende o perigo, que
está tentando arriscar nossas vidas dessa maneira!”

A isso, ela nada respondeu; só tremia com violência, arfando e soluçando, como se estivesse no limite do medo.

Durante alguns minutos, tentei convencê"-la, ressaltando a necessidade de termos cuidado e pedindo que tivesse coragem.
Não havia muito a temer agora, expliquei --- e tentava eu mesmo acreditar que isso era verdade ---, ela deveria ser
sensata e tentar não sair de casa por alguns dias.

Finalmente, desisti, desesperado. Não adiantava falar com ela, que, obviamente, não estava em si naquele momento. Por
fim, disse a ela que seria melhor que ela fosse para seu quarto se não conseguia se comportar de maneira racional.

Ainda assim, ela não me obedeceu. Então, sem demora, peguei"-a nos braços e a carreguei para o quarto. No começo, ela
gritou feito louca, mas, quando cheguei à escada, ela já havia voltado a tremer em silêncio.

Chegando a seu quarto, deitei"-a na cama. Ela ficou prostrada, bastante quieta, sem chorar ou falar nada --- só sentindo
verdadeiros calafrios de medo. Peguei uma manta em uma cadeira ali perto e a cobri. Não podia fazer mais nada por ela;
assim, fui ter com Pepper, deitado numa grande cesta. Minha irmã havia ficado com ele desde seu ferimento, para cuidar
dele, já que seu estado parecia mais grave do que eu havia imaginado; fiquei satisfeito ao perceber que, apesar de
seu atribulado estado mental, ela havia cuidado muito bem do meu velho cão. Inclinei"-me e falei com ele e, como
resposta, ele lambeu debilmente minha mão. Estava muito enfermo para fazer mais.

Depois, fui até a cama, inclinei"-me sobre minha irmã e perguntei como ela estava, ela, porém, só estremeceu ainda mais e,
por mais que isso me doesse, eu precisava admitir que minha presença só parecia piorar seu estado.

Assim, deixei"-a --- trancando a porta e colocando a chave no bolso. Parecia"-me a única coisa a fazer.

Fiquei o restante do dia entre a torre e o meu estúdio. Para me alimentar, passei o dia à base do pão inteiro que
trouxera da despensa e de vinho tinto.

Que dia longo e cansativo foi aquele. Se pelo menos eu pudesse sair aos jardins, como costumo fazer, teria me dado por
satisfeito, mas ficar preso naquela casa silenciosa, sem nenhuma companhia exceto a de uma mulher louca e um cão
doente, era suficiente para dar nos nervos da pessoa mais resistente. E no mato emaranhado que cercava a casa estavam à
espreita --- pelo menos assim eu suspeitava --- aquelas infernais criaturas suínas, à espera de uma oportunidade. Que homem
já se encontrou em apuros semelhantes?

Uma vez, à tarde e depois, fui visitar minha irmã. Na segunda vez, ela estava cuidando de Pepper, mas, ao me
aproximar, ela delicadamente foi para o outro canto do quarto, fazendo um gesto de medo que me entristeceu bastante.
Pobrezinha! O medo dela cortava"-me o coração, e eu não queria perturbá"-la sem necessidade. Confiava que melhoraria
depois de alguns dias; enquanto isso, eu nada podia fazer; julguei que ainda era necessário --- por pior que isso fosse ---
mantê"-la confinada em seu quarto. Mas uma coisa me deu alento: ela comera parte da refeição que eu havia levado para
ela na minha primeira visita.

E, assim, passou"-se o dia.

À medida que a noite caía, o ar foi esfriando e comecei a me preparar para passar uma segunda noite na torre --- levando
dois rifles a mais e um pesado casaco. Carreguei os rifles e deixei"-os lado a lado com o outro, já que tencionava
aquecer a noite das criaturas que poderiam surgir naquele período. Tinha bastante munição e desejava dar às feras
tamanha lição que elas aprenderiam quanto era inútil tentar entrar na casa.

Depois disso, fiz mais uma ronda, examinando especialmente os apoios que seguravam a porta do estúdio. Sentindo que já
havia feito tudo que estava a meu alcance para garantir nossa segurança, voltei à torre; no caminho, fiz uma visita a
Pepper e minha irmã. Pepper estava dormindo, mas acordou quando entrei e abanou o rabo ao me ver. Achei que ele
parecia um pouco melhor. Minha irmã estava deitada na cama mas não sei precisar se estava dormindo ou não; assim,
deixei"-os.

Chegando à torre, tentei acomodar"-me ali com o máximo de conforto que as circunstâncias permitiam e preparei"-me para
ficar de sentinela durante a noite. Aos poucos a escuridão chegou e logo os detalhes dos jardins mesclaram"-se às
sombras. Durante as primeiras horas, fiquei sentado, alerta, atento a qualquer som que pudesse indicar que havia algo
se mexendo lá embaixo. Estava escuro demais para que meus olhos tivessem alguma utilidade.

As horas passaram lentas, sem que nada de anormal acontecesse. A Lua surgiu, revelando os jardins, aparentemente
vazios e em silêncio. E assim foi durante toda a noite, sem perturbação ou ruído.

Perto do amanhecer, comecei a me sentir frio e rígido devido à minha longa vigília; o silêncio contínuo por parte das
criaturas também estava me perturbando sobremaneira. Eu desconfiava daquela quietude e preferiria mil vezes que elas
atacassem abertamente a casa. Pois pelo menos assim eu saberia o grau do perigo e poderia enfrentá"-lo, mas esperar
daquele jeito, durante uma noite inteira, imaginando todo tipo de diabruras desconhecidas, era algo capaz de colocar em
prova a sanidade de qualquer um. Algumas vezes, imaginei que tivessem sumido de vez, mas não conseguia, lá no fundo,
acreditar que isso fosse verdade.


\clearpage

\chapter{No porão}

\textsc{Finalmente,} por estar cansado e com frio, e com a preocupação que me afligia, resolvi caminhar pela casa; primeiro
visitei o estúdio em busca de uma taça de conhaque para me aquecer. Enquanto estava lá, examinei a porta
com atenção, ela estava exatamente como eu a havia deixado na noite anterior.

A luz do dia acabava de surgir quando saí da torre, mas ainda estava escuro demais dentro de casa para poder enxergar
sem alguma luz, então levei uma das velas do estúdio comigo para fazer minha ronda. Quando terminei de revistar o
térreo, a luz do dia penetrava débil pelas barras das janelas. Minha vistoria nada revelou de novo. Tudo parecia
estar em ordem e eu estava prestes a apagar a vela quando me veio a ideia de dar mais uma olhada no porão. Se me
recordava corretamente, eu não visitava o lugar desde minha apressada busca na noite do ataque.

Durante talvez uns trinta segundos, hesitei. De bom grado deixaria de lado aquela tarefa --- como, acredito, qualquer
homem deixaria ---, pois, de todos os enormes e pavorosos aposentos que existem nesta casa, o porão é o maior e mais
estranho. O lugar é uma enorme caverna soturna que a luz do dia não alcança. Todavia, não podia me esquivar da tarefa.
Achei que desistir seria a mais pura covardia. Além disso, disse a mim mesmo, o porão era, na verdade, o lugar em que
seria menos provável encontrar algo perigoso, já que só é possível entrar nele através de uma pesada porta de carvalho
cuja chave sempre tenho comigo.

É em uma das partes menores do porão que guardo meus vinhos; uma reentrância escura aos pés da escada que leva ao
porão; raramente avancei para além dela. Na verdade, exceto para procurar as coisas que já mencionei, duvido que já
tenha de fato visitado o porão.

Enquanto destrancava a enorme porta, no topo da escada, hesitei durante alguns instantes, ansioso devido ao estranho
odor de ambiente abandonado que me assaltava as narinas. Então, erguendo o cano da arma à minha frente, desci,
devagar, para a escuridão daquela região subterrânea.

Chegando ao fim da escada, parei durante alguns instantes e fiquei ouvindo. Tudo estava em silêncio, com exceção do
ruído distante de uma goteira, que caía, gota após gota, em algum ponto à minha esquerda. Ali parado, notei como a
chama da vela não se movia; não dançava ou cambaleava nem um pouco, de tão parado que era o ar ali dentro.

Sem fazer ruído, fui de aposento a aposento do porão. Tinha vaga lembrança de como estavam dispostos. Não me lembrava
muito bem deles quando da minha primeira busca. Só recordava a sucessão de diversas câmaras grandes, e uma maior
que as outras, cujo teto era sustentado por pilastras; fora isso, minha memória era incerta, e a lembrança predominante
era uma sensação generalizada de frio, escuridão e sombras. Agora, contudo, era diferente, pois, apesar de nervoso, eu
já havia me recomposto o suficiente para poder olhar em volta e observar a estrutura e o tamanho das câmaras que
penetrava.

Claro que, com a quantidade de luz emitida pela minha vela, não era possível examinar cada lugar demoradamente, mas
pude perceber, enquanto caminhava, que as paredes pareciam ser construídas com extrema precisão e acabamento; e, aqui e
ali, uma pilastra imensa erguia"-se para apoiar o teto abobadado.

Assim, finalmente cheguei à grande câmara do porão de que me lembrava. Vai"-se a ela passando por uma enorme entrada
em arco, na qual notei entalhes estranhos e fantásticos que lançavam misteriosas sombras sob a luz da vela. Ali,
examinando"-os, pensativo, ocorreu"-me quanto era estranho que eu tivesse tão pouca familiaridade com minha própria
casa. Mas isso pode ser facilmente perdoado ao levar em conta a dimensão desta antiga construção e o fato de que só eu
e minha velha irmã moramos nela, ocupando alguns poucos aposentos de acordo com nossa vontade.

Segurando alto a vela, entrei na grande câmara do porão e, mantendo"-me à direita, caminhei devagar até chegar ao
fim. Andava sem fazer ruído, olhando tudo com atenção. Mas, pelo menos até onde a luz alcançava, nada vi de peculiar.

Ao chegar ao fim, virei para a esquerda e, ainda próximo à parede, continuei até atravessar toda a enorme câmara.
Enquanto andava, notei que o chão era feito de rocha sólida, em alguns lugares coberto por um bolor úmido, em outros
descoberto, ou quase, exceto por uma leve camada de poeira cinza claro.

Eu havia parado na entrada. Mas, então, dei meia"-volta e fui até o centro da câmara, passando por entre as pilastras e
olhando para um lado e para outro enquanto caminhava. Quando estava mais ou menos na metade do caminho, bati o pé contra
algo que emitiu um som metálico. Agachando"-me rápido, segurei baixo a vela e vi que o objeto que eu havia chutado
era uma enorme argola metálica. Agachei"-me mais, limpei o pó ao redor e descobri que a argola estava presa a
um pesado alçapão, enegrecido pelo tempo.

Sentindo"-me animado com a descoberta, tentando imaginar aonde ele poderia levar, coloquei minha arma no chão e,
enfiando a vela no guarda"-mato, peguei a argola com as duas mãos e puxei. O alçapão rangeu alto --- o som ecoando de modo
vago naquele lugar imenso --- e abriu"-se com dificuldade.

Apoiando a beirada do alçapão no joelho, peguei a vela e segurei"-a dentro da abertura, movendo"-a para a direita e para
a esquerda, mas não consegui enxergar nada. Fiquei confuso e surpreso. Não havia sinal algum de degraus, nem mesmo
parecia que algum dia existiram. Nada exceto a escuridão vazia. Era como se eu estivesse olhando para um fosso sem
fundo e sem laterais. E então, enquanto eu olhava, totalmente perplexo, pareci ouvir, bem lá no fundo, como se viesse
das profundezas mais abissais, um som que era um leve sussurro. Inclinei a cabeça rapidamente, aproximando"-me da
abertura, e ouvi com atenção. Talvez fosse loucura minha, mas poderia jurar que ouvi um riso abafado, baixinho, que foi
aumentando e se transformou numa gargalhada horrível, distante e obscura. Assustado, dei um salto para trás, deixando o
alçapão cair com um barulho alto e oco que encheu o lugar de ecos. Mesmo então eu ainda parecia ouvir aquela risada de
escárnio, tão sugestiva, mas isso eu sabia que devia ser minha imaginação. O som que eu ouvira era baixo demais para
atravessar o alçapão pesado.

Durante um minuto inteiro, fiquei ali, tremendo --- olhando para trás e para frente, nervoso, mas o grande porão
continuava silencioso como um túmulo e, gradualmente, deixei de sentir medo. Com a mente mais calma, fiquei de novo
curioso para saber aonde aquele alçapão levava, mas não tinha coragem suficiente para fazer uma nova investigação.
Todavia, sabia que precisava bloquear aquele alçapão, e fiz isso colocando sobre ele grandes silhares, que eu vira
quando da minha investigação da parede leste. 

Depois de uma última investigação do restante do lugar, voltei pelo mesmo caminho, atravessando as câmaras do
porão, chegando até a escada e finalmente alcançando a luz do dia, sentindo enorme sensação de alívio por ter
completado a tarefa.


\clearpage

\chapter{A espera}

\textsc{O sol} agora estava quente, brilhando intenso, um maravilhoso contraste com o porão escuro e lúgubre, e foi com uma
sensação de relativa leveza que subi à torre para observar os jardins. De lá, descobri que tudo estava tranquilo e,
depois de alguns minutos, desci para o quarto de Mary.

Lá, após bater e receber uma resposta, destranquei a porta. Minha irmã estava sentada, tranquila, na cama, como se
estivesse me esperando. Parecia ter voltado a si e não fez nenhuma tentativa de se afastar quando me aproximei, mas
percebi que observava com atenção meu rosto, ansiosa, como se em dúvida; parecia parcialmente convencida de que não
havia por que ter medo de mim.

Às minhas perguntas sobre como ela estava se sentindo, ela respondeu que estava com fome, o que era esperado, e que
gostaria de descer para preparar o café da manhã, caso eu não me importasse. Fiquei alguns instantes refletindo se
seria seguro deixá"-la sair. Por fim, disse"-lhe que poderia ir, sob a condição de que me prometesse não tentar sair
de casa nem tentar abrir nenhuma das portas que levavam para fora. Quando mencionei as portas, uma expressão súbita
de medo despontou em seu rosto; ela nada disse além de me prometer o que eu havia lhe pedido e saiu do
quarto em silêncio.

Fui até o outro lado do quarto e me aproximei de Pepper. Ele havia acordado quando entrei, mas, exceto por um pequeno
latido de prazer e um ligeiro abanar do rabo, tinha ficado quieto. Agora, enquanto eu o acariciava, ele tentou ficar de
pé e conseguiu, mas logo caiu de lado com um pequeno uivo de dor.

Falei com ele e disse"-lhe para ficar quieto. Sentia"-me bastante satisfeito com seu progresso e também com a disposição
naturalmente bondosa de minha irmã, por cuidar tão bem dele, apesar de seu atribulado estado mental. Depois de algum
tempo, deixei"-o e desci para meu estúdio.

Logo Mary apareceu, trazendo uma bandeja com café da manhã fumegante. Quando ela entrou, vi que seu olhar demorou"-se
nos apoios que sustentavam a porta do estúdio; contraiu os lábios e acho que ficou um pouco pálida, mas isso foi tudo.
Depois de colocar a bandeja perto de meu cotovelo, ela estava saindo do aposento em silêncio quando eu a chamei. Ela
veio até mim, talvez um pouco tímida, como se estivesse assustada, e notei que sua mão segurava nervosa o avental.

``Ora, Mary'', eu disse. ``Não fique triste! Parece que as coisas estão melhores. Não vi nenhuma daquelas criaturas desde
o começo da manhã de ontem.''

Ela olhou para mim de um jeito curiosamente perplexo, como se não compreendesse. Então a inteligência de novo
transpareceu em seus olhos, assim como o medo, mas nada disse além de um murmúrio ininteligível de consentimento.
Depois disso, fiquei em silêncio; era evidente que qualquer menção às Coisas suínas era bem mais do que seu atual
estado de nervos poderia suportar.

Terminei o café da manhã e subi até a torre. Ali, durante grande parte do dia, vigiei sem cessar os
jardins. Umas duas vezes, desci até o subsolo da casa para ver como minha irmã estava. Todas as vezes, encontrei"-a
tranquila, curiosamente submissa. De fato, da última vez, ela até mesmo criou coragem para falar comigo, de livre e
espontânea vontade, sobre algum assunto da casa que precisava de atenção. Por mais que ela fizesse isso com uma timidez
quase extrema, recebi o ato com felicidade, já que eram as primeiras palavras que ela pronunciava de vontade própria
desde o crítico momento em que eu a peguei destrancando a porta de trás, prestes a se entregar às feras à espreita.
Fiquei pensando se ela tinha consciência do que tentara fazer, do perigo que aquilo representava, mas abstive"-me de
fazer perguntas, acreditando ser melhor deixá"-la em paz.

Naquela noite, dormi numa cama; a primeira vez em duas noites. Pela manhã, acordei cedo e caminhei pela casa. Tudo
estava como deveria estar, então subi para a torre, para observar os jardins. E lá, também, encontrei tudo
perfeitamente tranquilo.

Durante o café da manhã, quando vi Mary, fiquei tão satisfeito ao perceber que ela havia recuperado o
autocontrole a ponto de me cumprimentar de maneira natural. Ela dizia coisas sensatas, com voz calma; só
evitava qualquer menção aos últimos dois dias. Fiz sua vontade, tentando não levar a conversa nessa
direção.

Mais cedo, naquela manhã, eu fora visitar Pepper. Ele se recuperava rápido; imaginei que certamente estaria de pé em um
ou dois dias. Antes de sair da mesa do café da manhã, fiz referência a seu progresso. Na curta conversa que se seguiu,
fiquei surpreso ao compreender, pelos comentários de minha irmã, que ela ainda acreditava que o ferimento de Pepper
havia sido causado por um gato selvagem. Quase senti vergonha por enganá"-la. Mas eu havia contado a mentira para evitar
que ela se assustasse. Também eu estivera certo de que ela sabia a verdade, depois, quando aquelas feras atacaram a
casa.

Durante o dia, mantive"-me alerta; fiquei grande parte do tempo na torre, assim como no dia anterior, mas não havia
sinal das criaturas suínas, nem ruído. Diversas vezes pensei que as Coisas haviam finalmente ido embora, mas, até
então, eu me recusava a acreditar nisso a sério; agora, contudo, começava a achar que havia motivos para ter esperança.
Logo seriam três dias desde a última vez que eu vira aquelas Coisas; mesmo assim, sabia que precisava ter bastante
cautela. Considerando"-se o que eu já conhecia, aquele demorado silêncio poderia ser uma artimanha para tentar me fazer
sair da casa --- talvez direto para os braços dos monstros. Só de pensar em tal eventualidade já me deixava circunspecto.

E foi assim que o quarto, quinto e sexto dias passaram"-se, tranquilos; não fiz nenhuma tentativa de sair da casa.

No sexto dia, tive o prazer de ver Pepper de pé mais uma vez; por mais que ainda estivesse muito fraco, ele conseguiu
me fazer companhia durante todo o dia.


\clearpage

\chapter{A investigação dos jardins}

\textsc{Como o} tempo passava devagar! E não havia nada que indicasse que alguma daquelas feras ainda infestasse os jardins.

Foi no nono dia que finalmente decidi arriscar"-me, se é que havia algum risco, e sair. Com isso em mente, carreguei uma
das espingardas com afinco --- escolhendo"-a por ser mais letal do que um rifle para atirar de perto; depois de
averiguar o terreno lá de cima da torre, chamei Pepper para me seguir e desci. 

Na porta, devo confessar que hesitei. A ideia de que algo pudesse estar me esperando entre aqueles arbustos escuros não
encorajava minha resolução, mas essa hesitação durou apenas alguns segundos e logo eu puxava os ferrolhos e estava do
lado de fora, à porta.

Pepper veio atrás, parando na soleira para farejar, desconfiado; seu focinho subia e descia na ombreira da porta, como
se sentisse algum cheiro. De repente, ele deu meia"-volta e começou a correr para lá e para cá, em círculos e
semicírculos, ao redor da porta; e, finalmente, voltou à soleira. Ali, começou a investigar.

Até então, eu só ficara observando o cão, mas o tempo todo parcialmente atento ao emaranhado de plantas dos jardins que
se alastravam ao redor. Depois, fui até ele e, abaixando"-me, examinei a superfície da porta que ele farejava. Vi que
uma malha de arranhões entrecruzados cobria a madeira num emaranhado. Além disso, percebi que as próprias ombreiras
das portas apresentavam mordidas em determinados pontos. Fora isso, não descobri mais nada; assim, pondo"-me de pé,
comecei a fazer uma ronda pela parede externa da casa.

Pepper, assim que me afastei, saiu da porta e correu adiante, ainda encostando o nariz no chão e farejando enquanto
caminhava. Às vezes, ele parava para investigar. Em um ponto, poderia ser um buraco de bala no caminho ou, talvez, uma
mancha de pólvora; logo adiante, poderia ser um torrão de grama revirado, ou um pedaço de gramado amassado, mas, fora
essas minúcias, ele nada encontrou. Eu o observava com atenção enquanto ele andava, e não percebi nenhum receio em seu
comportamento que indicasse que ele sentia a proximidade das criaturas. Com isso, tive certeza de que aquelas odiosas
Coisas não estavam no jardim, pelo menos por enquanto. Não seria possível enganar Pepper facilmente, e era um alívio
perceber que ele saberia e me alertaria a tempo caso houvesse algum perigo.

Quando cheguei ao local onde eu havia atirado naquela primeira criatura, parei e investiguei com cuidado, mas não
vi nada. De lá, fui até o ponto onde a pedra do parapeito havia caído. Ela estava de lado, aparentemente no mesmo lugar
de quando eu havia atirado na fera que tentava movê"-la. A uns sessenta centímetros à direita da borda mais próxima,
havia uma grande depressão no chão, que era o lugar que ela havia atingido. A outra extremidade ainda estava com a
reentrância meio para fora e meio enterrada. Aproximando"-me, observei mais ainda a pedra. Era um pedaço enorme! E
aquela criatura a havia movido, sozinha, tentando alcançar o que estava embaixo.

Contornei a pedra até a outra extremidade. Ali, descobri que era possível ver embaixo dela a uma distância de uns
sessenta centímetros. Mesmo assim, não vi nenhuma das criaturas atingidas e fiquei bastante surpreso com isso.
Supus, como disse, que os restos mortais tivessem sido retirados, mas não imaginava que tivessem feito isso de
maneira tão perfeita a ponto de não deixar sinal algum embaixo da pedra que indicasse seu trágico destino. Várias
daquelas feras foram atingidas pela pedra, com tamanha força que talvez tenham ficado enterradas; agora, não havia
um único vestígio delas --- nem mesmo manchas de sangue.

Senti"-me mais confuso do que nunca enquanto revirava o assunto em minha mente; não conseguia pensar numa explicação
plausível; finalmente, desisti, pensando que aquilo era mais uma das muitas coisas que não têm explicação.

Dali, transferi minha atenção para a porta do estúdio. Agora, podia ver ainda mais claramente os efeitos da força
enorme que ela havia suportado; fiquei surpreso que, mesmo com os apoios, tivesse aguentado tão bem os ataques. Não
havia marcas de golpes --- na verdade, nenhum golpe fora desferido ---, mas a porta havia sido arrancada das dobradiças por
uma força enorme e sorrateira. Uma coisa que observei me deixou profundamente abalado: a ponta de um dos apoios
atravessou uma das almofadas da porta. Isso, por si só, já era suficiente para demonstrar a enorme força que as
criaturas precisaram fazer para arrombar a porta, e quanto quase o conseguiram.

Saindo dali, continuei meu passeio ao redor da casa, sem descobrir nada de muito interessante; com exceção dos fundos,
onde encontrei, entre o mato alto debaixo da janela quebrada, o pedaço de cano que eu arrancara da parede.

Depois, voltei para a casa e, recolocando o ferrolho na porta, subi à torre. Passei a tarde ali, lendo e olhando de vez
em quando para o jardim. Estava determinado, caso a noite passasse tranquila, a ir até o Fosso no dia seguinte.
Talvez eu descobrisse algo sobre o que acontecera. O dia se passou e a noite veio e se foi exatamente como as noites
anteriores.

Quando me levantei, já era de manhã, um dia bonito e claro; decidi colocar meu plano em ação. Durante o café da manhã,
refleti sobre aquilo com cuidado; depois, fui para o meu estúdio, em busca da minha espingarda. Além dela, carreguei e
coloquei no bolso uma pistola pequena, mas robusta. Estava bem ciente de que, caso houvesse de fato algum perigo, ele
estaria na direção do Fosso, e eu queria estar preparado.

Saí do estúdio e desci até a porta dos fundos, com Pepper em meu encalço. Lá fora, investiguei rapidamente os jardins
ao redor e então comecei a ir em direção ao Fosso. No caminho, fiquei bastante atento, já com a arma em riste. Percebi
que Pepper corria à frente sem nenhuma hesitação aparente. Com isso, deduzi que não havia nenhum perigo iminente e
apertei o passo atrás dele. Agora, ele já estava na parte mais alta do Fosso, farejando a borda. 

Um minuto depois, eu já estava ao seu lado, olhando para baixo. Durante alguns instantes, mal pude acreditar que era o
mesmo lugar, tão diferente estava. A ravina escura e cheia de árvores de duas semanas antes, com um riacho que corria
vagaroso, oculto pela folhagem, lá no fundo, já não existia mais. Em vez disso, meus olhos pairavam sobre um abismo de
margens irregulares, quase totalmente tomado por um lago lúgubre de águas turvas. Uma das laterais da ravina estava
completamente despida de vegetação, e só se via a rocha nua.

Um pouco para a esquerda, a lateral do Fosso parecia ter soçobrado por completo, formando uma fenda profunda em forma
de \textsc{v} na face rochosa. Essa fenda ia da parte superior da ravina quase até a água, penetrando a lateral do Fosso e
percorrendo uns doze metros. A parte de cima deveria ter pelo menos uns seis metros de largura; a partir daí, parecia
estreitar até ficar com uns dois. Mas o que chamou minha atenção mais do que essa fenda assombrosa foi um
buraco enorme, a certa distância da fenda, bem no ângulo do \textsc{v}. Era bem definido, muito semelhante a um pórtico, mas, como
estava oculto pelas sombras, eu não podia vê"-lo muito bem.

O outro lado do Fosso ainda mantinha sua vegetação, mas ela estava tão destruída em alguns pontos, e tão coberta de
poeira e detritos, que mal se distinguia como tal.

A minha primeira impressão, de que houvera um deslizamento de terra, não era, eu começava a perceber, suficiente para
dar conta de todas as mudanças que eu presenciava. E a água\ldots{}? Virei rapidamente, pois eu havia notado que, em algum
ponto à minha direita, havia o ruído de água corrente. Não vi nada, mas, agora que aquilo me chamava a atenção, pude
claramente ver que vinha de algum lugar do lado leste do Fosso.

Lentamente, avancei naquela direção; o som ficava cada vez mais nítido à medida que eu avançava, até que logo eu estava
bem acima dele. Mesmo então, não conseguia saber o que o causava; até que me ajoelhei e estiquei a cabeça sobre o
precipício. Um pouco mais adiante naquele abismo, vi outro e, ainda mais à frente, dois menores. Então era isso que
explicava a quantidade de água dentro do Fosso; se a queda de pedras e terra havia bloqueado a saída do riacho do
fundo, não havia dúvida de que ele contribuía em grande parte para a água.

Todavia, quebrei a cabeça em busca de uma explicação para a terrível aparência do lugar --- aquelas pequenas saídas
d'água e a enorme fenda mais adiante na ravina! Parecia"-me que somente o deslizamento de terra era pouco para explicar tudo
isso. Podia imaginar um terremoto ou uma grande explosão criando uma situação assim, mas não houve nenhum dos dois.
Depois, fiquei de pé rapidamente, lembrando"-me do enorme estrondo e da nuvem de poeira que se seguiu logo depois,
subindo bem alto no céu. Mas balancei a cabeça, incrédulo. Não! Deve ter sido o som das pedras e da terra caindo o que
eu ouvira; claro que o pó subiria, era evidente. Mesmo assim, apesar do meu raciocínio, eu tinha uma sensação ruim de
que essa teoria não satisfazia meu senso de plausibilidade; haveria outra que eu pudesse imaginar e que fosse
minimamente plausível? Pepper ficou sentado na grama enquanto eu examinava o lugar. Agora, enquanto me virara para ir
até o lado norte da ravina, ele se punha de pé e me seguia.

Aos poucos, e mantendo"-me atento e olhando em todas as direções, contornei o Fosso, mas não encontrei nada que já não
tivesse visto. Do lado oeste, podia ver quatro cachoeiras ininterruptas. Elas também estavam a uma distância
considerável da superfície do lago --- uns quinze metros, calculei.

Durante mais algum tempo, fiquei ali, vagando; de olhos e ouvidos bem atentos, mas sem ver ou ouvir nada suspeito. Todo
o lugar estava maravilhosamente tranquilo; de fato, com exceção do murmúrio contínuo da água, na parte superior, não
havia ruído algum que rompesse o silêncio.

Durante todo esse tempo, Pepper não havia demonstrado nenhum sinal de inquietude. Isso me parecia indicar que, pelo
menos por enquanto, não havia nenhuma daquelas criaturas suínas por perto. Pelo que eu podia perceber, ele dedicava
toda sua atenção a arranhar e farejar por entre a grama na beirada do Fosso. Às vezes, ele saía da beirada e corria na
direção de casa, como se seguisse rastros invisíveis; em todos os casos, voltava depois de alguns minutos. Eu
tinha pouca dúvida de que ele estava realmente traçando o trajeto das Coisas suínas e o fato de que cada um deles
parecia levá"-lo de volta ao Fosso me parecia prova de que as feras haviam voltado para o mesmo ponto de onde saíram.

Ao meio"-dia, fui para casa almoçar. À tarde, fiz uma investigação parcial dos jardins, acompanhado por Pepper, nada
descobri que pudesse indicar a presença das criaturas.

Certa vez, enquanto atravessávamos os arbustos, Pepper correu adiante, entrando por entre as moitas, dando um latido
destemido. Com isso, tive um sobressalto, sentindo um medo repentino, e apontei a arma, preparado para atirar; tudo
isso só para rir de nervoso logo depois, quando Pepper reapareceu, correndo atrás de um infeliz gato. Mais para o fim
da tarde, desisti de investigar e voltei para casa. De repente, quando estávamos passando por um grande aglomerado de
arbustos à nossa direita, Pepper desapareceu, e eu podia ouvi"-lo farejando e rosnando de maneira suspeita. Com o cano
da arma, separei a folhagem à minha frente e espiei lá dentro. Não havia nada de anormal, exceto que alguns dos galhos
estavam recurvos e quebrados, como se algum animal tivesse feito uma toca ali havia pouco tempo. Provavelmente, pensei,
aquele foi um dos locais ocupados por alguma das criaturas suínas na noite do ataque.

No dia seguinte, continuei minha investigação dos jardins, sem grandes resultados. À noite, eu já tinha percorrido
toda a extensão dos jardins e agora sabia, sem a menor dúvida, que não havia mais nenhuma daquelas Coisas escondida no
lugar. Desde então acredito que estava correto em minha suposição anterior, de que elas haviam ido embora
logo depois do ataque.


\clearpage

\chapter{O fosso subterrâneo}

\textsc{Outra semana} passou"-se durante a qual fiquei grande parte do tempo perto da boca do Fosso. Alguns dias antes, eu havia
chegado à conclusão de que o buraco em forma de arco, no ângulo da enorme fenda, era o local por onde as Coisas suínas
saíam, oriundas de algum lugar maldito nas entranhas da terra. Quanto isso estava próximo da verdade só fui descobrir
depois.


Talvez seja fácil compreender a minha extrema curiosidade, embora também sentisse medo, para saber a que lugar infernal
aquela abertura levava; até então, eu não havia pensado seriamente em investigar. Ainda me sentia muito abalado
pelo terror que eu provava pelas criaturas suínas para pensar em me aventurar, de livre e espontânea vontade, em um local
onde pudesse ter a menor chance de entrar em contato com elas.

Contudo, à medida que o tempo passava, essa sensação diminuía consideravelmente, foi então que, alguns dias depois,
acreditei talvez ser possível descer e investigar o buraco; não estava totalmente avesso à ideia, como se pode
imaginar. Mas não acredito que mesmo então eu de fato tencionasse jogar"-me em tal aventura imprudente. Pelo que eu sabia
até então, entrar naquela abertura escura talvez significasse até mesmo minha morte. Mesmo assim, tão perseverante é a
curiosidade humana que, por fim, meu único desejo era descobrir o que havia por trás daquela entrada lúgubre.

Lentamente, à medida que os dias se passavam, meu medo das Coisas suínas tornou"-se uma emoção do passado --- mais uma
memória desagradável e incrível do que qualquer outra coisa.

Assim, chegou o dia em que, jogando para o alto raciocínio e imaginação, providenciei uma corda em casa e,
amarrando"-a bem a uma árvore robusta, no topo da fenda, a uma pequena distância da beirada do Fosso, deixei a outra
extremidade cair na fenda até ficar bem em frente à entrada do negro buraco.

Com cuidado, e com muitas dúvidas sobre se não era um ato de loucura o que estava fazendo, desci devagar
pela parede, usando a corda como apoio, até chegar ao buraco. Ali, ainda segurando"-me à corda, fiquei de pé e espiei lá
dentro. Estava completamente escuro e eu não ouvia som algum. Contudo, alguns segundos depois, pensei ter ouvido algo.
Segurei a respiração e prestei atenção, mas o lugar continuava silencioso como um túmulo, e voltei a respirar normalmente.
Nesse mesmo instante, ouvi o ruído mais uma vez. Era como o som de uma respiração difícil --- profunda e bastante nítida.
Durante breves segundos, fiquei ali, petrificado, sem conseguir me mover. Os sons cessaram mais uma vez e eu
nada ouvia.

Enquanto estava ali em pé, nervoso, meu pé deslocou uma pedrinha que saiu rolando para dentro da escuridão com um ruído
surdo. No mesmo momento, o ruído se repetiu inúmeras vezes; cada eco que vinha depois era mais fraco que o anterior e
parecia viajar até bem longe, como se a uma distância bem remota. Quando novamente veio o silêncio, ouvi aquela
respiração furtiva. Toda vez que eu respirava, ouvia uma respiração em resposta. Os sons pareciam cada vez mais
próximos; depois, ouvi vários outros, mas mais baixos e distantes. Por que não agarrei a corda e fugi do perigo, não
sei dizer. Era como se estivesse paralisado. Comecei a suar muito e tentei umedecer os lábios com a língua.
Senti a garganta repentinamente seca e tossi, rouco. O ruído foi"-me devolvido em dezenas de tons horríveis, roufenhos,
como uma zombaria. Tentei em vão enxergar na escuridão, mas nada era visível. Tive uma sensação estranha, como se
estivesse sem ar, e de novo tossi, a boca seca. Outra vez o eco capturou o som, subindo e descendo de maneira
grotesca e morrendo aos poucos, até abafar"-se em silêncio.

Então, de repente, tive uma ideia e prendi a respiração. A outra respiração parou. Respirei novamente e, mais uma
vez, a outra respiração recomeçou. Agora eu não sentia mais medo. Sabia que os estranhos ruídos não eram feitos por
alguma criatura suína à espreita; eram só o eco de minha respiração.

Porém fiquei tão assustado que me senti contente por subir pela fenda e içar a corda. Estava abalado e nervoso demais
para pensar em entrar naquele buraco escuro, então voltei para casa. Senti"-me melhor na manhã seguinte; mesmo
assim, não conseguia criar coragem suficiente para explorar o lugar.

Durante todo esse tempo, a água no Fosso subia lentamente; agora, estava um pouco abaixo da abertura na parede. Na
velocidade com que subia, estaria alinhada com o chão dentro de menos de uma semana; percebi que, a não ser que eu
investigasse o local em breve, eu jamais o faria, a água subiria e subiria até que a própria abertura ficaria
submersa.

Talvez tenha sido esse pensamento que me impeliu a agir; qualquer que fosse o motivo, poucos dias depois lá estava eu no
topo da fenda e bem equipado para a tarefa.

Desta vez, eu estava determinado a não titubear e chegar ao fim da investigação. Com essa intenção em mente, eu trouxera,
além da corda, um feixe de velas para usar como lanterna e também a minha espingarda de cano duplo. No meu cinto,
levava uma garrucha carregada com munição de espingarda.

Assim como antes, amarrei a corda à árvore. Depois, com a arma pendurada nos ombros com um pedaço de corda robusta,
desci pela beirada do fosso. Com esse movimento, Pepper, que estivera observando atento o que eu fazia, ficou de pé e
correu até mim, meio latindo, meio reclamando, um gesto que me pareceu ser de advertência. Mas eu estava decidido e fiz um
gesto para que ele se deitasse. Adoraria levá"-lo comigo, mas isso era quase impossível, dadas as circunstâncias. Quando
meu rosto ficou no mesmo nível da beirada do Fosso, ele me lambeu, bem na boca; então, agarrando a manga de minha
roupa entre os dentes, começou a puxar com força. Era evidente que não queria que eu fosse. Mas, como eu já estava
decidido, não tencionava desistir; ralhando rispidamente com Pepper para que me soltasse, continuei a descer,
deixando o pobrezinho lá em cima, latindo e chorando como se fosse um filhote abandonado.


Com cuidado, fui abaixando meu corpo de projeção a projeção. Sabia que, se escorregasse, eu me molharia.

Ao chegar à entrada, soltei a corda e desamarrei a arma. Então, olhando uma última vez para o céu --- o qual percebi
estar ficando rapidamente nublado ---, dei alguns passos para frente, para me proteger do vento, e acendi uma das velas.
Levantando"-a acima da cabeça e segurando firme a arma, comecei a andar, lentamente, olhando para todos os lados.

Durante o primeiro minuto, podia ouvir o melancólico som do choro de Pepper vindo até mim. Aos poucos, à medida que
penetrava na escuridão, seu choro ficou mais distante; até que, depois de algum tempo, eu nada ouvia. O caminho tendia
a descer um pouco e a ir para a esquerda, e continuava ainda para a esquerda, até que descobri que estava me levando
bem na direção da casa.

Com muita cautela, eu seguia adiante, sempre parando depois de alguns passos para prestar atenção aos sons. Talvez
tivesse percorrido uns cem metros quando, de repente, tive a impressão de captar um som muito baixo, vindo de algum ponto na
passagem atrás de mim. Com o coração batendo com força, fiquei atento. O ruído ficou mais nítido e parecia estar se
aproximando rápido. Agora, podia ouvi"-lo distintamente. Era o som suave e abafado de alguém correndo. Nos
primeiros instantes do pânico, fiquei parado, hesitante, sem saber se avançava ou recuava. Então, dando"-me conta de
repente sobre o melhor a fazer, fiquei com as costas rentes à parede rochosa à minha direita e, segurando a vela acima
da cabeça, esperei --- com a arma em punho ---, amaldiçoando minha estúpida curiosidade por ter me levado àquela difícil
situação.

Não precisei esperar mais que alguns segundos para ver dois olhos refletindo a luz da vela na penumbra. Ergui a arma,
usando somente a mão direita, e mirei rápido. Enquanto fazia isso, algo pulou da escuridão com um latido ensurdecedor
de alegria que acordou os ecos como se fossem trovões. Era Pepper. Como ele havia conseguido descer pela fenda eu não
saberia dizer. Enquanto eu fazia carinho, nervoso, em sua pelagem, percebi que ele estava encharcado e concluí que
devia ter tentado me seguir e caiu na água, e de lá não deve ter sido muito difícil escalar até a abertura.

Depois de esperar um minuto, mais ou menos, até me recompor, continuei meu caminho, com Pepper me seguindo sem fazer
ruído. Sentia"-me curiosamente feliz por ter meu velho amigo comigo. Ele me fazia companhia e, de algum modo, com ele
por perto, eu sentia menos medo. Eu também sabia quão rápido seus ouvidos afiados detectariam a presença de alguma
criatura indesejável, caso houvesse alguma na escuridão que nos envolvia.

Durante alguns minutos, continuamos caminhando devagar; o caminho ainda levava na direção da casa. Logo, concluí,
estaríamos abaixo dela, caso o caminho cobrisse uma distância suficiente. Eu seguia à frente, cauteloso, e
continuei assim durante mais uns cinquenta metros, aproximadamente. Então parei e segurei a vela no alto, e tive
grande motivo para me sentir grato por ter feito isso: pois ali, a uns três passos de distância, o caminho desaparecia
e, em seu lugar, havia uma fossa escura que me fez sentir um arrepio de medo repentino.

Com extremo cuidado, andei devagar para frente e olhei para baixo, mas não consegui enxergar nada. Depois, fui para o
lado esquerdo da passagem, para ver se o caminho continuava. Ali, contra a parede, descobri que um peitoril estreito,
com menos de um metro de largura, seguia adiante. Com cautela, pus meu pé sobre ele, mas não o percorri muito antes de
me arrepender por me aventurar ali. Pois, logo após alguns passos, a trilha já estreita transformava"-se em mera
beirada, flanqueada pela rocha inexorável da enorme muralha que avultava até alcançar um teto que eu não enxergava, de
um lado, e, do outro, o enorme abismo. Não podia evitar pensar em quanto estaria indefeso caso fosse atacado ali, sem
espaço para virar, onde até o coice de minha arma talvez fosse suficiente para me fazer cair de cabeça nas profundezas
lá embaixo.

Para meu grande alívio, um pouco mais adiante, a trilha de repente alargava"-se mais uma vez, assumindo a dimensão
original. Aos poucos, à medida que avançava, percebi que o caminho tendia sempre à direita, e então, depois de alguns
minutos, descobri que não estava indo para frente; estava simplesmente contornando aquele enorme abismo. Eu havia,
evidentemente, chegado ao fim da grande passagem.

Cinco minutos depois, eu estava no mesmo ponto de onde havia saído; tendo feito a volta completa ao redor do que eu
agora imaginava ser um enorme fosso, cuja abertura deveria ter pelo menos uns cem metros de largura.

Durante certo tempo, fiquei ali, perdido em meus pensamentos, perplexo. ``O que isso significa?'', era a pergunta que
ecoava em meu cérebro.

De repente, tive uma ideia e saí procurando uma pedra. Logo achei um pedaço de rocha do tamanho de um pão. Enfiando a
vela numa fresta no chão, afastei"-me um pouco da beirada e, correndo um pouco, lancei a pedra no abismo --- a ideia era
lançá"-la de modo que não atingisse as laterais. Então me inclinei para frente e fiquei ouvindo; por mais que
ficasse totalmente em silêncio, durante quase um minuto, nenhum som retornou da escuridão.

Eu soube que a profundidade do buraco deveria ser imensa; a pedra, caso tivesse atingido algo, era grande o
suficiente para ecoar naquele estranho lugar, durante um período indefinido. A caverna havia ecoado os sons de meus
passos indefinidamente. Era um lugar pavoroso, e eu de muito bom grado teria dado meia"-volta e deixado os mistérios de
sua solidão sem solução, mas fazê"-lo significaria admitir minha derrota.

Então tive a ideia de tentar ver o abismo. Ocorreu"-me que, se eu colocasse velas ao redor do buraco, poderia pelo
menos ter um vislumbre do lugar.

Descobri, contando, que trouxera quinze velas no feixe --- sendo que minha primeira intenção era, como eu havia dito,
fazer uma tocha com elas. Comecei a colocá"-las ao redor da fossa, com um intervalo de cerca de vinte metros entre uma e
outra.

Depois de completar o círculo, fiquei na passagem e tentei ter uma ideia de como era o lugar. Mas logo descobri
que as velas eram insuficientes para o meu propósito. Elas pouco deixaram a
penumbra visível. O que fizeram de fato foi confirmar minha estimativa sobre o tamanho da abertura; por mais que não
mostrassem o que eu queria ver, o contraste que faziam com a escuridão profunda estranhamente me agradava. Era como se
quinze pequenas estrelas brilhassem na noite subterrânea.

De repente, enquanto eu estava ali de pé, Pepper deu um uivo, que foi carregado pelos ecos e se repetiu com
terríveis variações, morrendo lentamente. Com um movimento rápido, segurei alto a única vela que deixei comigo e olhei
de soslaio para o cão; no mesmo instante, pensei ouvir um ruído, como se fosse uma risada diabólica, erguer"-se das
profundezas do fosso, até agora silencioso. Fiquei assustado; depois, lembrei que talvez fosse o eco do uivo de Pepper.

Pepper havia se afastado de mim, adiantando"-se alguns passos na passagem; farejava o chão rochoso e acho que o ouvi
bebendo água. Fui até ele, segurando a vela baixo. Quando me movi, ouvi minhas botas fazendo \textit{tcháp},
\textit{tcháp}; a luz refletia"-se em algo que brilhava e passava correndo pelos meus pés, indo na direção
da fossa. Abaixei"-me e olhei, e então soltei uma exclamação de surpresa. De algum lugar mais acima na passagem, uma
corrente d’água corria rápido na direção da grande abertura e aumentava a cada segundo.

Novamente, Pepper soltou um uivo grave e, correndo até mim, agarrou meu casaco e tentou me puxar para o caminho que
levava à entrada. Com um gesto nervoso, soltei"-me dele e fui rápido até a parede da esquerda. Se havia algo vindo, eu
teria a parede às minhas costas.

Enquanto olhava ansioso para o caminho, minha vela refletiu em algo mais para o fundo da passagem. No mesmo
instante, ouvi um barulho como se fosse um murmúrio e que foi ficando mais alto, até encher a caverna de maneira
ensurdecedora. Do fosso veio um eco grave, oco, como se fosse o soluço de um gigante. Eu havia saltado para um dos
lados, para a beirada que contornava o abismo e, ao me virar, vi uma grande muralha de água espumante passando por mim
e despencando tumultuosamente no abismo. Uma nuvem de respingos explodiu sobre mim, apagando minha vela e me deixando
totalmente molhado. Eu ainda segurava minha arma. As três velas mais próximas apagaram"-se; as mais distantes só
tremularam de leve. Depois dessa primeira precipitação, o fluxo de água foi diminuindo até tornar"-se uma corrente
contínua, talvez com menos de meio metro de profundidade; embora eu só pudesse ver isso depois de pegar uma das velas
acesas e, com ela, começar a investigar. Pepper, felizmente, havia me seguido quando pulei para a beirada e, agora,
sentindo"-se bastante intimidado, mantinha"-se bem perto de mim.

Uma breve investigação mostrou que a água chegava até o outro lado da passagem e corria a uma velocidade tremenda. Ela
ficou mais funda mesmo enquanto eu estava ali parado. Eu podia imaginar o que acontecera. A água da
ravina havia penetrado, de alguma maneira, na passagem. Caso isso fosse de fato verdade, ela aumentaria de volume cada
vez mais, até eu não conseguir mais sair do lugar. Esse pensamento era assustador. Era óbvio que eu precisava sair dali o
mais rápido possível.

Pegando a arma pela coronha, usei"-a para medir a profundidade da água. Chegava logo abaixo dos joelhos. O barulho que
ela fazia ao cair no fosso era ensurdecedor. Então, chamando Pepper, fui para o meio da enchente, usando a arma como
cajado. No mesmo instante, a água começou a agitar"-se acima dos meus joelhos, quase alcançando a parte superior das
coxas, tamanha era a velocidade com que corria. Durante um breve momento, quase escorreguei, mas pensar no que poderia
estar atrás de mim me estimulava a me esforçar e, dando um passo de cada vez, fui avançando.

Eu nada sabia de Pepper a princípio. Tudo o que conseguia fazer era tentar manter"-me de pé e fiquei radiante ao vê"-lo
aparecer ao meu lado. Ele avançava na água com determinação. Pepper é um cão grande, com pernas compridas e finas, e
imagino que a água não tivesse tanta pressão sobre elas como tinha sobre as minhas. De qualquer modo, ele conseguia
avançar bem mais do que eu; tomou a dianteira, como um guia, e conscientemente ou não, ajudava a diminuir um pouco a
força da água sobre mim. Assim avançamos, um passo de cada vez, arfando, com dificuldade, até que conseguimos
atravessar mais ou menos uns cem metros. Talvez porque eu não estivesse mais tomando tanto cuidado, ou por
haver um ponto escorregadio no chão rochoso, de repente, escorreguei e caí de cara. No mesmo instante, a água me cobriu
como se fosse uma catarata, lançando"-me para baixo, para aquele buraco sem fundo, com uma velocidade aterrorizante.
Tentei freneticamente, mas era impossível conseguir me agarrar a algo. Eu nada podia fazer além de arfar e me afogar.
De repente, algo agarrou meu casaco e me fez parar. Era Pepper. Sentindo minha falta, ele deve ter voltado correndo, em
meio a todo aquele tumulto e escuridão, para tentar me achar, me agarrou e me segurou até que eu pudesse ficar
de pé de novo.

Tenho uma vaga lembrança de ter visto, durante um breve instante, o brilho de diversas luzes, mas nunca tive certeza.
Se minhas impressões estiverem corretas, devo ter sido arrastado até a beirada daquele abismo terrível quando Pepper
conseguiu me fazer parar. As luzes, é claro, só poderiam ser as chamas distantes das velas que eu havia deixado
acesas. Mas, como disse, não tenho de modo algum certeza. Meus olhos estavam cheios d’água e eu estava bastante
abalado.

E lá estava eu, sem minha arma, sem nenhuma luz, extremamente confuso, com a água ficando cada vez mais funda;
dependendo somente de meu velho amigo Pepper para me ajudar a sair daquele lugar infernal.

Eu estava de frente para a corrente. Naturalmente, era o único modo com que eu poderia sustentar minha posição durante
alguns instantes, pois mesmo meu caro Pepper não conseguiria me segurar durante muito tempo contra aquela terrível
corrente sem ajuda minha, ainda que imprecisa.

Talvez um minuto tenha se passado em que minha situação parecia muito incerta; depois, gradualmente, recomecei meu
tortuoso percurso subindo a passagem. E aí começou minha implacável luta contra a morte, da qual sempre espero sair
vitorioso. Aos poucos, furioso, quase sem esperanças, lutei; e meu fiel Pepper conduzia"-me, arrastava"-me, para
cima e para frente, até que, finalmente, vislumbrei a abençoada luz. Era a entrada. Mais alguns metros e eu chegava à
abertura, com a água irrompendo em uma torrente voraz na altura dos meus quadris.

Agora eu compreendia a causa da catástrofe. Chovia muito, torrencialmente. A superfície do lago estava no mesmo nível
da parte inferior da abertura --- não! Mais do que isso, estava mais para cima. Era evidente que a chuva havia aumentado o
volume do lago e fez com que ele subisse prematuramente; pois, na velocidade em que a ravina estava se enchendo, ele só
teria chegado à entrada depois de uns dois dias. 

Por sorte, a corda pela qual eu havia descido caía sobre a abertura, acima das águas que corriam. Agarrei a ponta e
amarrei"-a bem firme ao redor do corpo de Pepper; então, juntando as forças que me restavam, comecei a
escalar o penhasco. Cheguei à beira do Fosso no último estágio de exaustão. Mas eu ainda precisava ter forças para um
último feito, que era puxar Pepper para local seguro.

Sentindo extremo cansaço, lentamente, comecei a puxar a corda. Uma ou duas vezes, pareceu"-me que eu não conseguiria,
pois Pepper é um cão pesado e eu estava completamente exausto. Mas soltar a corda significaria morte certa para meu
velho amigo e esse pensamento fez com que eu me esforçasse mais. Tenho apenas uma vaga lembrança do fim. Lembro"-me de
puxar a corda e que esses momentos estranhamente pareceram passar muito devagar. Também lembro ter visto o focinho de
Pepper surgir na beirada do Fosso depois do que pareceu ser uma eternidade. De repente, tudo ficou escuro.


\clearpage

\chapter{O alçapão no grande porão}

\textsc{Acredito que} tenha desmaiado, pois a lembrança que tenho logo em seguida é que abri os olhos e já estava anoitecendo.
Eu estava deitado de costas, com uma perna dobrada sob a outra, e Pepper lambia minhas orelhas. Eu me sentia
terrivelmente enrijecido, minha perna estava dormente do joelho para baixo. Durante alguns minutos, fiquei deitado
assim, confuso; então, lentamente, fiz força para me sentar e olhar ao redor.

A chuva cessara, mas as árvores ainda gotejavam muito. Do Fosso, ouvi o murmúrio contínuo da água que corria. Senti
frio, calafrios. Minhas roupas estavam encharcadas e eu sentia dores em todo o corpo. Muito devagar, minha perna
dormente voltou à vida e, depois de alguns instantes, ensaiei ficar de pé. Só consegui na segunda tentativa, mas estava
cambaleante, muito fraco. Tinha a impressão de que ia ficar doente e decidi voltar rápido para casa, zonzo. Meus
passos eram trôpegos e minha cabeça estava confusa. A cada passo que dava, sentia dores lancinantes em meus membros.

Havia dado talvez uns trinta passos quando o choro de Pepper chamou minha atenção, e virei, rígido, na direção dele. O
velho cão tentava me seguir, mas não conseguia por causa da corda com que eu o havia içado, já que ela ainda estava
amarrada ao redor de seu corpo e a outra extremidade continuava presa à árvore. Durante alguns instantes, tentei
desatar os nós, sentindo"-me fraco, mas eles estavam apertados e molhados e eu não consegui. Então lembrei que tinha
uma faca e, depois de um minuto, consegui cortar a corda.

Mal sei dizer como cheguei à casa e, dos dias posteriores, lembro"-me menos ainda. De uma coisa estou certo: não fosse o
amor e os cuidados constantes de minha irmã, eu não estaria escrevendo neste momento.

Quando recobrei os sentidos, descobri que estivera de cama durante quase duas semanas. Outra semana passou"-se antes que
eu tivesse forças para sair vacilante para os jardins. Mesmo então, não conseguia andar até o Fosso. Gostaria
de ter perguntado à minha irmã até que ponto a água havia subido, mas achei melhor não mencionar o assunto. De fato,
desde então, adotei como regra jamais falar com ela sobre as coisas estranhas que acontecem nesta grande e antiga casa.

Foi só alguns dias depois que consegui ir até o Fosso. Lá descobri que durante as semanas de minha ausência houve uma
mudança tremenda. Em vez da ravina três terços cheia d'água, agora eu olhava para um enorme lago, cuja plácida
superfície refletia friamente a luz. A água havia chegado a uns dois metros da beirada. Somente em uma parte as águas
não estavam calmas, e era sobre o local onde, nas profundezas daquelas águas silenciosas, abria"-se a entrada subterrânea para
aquele enorme fosso. Neste ponto, a água borbulhava continuamente; de vez em quando, um curioso
borbotão, que mais parecia um suspiro, subia até a superfície. Fora isso, não havia nada que indicasse as coisas
ocultas lá embaixo. Enquanto olhava, ali parado, pensei maravilhado em quanto tudo havia dado certo. A entrada do lugar
de onde as criaturas suínas surgiram estava lacrada por uma força que me fazia sentir que não havia mais nada a temer. Mesmo assim, com esse sentimento, veio a sensação de que, agora, eu jamais saberia detalhes sobre o lugar
de onde aquelas Coisas terríveis surgiram. Estava totalmente fechado, para todo o sempre, oculto à curiosidade humana.

Era estranho --- por estar ciente daquele infernal buraco subterrâneo --- quão adequado era o nome Fosso para o lugar.
Fico curioso para saber como deram esse nome, e quando. Obviamente, podemos concluir que o formato e a profundidade da
ravina já sugeririam o nome “Fosso”. Todavia, não seria possível que esse nome tivesse, durante todo esse tempo, algum
significado maior que indicasse --- ou assim eu suspeitava --- o Fosso ainda maior e mais assombroso, oculto no fundo da
terra, debaixo desta velha casa? Sob esta casa! Mesmo agora, a ideia me parece estranha e terrível. Pois eu descobri,
sem a menor dúvida, que o Fosso abre"-se bem debaixo desta casa, a qual obviamente está apoiada em algum ponto sobre o
centro do Fosso, sobre um imenso teto arqueado de rocha sólida.

Quanto a isso, aconteceu que, em uma ocasião em que precisei descer ao porão, tive a ideia de fazer uma visita à grande
câmara onde fica o alçapão, para ver se tudo estava como antes.

Ao chegar lá, caminhei devagar rumo o centro, até chegar ao alçapão. Lá estava ele, com as pedras empilhadas em
cima, exatamente como eu o vira da última vez. Eu segurava uma lamparina e pensei que agora seria uma boa hora para
investigar o que havia debaixo daquele enorme pedaço de carvalho. Colocando a lamparina no chão, derrubei as pedras de
cima do alçapão e, agarrando a argola, puxei"-a. Assim que o fiz, o porão encheu"-se do som trovejante que vinha lá de
baixo. Ao mesmo tempo, uma brisa úmida soprou sobre meu rosto, trazendo consigo pequenos respingos. Com isso, deixei
cair o alçapão às pressas, sentindo um pouco de medo e enorme assombro.

Durante alguns instantes, fiquei ali parado, perplexo. Não sentia muito medo. O medo constante das Coisas suínas havia
desaparecido havia tempos, mas eu sem dúvida estava nervoso e aturdido. Então, de repente, tive uma ideia: ergui o
pesado alçapão, sentindo"-me animado. Deixando"-o aberto, peguei a lamparina e, ajoelhando"-me, meti"-a na abertura. Ao
fazer isso, o vento úmido e os respingos atingiram meus olhos, impedindo"-me de enxergar durante alguns instantes. Mesmo
quando minha visão havia voltado ao normal, não consegui distinguir nada lá embaixo além da escuridão e do redemoinho
de respingos.

Já que era inútil tentar enxergar alguma coisa com a luz tão alta, tateei os bolsos em busca de algum barbante com o
qual poderia abaixá"-la na abertura. Enquanto eu tateava, a lamparina escorregou de meus dedos e caiu na escuridão.
Durante breve instante, testemunhei sua queda e vi a luz brilhar sobre uma massa tumultuosa de espuma branca, a uns
vinte ou trinta metros abaixo de mim. Então ela sumiu. Minha suposição mostrava"-se correta, e agora eu sabia a causa
do barulho e da umidade. O grande porão estava ligado ao Fosso por meio do alçapão, que se abria logo acima dele; a
umidade eram os respingos oriundos da água que despencava nas profundezas.

Num átimo, pensei numa explicação para certas coisas que até então me deixavam perplexo. Agora, podia entender como os
ruídos --- na primeira noite da invasão --- pareciam estar vindo bem debaixo de meus pés. E a risada que havia soado quando
abri o alçapão pela primeira vez! Evidentemente, algumas daquelas Coisas suínas deveriam estar bem embaixo de mim.

Outro pensamento ocorreu"-me. Teriam as criaturas se afogado? Será que poderiam se afogar? Lembrei que não consegui
encontrar nenhuma evidência de que meus tiros foram fatais. Será que estavam vivas do modo como compreendíamos a vida,
ou eram mortos"-vivos? Tais pensamentos corriam"-me pela mente enquanto eu estava ali parado no escuro, tateando os
bolsos em busca de fósforos. Agora, já estava com a caixa na mão e, ao acender um, fui até o alçapão e o
fechei. Empilhei de novo as pedras sobre ele e saí do porão.

Portanto, imagino que a água continue a descer trovejando para dentro daquele infernal Fosso sem fundo. Às vezes, sinto
um desejo inexplicável de descer até o grande porão, abrir o alçapão e fitar aquela escuridão impenetrável e úmida. Às
vezes, o desejo é quase insuportável em sua intensidade. Não é a mera curiosidade que me impele, e sim como se
estivesse sob o efeito de alguma influência inexplicável. Mesmo assim, jamais vou até lá, e tenciono lutar contra esse
estranho desejo e derrotá"-lo, assim como lutaria contra o ímpio desejo da autodestruição.

Essa ideia de que há alguma força intangível atuando sobre mim talvez não faça sentido. Todavia, meus instintos me
alertam de que não é esse o caso. Em assuntos assim, a razão me parece menos digna de confiança do que a intuição.

Para encerrar, há um pensamento que me persegue de maneira cada vez mais
constante. É o de que habito uma casa muito estranha; uma casa horrível.
Começo a duvidar de que faço a coisa certa ao continuar aqui. Mas, se eu fosse
embora, para onde iria? Onde conseguiria ter o isolamento e a sensação da
presença dela,\footnote{ Uma interpolação aparentemente sem significado. Não
consigo encontrar nenhuma referência prévia no manuscrito a esse assunto.
Contudo, ela se torna mais clara à luz dos acontecimentos posteriores.}
as únicas coisas que tornam minha vida suportável?


\clearpage

\chapter{O Mar Adormecido}

\textsc{Durante considerável} período após o último incidente narrado em meu diário, pensei seriamente em abandonar esta
casa, e bem poderia tê"-lo feito, não fosse o acontecimento fantástico que estou prestes a narrar.

Como foi acertada minha intuição, lá no fundo de minha alma, de permanecer aqui --- apesar das visões e vislumbres de
coisas desconhecidas e inexplicáveis, pois, se não tivesse ficado, não teria visto outra vez o rosto daquela que amei.
Sim, embora poucos saibam, com exceção de minha irmã, Mary, eu amei e --- ah! pobre de mim! --- perdi.

Eu contaria a história daqueles doces dias, mas isso seria o mesmo que fazer doer velhas feridas; todavia, depois de
tudo o que aconteceu, que cautela preciso ter? Pois ela veio até mim do desconhecido. Estranhamente, ela me admoestou;
admoestou com veemência a respeito desta casa; implorou para que eu saísse daqui; mas admitiu, quando a questionei, que
ela não teria vindo até mim se eu estivesse em outro local. Não obstante, apesar de tudo isso, ela ainda me admoestou
rispidamente, dizendo"-me que era um lugar havia muito possuído pelo mal, sob o poder de leis terríveis das quais ninguém
aqui tem conhecimento. E eu --- eu apenas indaguei, novamente, se ela teria vindo até mim caso eu estivesse em outro
lugar, e ela só continuou imóvel, em silêncio.

E foi assim que cheguei ao local do Mar Adormecido --- ou assim ela o chamava, no doce linguajar que usava comigo. Eu
estava acordado até mais tarde em meu estúdio, lendo; devo ter cochilado em cima do livro. De repente, acordei e fiquei
ereto na cadeira, com um sobressalto. Durante alguns instantes, olhei ao redor, com a sensação confusa de que havia
algo de errado. O aposento tinha uma aparência enevoada, o que dava uma curiosa aparência suave a cada mesa, cadeira e
móvel.

Aos poucos, a névoa aumentou; ela crescia do nada. Então, devagar, uma luz branca e suave começou a brilhar no
estúdio. O brilho das chamas das velas atravessava"-a debilmente. Olhei de um lado para o outro e percebi que ainda
conseguia ver cada móvel, mas de um jeito estranhamente surreal, como se o fantasma de cada mesa e cadeira estivesse
substituindo as próprias peças maciças.

Enquanto as olhava, vi que elas desapareciam cada vez mais; até que, lentamente, sumiram por completo. Agora,
eu olhava de novo para as velas. Elas brilhavam fracas e, enquanto eu as observava, ficaram ainda mais irreais e
também desapareceram.

O estúdio agora estava tomado por uma luz de crepúsculo suave, mas luminosa, como se fosse uma delicada névoa de luz.
Fora isso, eu não enxergava mais nada. Até as paredes haviam sumido. 

Logo depois, percebi um som contínuo e baixo que pulsava através do silêncio que me envolvia. Prestei atenção. Ele foi
ficando mais nítido até que tive a impressão de estar ouvindo o ritmo de um enorme oceano. Não sei dizer quanto tempo
passei assim, mas, depois de algum tempo, parecia que conseguia enxergar através da névoa e fui percebendo que
estava parado às margens de um mar imenso e silencioso. A praia era regular e comprida, e desaparecia ao longe, tanto à
direita quanto à esquerda. À minha frente, estava a calma imensidão de um mar adormecido. Às vezes, parecia que eu
conseguia enxergar um leve brilho de luz sob sua superfície, mas não tinha certeza. Atrás de mim, avultavam, com
altitude extraordinária, penhascos negros e esqueléticos. Acima de mim, o céu tinha uma cor uniforme, cinzenta e fria ---
todo o lugar era iluminado por um imenso globo de fogo pálido que flutuava um pouco acima do horizonte distante e
irradiava uma luz semelhante a espuma sobre as águas plácidas.

Além do suave murmúrio do oceano, prevalecia o silêncio intenso. Durante um bom tempo fiquei ali, olhando para o
horizonte daquele lugar estranho. Enquanto olhava, aparentemente uma bolha de espuma branca emergiu das profundezas e,
mesmo agora sem saber como isso se sucedeu, eu estava olhando, não, fitando a face Dela --- sim! a face Dela ---, fitando
dentro de sua alma; e ela olhava para mim com um misto tão grande de alegria e tristeza que corri, cego, em sua
direção estranhamente gritando, numa verdadeira agonia de recordação, terror e esperança, para que ela viesse até mim.
Todavia, apesar de meus gritos, ela continuava ali, sobre o oceano, e somente balançou a cabeça, com pesar, mas em seu
olhar estava a conhecida luz terrena da ternura, que eu conhecera antes de nos separarmos.

Àquele seu ato de maldade, fiquei desesperado e ensaiei entrar na água e ir até ela, mas, por mais que quisesse, não
consegui. Algo, alguma barreira invisível, segurava"-me, e vi"-me resignado a ficar onde estava e gritar para ela com
todo o meu ser: ``Oh, minha querida, minha querida\ldots{}'' Mas não podia dizer nada além disso, tamanha era a intensidade
de meus sentimentos. Com isso, ela veio rapidamente até mim, e me tocou, e foi como se eu visse o paraíso. Mas,
quando estiquei minhas mãos em sua direção, ela me afastou com mãos ternas e severas, e me senti
embaraçado\ldots{}\footnote{ Aqui, a caligrafia está indecifrável devido ao dano sofrido por esta parte do manuscrito. Seguem-se os
fragmentos legíveis das folhas despedaçadas.}

\section{Fragmentos}

\ldots{} por entre lágrimas\ldots{} o som da eternidade em meus ouvidos, separamo"-nos\ldots{} Ela, a quem amo. Oh, meu
Deus!\ldots{}

Fiquei um bom tempo atordoado, e então estava sozinho na escuridão da noite. Sabia que havia feito a viagem de volta,
mais uma vez, ao universo conhecido. Logo depois, saí daquela enorme escuridão. Eu havia passado por entre as
estrelas\ldots{} um grande período\ldots{} o Sol, distante e remoto.

Penetrei no abismo que separa o nosso sistema dos outros sóis. Enquanto viajava velozmente pela escuridão que os
dividia, observei que o brilho e o tamanho do nosso Sol aumentavam constantemente. Certa vez, olhei de soslaio para
trás, para as estrelas, e as vi mover"-se, por assim dizer, em meu encalço, contra o imenso pano de fundo da escuridão
da noite, tão grande era a velocidade do meu espírito ao passar.

Fiquei mais próximo do nosso sistema e agora podia ver o brilho de Júpiter. Algum tempo depois, pude distinguir o
brilho azulado e frio da luz da Terra\ldots{} Fiquei estupefato. Ao redor do Sol parecia haver objetos luminosos que se
moviam em órbitas rápidas. Mais para dentro, quase perto da intensa glória luminosa do Sol, circulavam dois pontos de
luz que se movimentavam rapidamente e, a certa distância deles, havia um ponto brilhante e azul que eu sabia ser a
Terra. Ela circundava o Sol num espaço de tempo que me parecia ser menor do que um minuto terrestre.

\ldots{} mais perto, mais rápido. Vi o brilho radiante de Júpiter e Saturno, girando com uma velocidade incrível em suas
gigantescas órbitas. E eu continuava a me aproximar do centro e pude ver esse estranho espetáculo --- o movimento
circular e visível dos planetas ao redor do sol"-mãe. Era como se o tempo tivesse sido eliminado para que eu visse
aquilo; de tal maneira que, para meu espírito descarnado, um ano não era mais do que mero momento para uma alma
presa à Terra.

A velocidade dos planetas pareceu aumentar; agora, eu via o Sol circundado por círculos de fogo, finos como fios de
cabelo, de cores diferentes --- os percursos dos planetas, girando violentamente ao redor da chama central\ldots{}

\ldots{} o Sol ficou enorme, como se tivesse saltado na minha direção\ldots{} E agora eu estava dentro das órbitas dos planetas
exteriores, voando rapidamente rumo à Terra, que brilhava através do esplendor azulado de sua órbita, como se através
de uma névoa de fogo, e circundava o Sol numa velocidade abismal\ldots{}\footnote{ Mesmo a mais rigorosa análise não me permitiu decifrar mais do excerto danificado do manuscrito. Ele volta a ser
legível no capítulo intitulado “O ruído na noite”.} 


\clearpage

\chapter{O ruído na noite}

\textsc{E, agora,} chego ao mais estranho dos estranhos acontecimentos que me acometeram nesta casa de mistérios. Ocorreu há
pouco tempo --- faz pouco menos de um mês; e não tenho muitas dúvidas de que o que vi era, na realidade, o fim de todas as
coisas. Todavia, vamos à história.

Não sei o porquê disso, mas, até agora, nunca consegui descrever essas coisas logo depois que aconteciam. É como se eu
precisasse esperar passar algum tempo para recuperar meu equilíbrio e digerir, por assim dizer, as coisas que vi e
ouvi. Sem dúvida, só poderia ser assim, pois, ao esperar algum tempo, acabo vendo os incidentes com mais clareza,
descrevo"-os com mente mais calma e crítica. Isto que escrevo, por exemplo.

Estamos no fim de novembro. Minha história relata o que ocorreu na primeira semana do mês.

Era noite, por volta de onze horas. Pepper e eu fazíamos companhia um ao outro no estúdio --- o antigo, enorme aposento
onde leio e trabalho. Por coincidência, eu lia a Bíblia. Nos últimos tempos, venho me interessando cada vez mais por este
formidável e antigo livro. De repente, a casa estremeceu e ouvi um zunido contínuo, baixo e distante que logo se
transformou num grito abafado ao longe. Estranhamente, lembrou"-me o ruído que faz um relógio quando a presilha é
liberada e deixamos sua corda acabar. O som parecia vir de alguma altitude remota --- em algum ponto do céu noturno. O
tremor não se repetiu. Olhei para Pepper. Ele dormia tranquilo.

Aos poucos, o zunido foi diminuindo e veio um longo silêncio.

De repente, um brilho iluminou a janela mais extrema que se projeta além da lateral da casa, de tal forma que,
dela, é possível olhar tanto para leste como para oeste. Fiquei curioso e, depois de breve hesitação, fui até
o outro lado do estúdio e puxei as cortinas. Quando o fiz, vi o Sol erguer"-se atrás do horizonte. Ele subia com um
movimento contínuo e perceptível. Eu conseguia ver seu movimento. Em um minuto, ou pelo menos assim parecia, ele já
havia alcançado a copa das árvores através das quais eu o observara. Subia e subia --- e já era de dia. Podia ouvir um
zunido agudo, semelhante ao que faz um mosquito, atrás de mim. Olhei ao redor e soube que o ruído vinha do relógio.
Enquanto olhava, vi que ele havia avançado uma hora. O ponteiro dos minutos fazia a volta no mostrador mais rápido do
que um ponteiro de segundos normal. O ponteiro das horas movia"-se rápido de um ponto para o seguinte. Senti um
misto de assombro e torpor. Logo depois, ou assim me pareceu, as duas velas apagaram"-se quase simultaneamente. Virei
rápido para a janela, pois eu vira a sombra dos caixilhos passando pelo chão e vindo na minha direção, como se alguém
tivesse levado uma grande lamparina para o lado de fora.

Agora via que o Sol estava a pino e se movia visivelmente. Passava por cima da casa com um movimento constante e
extraordinário. Quando a janela ficou do lado com sombra, vi outra coisa extraordinária. As finas nuvens não passavam
lentas pelo céu --- corriam em disparada, como se soprasse um vento de cem quilômetros por hora. Quando passavam,
mudavam de forma cerca de mil vezes por minuto, como se se contorcessem com estranha vida, e logo desapareciam. Depois,
outras surgiam, para logo desaparecerem da mesma maneira.

A oeste, vi o Sol cair com um movimento incrível, contínuo e rápido. A leste, as sombras de todas as coisas visíveis
deslizavam rumo ao cinza que surgia. E o movimento das sombras era visível para mim --- o arrastar furtivo e
serpentiforme das sombras das árvores ao vento. Era um estranho espetáculo.

Rapidamente, o estúdio começou a escurecer. O Sol deslizou para o horizonte e pareceu desaparecer de meu campo de visão
quase com um solavanco. Através da luz cinzenta do rápido fim do dia, vi a meia"-lua caindo do céu ao sul, indo em
direção a oeste. O fim do dia pareceu transformar"-se quase que num instante em noite. Acima de mim, as várias
constelações rumavam num movimento circular e silencioso para oeste. A Lua caiu por entre as últimas profundezas do
abismo da noite, e só restou a luz das estrelas\ldots{}

A essa altura, o zunido no canto do estúdio já havia cessado, o que indicava que o relógio estava sem corda. Alguns
minutos passaram"-se e vi o céu a leste clarear. Uma manhã cinzenta e melancólica cobriu a escuridão e ocultou a marcha
das estrelas. Lá em cima, movia"-se, com um desenrolar eterno e pesado, um céu vasto e sem fronteiras de nuvens
cinzentas --- um céu nublado que, na duração de um dia ordinário na terra, teria parecido imóvel. O Sol estava oculto,
mas, de vez em quando, o mundo clareava e escurecia, clareava e escurecia, debaixo das ondas de luz e sombras sutis\ldots{}

A luz continuava a rumar para o oeste e a noite caiu sobre a Terra. Uma grande chuva pareceu vir com ela, e um vento
com ruído quase ensurdecedor --- como se toda a ventania de uma noite estivesse condensada no período de menos de um
minuto.

Esse ruído sumiu quase que imediatamente e as nuvens abriram"-se; assim, eu podia, mais uma vez, ver o céu. As estrelas
voavam para o oeste com velocidade impressionante. Agora eu percebia, pela primeira vez, que, embora o ruído do vento
tivesse desaparecido, continuava a ouvir um som “borrado” e constante. Agora que o notava, me dei conta de que o estivera
ouvindo durante todo esse tempo. Era o ruído do mundo.

Então, no mesmo instante em que compreendia tantas informações, veio a luz do leste. Em poucos segundos, o Sol
ergueu"-se rápido. Vi"-o através das árvores e logo ele estava acima delas. Subiu e subiu, altivo, e todo o mundo
ficou iluminado. Passou com um movimento ágil e constante para o ponto mais alto de sua trajetória e de lá caiu para
oeste. Vi o dia desenrolar"-se visivelmente sobre minha cabeça. Algumas poucas e leves nuvens flutuaram para o norte e
desapareceram. O Sol mergulhou com uma queda célere e certeira, ao meu redor vi, durante alguns segundos, o tom cada
vez mais cinzento do crepúsculo.

A sudoeste, a Lua descia ligeira. A noite já havia chegado. Depois do que pareceu um minuto, a Lua mergulhou nas
profundezas que restavam do céu escuro. Mais um minuto e o céu a leste brilhava com o poente que surgia. O Sol saltou
acima de mim com uma rapidez assustadora e subiu ainda mais rápido para o seu zênite. Então, de repente, algo novo
surgiu no horizonte. Uma nuvem negra de tempestade corria, vinda do sul, e pareceu dominar todo o céu num único
instante. Quando veio, vi que a extremidade que avançava era recurva, como se fosse um tecido negro e monstruoso no
céu, retorcendo"-se e ondulando"-se veloz, com um movimento horrível. Num átimo, todo o ar encheu"-se de chuva e
surgiu uma centena de clarões de relâmpago, como se fossem uma única torrente. No mesmo segundo, o ruído do mundo foi
abafado pelo urrar do vento e senti meus ouvidos doer sob o impacto ensurdecedor dos trovões.

Em meio a essa tempestade, veio a noite; então, dentro do espaço de um minuto, a tempestade cessara e só havia o
ruído constante e indistinto do mundo em meus ouvidos. Lá em cima, as estrelas deslizavam rapidamente para oeste;
algo, talvez a velocidade específica com que elas surgiam, fez"-me perceber, pela primeira vez, de maneira bastante
nítida, que era o mundo que girava. De repente, senti que o mundo, com sua enorme e escura massa, revolvia visivelmente
contra o céu estrelado.

A alvorada e o Sol pareceram surgir juntos, tão rápida era a progressão da velocidade do mundo. O Sol subiu numa curva
comprida e constante; passou de seu ponto mais alto e desceu para o céu a oeste, desaparecendo. Eu mal percebi o cair
da noite, de tão breve. Logo enxergava as constelações que voavam e a precipitação a oeste da Lua. No espaço de meros
segundos, ela deslizou ligeira pelo azul noturno e desapareceu. Quase que imediatamente, veio a manhã.

Nesse momento pareceu haver uma grande aceleração. O Sol desenhou um caminho certeiro pelo céu, desapareceu atrás do
horizonte a oeste e a luz veio e se foi com igual pressa.

À medida que o dia seguinte abria"-se e fechava"-se sobre o mundo, percebi de repente uma camada de neve sobre a terra. A
noite veio e, quase que em seguida, o dia. Durante o breve salto do Sol, vi que a neve havia desaparecido; e então,
mais uma vez, era noite.

E assim tudo se passou; mesmo depois de todas as coisas incríveis que testemunhei, eu sentia o tempo todo imensa
perplexidade. Ver o Sol erguer"-se e se pôr dentro de um espaço de segundos; ver (depois de uma pausa) a Lua saltar ---
uma esfera pálida, cada vez maior --- e subir para o céu da noite, e deslizar, com estranha rapidez, através do enorme
arco da noite azul; e, logo depois, ver o Sol em seu encalço, despontando no céu a leste, como se a perseguisse; e
de novo a noite, com a passagem veloz e fantasmagórica das constelações de estrelas; tudo isso fazia com que eu não
conseguisse acreditar em meus olhos. Mas assim se sucedeu --- o dia passando da alvorada para o anoitecer, a
noite transformando"-se rapidamente em dia, cada vez mais rápido e mais rápido.

As últimas três passagens do Sol mostraram"-me a terra cheia de neve, o que, à noite, durante alguns segundos, pareceu
extremamente estranho sob a luz cada vez mais rápida da Lua que se erguia e caía. Mas agora, durante um breve
intervalo, o céu ficou oculto por um oceano de nuvens oscilantes cor de chumbo, que alternavam entre mais claras
e mais escuras com a passagem do dia e da noite.

As nuvens propagavam"-se e desapareciam, e mais uma vez estava adiante de mim a visão do Sol que subia rápido, e as
noites vieram e se foram como sombras.

Cada vez mais rápido girava o mundo. Agora, cada dia e cada noite encerravam"-se dentro do espaço de meros segundos;
ainda assim, a velocidade aumentava.

Foi algum tempo depois que percebi que o Sol começara a deixar um leve rastro de fogo atrás de si. Isso, claro,
se devia à velocidade com que ele, aparentemente, percorria os céus. À medida que os dias avançavam, cada um mais
rápido que o anterior, o Sol começou a adquirir a aparência de um enorme cometa flamejante,\footnote{ O Eremita usa isso
como metáfora, evidentemente no sentido da concepção leiga de um cometa.} refulgindo pelo céu em
curtos intervalos periódicos. À noite, a Lua tinha, com maior semelhança, o aspecto de um cometa; era uma forma de fogo
pálida, bastante luminosa, que viajava rápido e deixava rastros de uma chama fria. As estrelas agora pareciam apenas
meros fios de fogo contra a escuridão.

Certa vez, dei as costas à janela e olhei rápido para Pepper. Durante o breve lampejo de um dia, vi que ele dormia,
tranquilo, e voltei a observar.

O Sol agora surgia repentino no horizonte a leste como um assombroso foguete, parecendo ocupar não mais do que um
segundo ou dois em sua rota de leste a oeste. Eu não conseguia mais ver a passagem das nuvens no céu, que agora parecia
um pouco mais escuro. As noites breves pareciam ter perdido sua peculiar escuridão, de tal forma que os fiapos de fogo
das estrelas que voavam apareciam apenas de leve. À medida que a velocidade aumentava, o Sol começou a oscilar bem
lentamente no céu, do sul para o norte, e, depois, também devagar, do norte para o sul.

Assim, em meio à minha estranha confusão mental, as horas se passaram.

Durante todo esse tempo, Pepper dormia. Logo depois, sentindo"-me solitário e perturbado, chamei baixinho o seu nome,
mas ele não atendeu. Chamei de novo, erguendo um pouco a voz; ele continuava imóvel. Fui até onde ele estava e o
toquei com o pé para acordá"-lo. Com isso, por mais suave que fosse meu movimento, ele caiu em pedaços. Foi exatamente
isso que aconteceu: ele literalmente se desfez, transformando"-se numa pilha de ossos e pó.

Durante o intervalo de talvez um minuto fiquei olhando sem parar aquele amontoado disforme que fora Pepper. Fiquei ali
parado, atônito. ``O que poderia ter acontecido?'', perguntei a mim mesmo, sem entender o triste significado daquele
pequeno montinho de cinzas. Então, mexendo nele com o pé, ocorreu"-me que isso só poderia ter acontecido num grande
intervalo de tempo\ldots{} Anos e anos. 

Lá fora, a luz trêmula que dava voltas dominava o mundo. Do lado de dentro, eu continuava ali, tentando entender o
significado daquilo --- o que significava aquele pequeno monte de cinzas e ossos secos sobre o tapete. Mas não conseguia
raciocinar de maneira coerente.

Olhei ao redor do aposento e agora, pela primeira vez, percebi quanto o lugar parecia velho e cheio de pó. Havia pó e
sujeira em todos os lugares; em pequenos montinhos nos cantos, cobrindo toda a mobília. O próprio tapete estava
invisível sob uma camada daquela mesma matéria, que tudo permeava. Enquanto eu caminhava, pequenas nuvens dessa substância
subiam sob o movimento de meus pés e o pó chegava até minhas narinas, com um odor seco e amargo que me fazia arfar e
respirar com dificuldade.

Então, enquanto pousava outra vez o olhar sobre os restos mortais de Pepper, fiquei ereto e dei asas à minha confusão
--- questionando, em voz alta, se os anos de fato estavam passando; e se o que eu imaginara ser um tipo de visão era
de fato realidade. Parei de repente. Tive uma ideia. Rápido, mas com passos que, pela primeira vez,
percebi, vacilavam, atravessei o aposento até o grande espelho do tremó e olhei. Também estava sujo demais para ter
reflexo e, com as mãos trêmulas, comecei a esfregar a superfície para tirar o pó. Logo, pude ver a mim mesmo.
O pensamento que me ocorrera confirmou"-se. Em vez do homem grande e saudável que mal aparentava ter cinquenta anos de
idade, eu agora olhava para um homem decrépito e encurvado, de ombros caídos, cujo rosto portava as rugas de um século.
O cabelo --- que havia poucas horas era quase totalmente negro --- agora era de um branco platinado. Só os olhos brilhavam.
Aos poucos, percebi naquele homem idoso uma leve semelhança com o meu eu de dias pregressos.

Eu me virei e caminhei com passos trôpegos até a janela. Agora sabia que estava velho, e isso parecia confirmar meu
andar trêmulo. Durante um breve período, fiquei observando melancólico a paisagem borrada do cenário que mudava. Mesmo
naquele breve período, um ano havia se passado e, com um gesto de irritação, saí da janela. Quando o fiz, percebi em
minha mão os tremores da idade avançada; um soluço embargado irrompeu de meus lábios.

Durante algum tempo, caminhei, trêmulo, entre a janela e a mesa; meu olhar vagava aqui e acolá, incerto. Como o
aposento estava dilapidado. Em todos os lugares havia aquela grossa camada de pó --- espessa, inerte, negra. A proteção
da lareira agora era só uma estrutura feita de ferrugem. As correntes que seguravam os pesos de latão do relógio
ficaram enferrujadas havia tempos, e agora os pesos estavam no chão; eles mesmos somente dois cones de azinhavre.

Olhando ao redor, era como se eu pudesse ver a mobília do aposento apodrecendo e se desintegrando diante de meus
olhos. E isso não era imaginação: pois, no mesmo instante, a estante de livros que ladeava a parede caiu, com
um som de madeira apodrecida quebrando e cedendo, precipitando seu conteúdo no chão, enchendo o estúdio duma sufocante
nuvem de partículas de pó.

Como eu me sentia cansado. Enquanto andava, era como se pudesse ouvir as próprias articulações ressequidas ranger
e estalar a cada passo. Pensei em minha irmã. Estaria ela morta, assim como Pepper? Tudo havia acontecido tão rápido,
de maneira tão repentina. Tudo aquilo deveria mesmo ser o início do fim de todas as coisas! Ocorreu"-me que deveria sair
à procura dela, mas me sentia totalmente exaurido. Além disso, ela estava se comportando de maneira tão estranha quanto
a esses acontecimentos ultimamente. Ultimamente! Repeti a palavra e soltei um riso débil; um riso amargo, ao perceber
que falava de uma época de meio século atrás. Meio século! Talvez fosse o dobro disso!

Fui devagar até a janela e de novo prestei atenção ao mundo. A melhor maneira de descrever a passagem do dia e da
noite, naquele estágio, é dizer que era um enorme e forte clarão de luz. A cada momento, a aceleração do tempo
continuava; de tal forma que, durante a noite, eu via a Lua somente como um rastro oscilante de fogo pálido, que
variava de mera linha de luz até um percurso nebuloso, e depois diminuía de novo, desaparecendo periodicamente.

O clarão dos dias e das noites ficou mais rápido. Os dias ficaram perceptivelmente mais escuros, e uma estranha luz de
alvorecer pairava na atmosfera. As noites estavam tão mais claras que mal era possível ver as estrelas, exceto aqui e
ali na forma de uma linha ocasional, fina como um fio de cabelo, que parecia oscilar um pouco com a Lua.

Rápido, cada vez mais rápido, surgia o clarão que alternava o dia e a noite; aparentemente num átimo, percebi que
esse clarão havia sumido e, em seu lugar, restava uma luz quase constante que caía sobre todo o mundo, vinda do
eterno fluxo de labaredas que se agitavam para cima e para baixo, para o norte e para o sul, em imensos e assombrosos
movimentos oscilantes.

O céu agora estava bem mais escuro, e havia em seu azul uma pesada penumbra, como se a vasta escuridão espiasse a Terra
através dele. Mas havia nele uma clareza estranha e terrível, e também um vazio. Periodicamente, eu vislumbrava um leve
caminho de fogo pálido que oscilava, fino e escuro, rumo ao percurso do Sol; desaparecia e reaparecia. Era o trajeto da
Lua, quase invisível.

Observando a paisagem, mais uma vez fiquei ciente de um ``adejo'' difuso que vinha ou da luz do pesado e oscilante fluxo do
Sol ou era resultado das mudanças incrivelmente rápidas na superfície da Terra. Periodicamente, a neve parecia
surgir de repente sobre o mundo e desaparecer de maneira igualmente abrupta, como se um gigante invisível agitasse
um lençol branco sobre a Terra.

O tempo voava e o cansaço que eu sentia passou a ficar insuportável. Dei as costas à janela e fui até o outro lado do
estúdio, a poeira pesada abafando o som de meus passos. A cada passo que eu dava, maior era a força que precisava
despender. Uma dor intolerável acometia cada um de meus membros e articulações, e eu tentava traçar meu caminho com
passos incertos e cansados.

Na parede oposta, fiz uma pausa, sentindo"-me fraco, e tentei lembrar o que tencionava fazer. Olhei para a
esquerda e vi minha velha cadeira. A ideia de me sentar nela trouxe breve sensação de conforto à minha perplexa
decrepitude. Como eu estava tão exausto e velho, só conseguia manter"-me de pé e desejar percorrer aqueles poucos
metros. Continuei parado, trôpego. Agora, até o chão parecia um bom lugar para descansar, mas a camada de poeira era
tão espessa, imóvel, negra. Com grande esforço, virei e fui em direção à cadeira. Cheguei a ela com um gemido de
gratidão. Sentei.

Tudo ao meu redor parecia começar a ficar indistinto. Tudo era tão estranho, tão inconcebível. Na noite anterior, eu era
relativamente forte, por mais que fosse um homem já de idade; agora, poucas horas depois\ldots{}! Olhei para o pequeno
montinho de pó que fora Pepper. Horas! E ri, um riso fraco, amargo; um riso estridente e desafinado que chocou meus
sentidos cada vez mais fracos.

Durante um breve período de tempo, devo ter cochilado. Então abri os olhos, sobressaltado. Em algum lugar do
aposento, ouvira o som abafado de algo caindo. Procurei sua origem e vi, vagamente, uma nuvem de poeira pairando sobre
uma pilha de destroços. Mais perto da porta, o som de mais alguma coisa caindo com um estrondo. Era um dos armários,
mas eu estava cansado e não prestei atenção. Fechei os olhos e continuei sentado, num estado sonolento de
semiconsciência. Em uma ou outra ocasião, como se através de uma grossa névoa, ouvi ruídos distantes. Então devo ter
adormecido.


\clearpage

\chapter{O despertar}

\textsc{Acordei sobressaltado.} Durante alguns segundos, tentei adivinhar onde estava. E então me lembrei\ldots{}

O aposento ainda era iluminado por aquela estranha luz --- metade do Sol, metade da Lua. Eu me sentia revigorado e a dor
e cansaço de antes haviam sumido. Fui lentamente até a janela e olhei. Lá em cima, o rio de chamas subia e descia,
para o norte e para o sul, num semicírculo dançante de fogo. Como uma poderosa naveta no tear do tempo, ele parecia ---
num repentino devaneio meu --- afastar as agulhas do tempo. Pois a passagem do tempo tinha sido acelerada de tal modo que
não havia mais a sensação do Sol passando do leste para o oeste. O único movimento aparente era o oscilar do percurso
do Sol do norte para o sul, e agora ele estava tão rápido que seria mais bem descrito como um tremor.

Enquanto observava, lembrei"-me de repente, de modo inconsequente, da minha última viagem para os mundos Distantes.\footnote{ Evidentemente, 
uma referência a algo que se sucedeu nas páginas destruídas e faltantes do manuscrito. Ver ``Fragmentos'', no capítulo
``O Mar Adormecido''.} Lembrei"-me da visão repentina que tivera enquanto me aproximava do Sistema Solar, dos planetas que giravam
rápidos ao redor do Sol --- como se o poder do tempo tivesse sido colocado em suspenso e a Máquina do Universo houvesse permitido que
uma eternidade se passasse em poucos momentos ou horas. A memória foi"-se e deixou a sugestão, embora apenas
parcialmente compreendida, de que a mim fora permitido vislumbrar o futuro. Olhei mais uma vez para o tremor aparente
do fluxo do Sol. A velocidade parecia aumentar até mesmo enquanto eu observava. Diversas vidas passaram"-se enquanto eu
olhava.

De repente, dei"-me conta, com seriedade um tanto grotesca, de que ainda estava vivo. Pensei em Pepper e tentei imaginar
por que é que não segui o mesmo destino. Sua hora chegara e era provável que ele falecera em decorrência da mera passagem
dos anos. Porém lá estava eu vivo, centenas de milhares de séculos após os anos de vida a que tinha direito.

Durante algum tempo, ensimesmado, refleti. ``Ontem\ldots{}'' Mas parei, de repente. Ontem! Não havia ontem. O ontem do
qual falava fora engolido pelo abismo de anos, de eras que se passaram. Fiquei atordoado com esse pensamento.

Logo depois, dei as costas à janela e olhei ao redor. O aposento parecia diferente --- estranha e completamente
diferente. Então percebi o que é que o fazia parecer tão estranho. O estúdio estava nu: não havia nenhuma mobília;
nem mesmo uma decoração sequer. Aos poucos, meu assombro foi"-se dissipando à medida que eu me dava conta
de que aquilo era o fim inevitável do processo de deterioração, cujo início eu testemunhara antes de adormecer.
Milhares de anos! Milhões de anos!

Sobre o assoalho havia uma profunda camada de pó que quase alcançava o assento da janela. Aumentara imensuravelmente
enquanto eu dormia; era o pó de eras e eras sem fim. Sem dúvida, os átomos da mobília velha e deteriorada ajudaram a
aumentar a quantidade de pó; em algum ponto ali, no meio de tudo, estava Pepper, morto havia muito.

No mesmo instante, ocorreu"-me que eu não tinha nenhuma lembrança de ter andado com poeira até a altura dos joelhos
depois de despertar. É verdade que uma infinidade de anos havia se passado desde que eu me aproximara da janela, mas
isso evidentemente não era nada se comparado com os incontáveis períodos de tempo que, imaginava eu, passaram"-se
enquanto eu dormia. Agora, lembrava que eu havia adormecido sentado na minha velha cadeira. Será que ela havia\ldots{}?
Olhei de relance para onde ela costumava ficar. Obviamente, não existia mais cadeira alguma. Não conseguia determinar se
ela tinha desaparecido antes ou depois de eu acordar. Se tivesse se desfeito em pó embaixo de mim, eu certamente teria
acordado com a queda. Mas então lembrei que a espessa poeira que cobria o chão teria sido suficiente para amortecer
minha queda; assim, era bem possível que eu houvesse dormido sobre a poeira durante um milhão de anos ou mais.

Enquanto esses pensamentos voavam em minha mente, olhei rapidamente, mais uma vez, por acaso, para o ponto onde ficava
minha cadeira. Então, pela primeira vez, percebi que não havia marcas na poeira das minhas pegadas entre a cadeira e
a janela. Mas então eras haviam se passado desde que eu acordara --- dezenas de milhares de anos!

De novo meu olhar pousou, pensativo, sobre o local onde antes estivera minha cadeira. De repente, passei da
abstração para a atenção; pois ali, naquele local, discerni uma ondulação comprida, arredondada pela pesada poeira.
Todavia, o pó não cobria tudo a ponto de eu não ser capaz de precisar o que era aquilo. Eu soube --- e estremeci ao me dar conta
--- que aquilo era um corpo humano, morto havia eras, deitado bem embaixo do local onde eu adormecera. Estava deitado do
lado direito, com as costas voltadas para mim. Eu podia discernir cada curva e contorno, suavizado e desfeito, por
assim dizer, no pó enegrecido. De maneira meio vaga, tentei imaginar a origem de sua presença. Devagar, comecei a
ficar estupefato ao pensar que ele estava exatamente onde eu poderia ter caído quando a cadeira se desfez.

Aos poucos, a ideia começou a se formar no meu cérebro; um pensamento que abalava minha alma. Parecia horrível,
insuportável; todavia, foi crescendo em minha mente, cada vez mais, até tornar"-se uma convicção. O corpo sob aquela
camada, sob aquele manto de pó, nada mais era do que meu corpo morto. Não tentei provar minha hipótese. Eu
agora sabia e fiquei pensando se não desconfiara o tempo todo. Agora, eu era algo imaterial.

Fiquei ali um tempo, parado, tentando ajustar meus pensamentos a esse novo problema. Com o tempo --- quantos milhares de
anos, não sei dizer --- consegui alcançar certo grau de sossego, o suficiente para me fazer prestar atenção ao que se
passava ao meu redor.

Agora, eu via que o monte de pó alongado tinha afundado e ficado no mesmo nível do resto da poeira que tudo cobria.
Novos átomos de poeira, impalpáveis, haviam se depositado sobre a mistura do pó do cadáver moído pelas eras. Durante um
bom tempo continuei ali parado, de costas para a janela. Gradualmente, fui me recompondo, enquanto o mundo deslizava
através dos séculos rumo ao futuro.

Logo depois, passei a observar o aposento. Agora, via que o tempo dava início a sua ação destrutiva, mesmo nesta
estranha e antiga construção. O fato de esta casa ter aguentado todos esses anos parecia"-me prova de que ela era
diferente de qualquer outra. Acho que nunca pensei que ela pudesse se decompor. Não saberia dizer por quê. Foi só
depois de meditar sobre o assunto durante considerável período de tempo que me dei conta de que o tempo
extraordinário por que ela passou incólume já teria sido suficiente para pulverizar as próprias pedras sobre as quais
fora construída, caso essas pedras fossem oriundas de rochas naturais. Sim, sem dúvida a casa estava se decompondo agora.
Todo o gesso das paredes havia sumido, assim como toda a madeira do aposento desaparecera, muitas eras antes.

Enquanto estava ali, em contemplação, um pedaço de vidro de uma das pequenas vidraças em forma de losango caiu, com um
ruído surdo, na poeira que cobria o peitoril atrás de mim e se desfez num pequeno montinho de pó. Quando me virei,
depois de contemplá"-lo, vi luz penetrar pelas frestas de algumas das pedras que formavam a parede externa.
A argamassa estava caindo\ldots{}

Depois de algum tempo, me virei mais uma vez para a janela e observei a paisagem lá fora. Agora descobria que a velocidade
do tempo estava absurda. O movimento lateral do feixe do Sol havia ficado tão rápido que fez com que o dançante
semicírculo em chamas se unisse ao lençol de fogo que cobria metade do céu a sul, de leste para oeste, desaparecendo
nele. 

Voltei os olhos para os jardins. Eram só um borrão verde, de tom pálido e encardido. Tinha a impressão de que estavam
mais altos do que antigamente e também a sensação de que estavam mais próximos da janela, como se o solo tivesse
ficado mais alto. Mesmo assim, os jardins ainda estavam bem abaixo de mim, pois a rocha na boca do fosso, sobre a qual
se ergue esta casa, faz um arco bem alto.

Foi mais tarde que notei uma mudança na cor constante dos jardins. O verde encardido e pálido ficava cada vez mais
claro, mais para o branco. Finalmente, depois de um grande intervalo, os jardins adotaram um tom branco acinzentado, e
assim permaneceram durante um bom tempo. Contudo, por fim, o tom cinzento começou a desaparecer, assim como o verde, e
transformou"-se num branco morto. E assim permaneceu, constante e imutável. E foi assim que soube que, finalmente, a
neve cobria todo o Hemisfério Norte.

Então, com o passar de milhões de anos, o tempo continuou avançando por toda a eternidade, até o fim --- o fim a
respeito do qual, em meus antigos dias terrestres, eu havia refletido vagamente, de maneira incerta e especulativa.
Agora ele se aproximava de uma maneira nunca imaginada.

Lembro que, a essa altura, comecei a sentir uma curiosidade bastante aguda, embora mórbida, a respeito do que
aconteceria quando viesse o fim --- mas, estranhamente, não conseguia imaginar nada.

Enquanto isso, o processo constante de decomposição continuava. Os últimos pedaços de vidro fazia muito haviam
desaparecido; de vez em quando, um baque leve e surdo e uma pequena nuvem de fumaça indicavam que algum fragmento de
pedra ou argamassa caía.

Olhei de novo para cima, para o lençol de fogo que estremecia nos céus acima de mim e alcançava os confins ao sul.
Enquanto olhava, tive a impressão de que ele havia perdido parte de seu brilho --- que tinha um tom mais profundo e
opaco.

Olhei mais uma vez para o borrão branco da paisagem do mundo. Às vezes, meu olhar voltava"-se para o lençol ardente de
chamas embotadas que ao mesmo tempo era e ocultava o Sol. Às vezes, olhava atrás de mim, para a escuridão cada vez
maior do enorme e silencioso aposento, com seu tapete de eras de pó adormecido\ldots{}

E assim continuei observando, através das eras efêmeras, perdido em pensamentos e divagações que me exauriam a alma,
sentindo um cansaço até então desconhecido.


\clearpage

\chapter{A rotação cada vez mais lenta}

\textsc{Talvez um} milhão de anos depois, percebi, sem a menor sombra de dúvida, que o lençol de fogo que iluminava o mundo
estava de fato ficando mais escuro.

Outro enorme período de tempo passou"-se e toda a gigante chama havia adquirido uma cor acobreada e escura.
Gradualmente, foi ficando mais escura, passando de cobre para vermelho acobreado, e, a partir daí, às vezes adquiria um
matiz escuro e arroxeado e, com ele, um estranho tom de sangue.

Embora a luz estivesse diminuindo, eu não conseguia perceber nenhuma diminuição na velocidade aparente do Sol. Ele
ainda dominava o céu em seu atordoante véu de velocidade. 

O mundo, ou pelo menos tudo o que dele eu conseguia enxergar, havia adquirido uma terrível penumbra, como se, de fato,
os últimos dias do Universo se aproximassem.

O Sol estava morrendo; disso, não havia muita dúvida; mesmo assim a Terra continuava girando, através do espaço e de
toda a eternidade. A essa altura, lembro que uma extraordinária sensação de confusão tomou conta de mim. Depois
percebi que minha mente vagava por entre um caos de fragmentos de teorias modernas e a antiga história bíblica sobre o
fim do mundo.

Então, pela primeira vez, surgiu"-me qual um clarão a lembrança de que o Sol, com seu sistema de planetas, estava, e
estivera, viajando pelo espaço com uma velocidade incrível. De repente, surgiu a questão: para onde? Durante um bom
tempo, fiquei pensando nisso, mas, finalmente, sentindo um pouco quão fúteis eram esses questionamentos, deixei meus
pensamentos voltarem"-se para outras coisas. Fiquei tentando imaginar durante mais quanto tempo a casa continuaria a
existir. Também fiquei pensando se estaria condenado a permanecer desencarnado sobre a Terra durante toda a era de
trevas que eu sabia estar a caminho. A partir desses pensamentos, de novo passei a especular sobre a possível direção da
jornada do Sol através do espaço\ldots{} E, assim, mais um bom tempo se passou.

Gradualmente, à medida que o tempo avançava, comecei a sentir o frio de um inverno rigoroso. Então lembrei que, com a
morte do Sol, o frio necessariamente seria absurdamente intenso. Devagar, bem devagar, à medida que eras
transformavam"-se em eternidade, a Terra mergulhava numa penumbra cada vez mais escura e vermelha. A chama opaca no
firmamento passou a ter um matiz mais escuro, bastante sombrio e turvo.

Então, finalmente, percebi uma mudança. A sombria cortina de chamas que estremecia nos céus e nos confins ao sul
começou a ficar mais fina, a contrair"-se; nela, assim como alguém vê as rápidas vibrações da corda de uma harpa, observei
mais uma vez o feixe do Sol estremecer, de maneira estonteante, para o norte e para o sul.

Aos poucos, a aparência de lençol de fogo foi sumindo, e eu agora via com clareza a lenta morte do feixe do Sol.
Todavia, mesmo então, a velocidade de sua passagem era inconcebivelmente rápida. Durante todo esse processo, o brilho
daquele arco de fogo ficava cada vez mais opaco. Abaixo dele, o mundo assomava com aparência turva --- uma região
fantasmagórica, indistinta.

Lá em cima, o rio de chamas oscilava lento, cada vez mais lento, até que, finalmente, passou a balançar entre o norte e
o sul em grandes e pesados intervalos que duravam segundos. Um longo período passou"-se e agora cada movimento do
enorme cinturão durava quase um minuto; de tal forma que, depois de algum tempo, deixei de perceber um movimento
aparente; e o fluxo de fogo corria num rio constante de chama opaca através do céu aparentemente morto.

Um período de tempo indefinido passou"-se, e parecia que o arco de fogo tornava"-se cada vez menos definido. Parecia
também ficar cada vez mais atenuado, e acredito ter visto listras negras de vez em quando. Logo depois, o constante fluxo
cessou e pude perceber um escurecimento momentâneo, mas regular, do mundo. Ele foi aumentando até que, mais uma vez, a
noite caía em intervalos breves, mas periódicos, sobre a Terra cada vez mais fatigada. 

As noites tornaram"-se cada vez mais longas e os dias se igualavam a elas em extensão até que, finalmente, o dia e a
noite se expandiram até durar segundos, e o Sol aparecia, mais uma vez, como uma bola quase invisível de cor
vermelho"-cobre, envolvido pela névoa brilhante de seu trajeto. Correspondentes às linhas negras que às vezes apareciam
em seu encalço, havia agora grandes cinturões escuros bastante nítidos sobre o próprio Sol semivisível.

Anos e mais anos passaram, e os dias e as noites alargavam"-se em minutos. O Sol deixou de ter uma cauda, agora nascia
e se punha --- um enorme globo de matiz acobreado em chamas; em algumas partes, tinha anéis cor de sangue; em outras, os
anéis escuros que já mencionei. Esses círculos --- tanto os vermelhos como os negros --- tinham espessuras variadas.
Durante algum tempo, fiquei confuso demais para entender qual era sua causa. Então me ocorreu que era bem pouco
provável que o Sol esfriasse de maneira uniforme e que essas marcas provavelmente se deviam a diferenças de
temperatura em várias áreas; as vermelhas representavam as partes em que o calor ainda era intenso e as negras,
as partes que já estavam relativamente mais frias.

Havia pensado quão peculiar era o Sol esfriar em anéis perfeitamente definidos, até dar"-me conta de que possivelmente
eram somente pedaços isolados aos quais a velocidade da rotação do Sol dera a aparência de cinturões. O próprio Sol era
infinitamente maior do que o Sol que eu conhecera em minha vida no antigo mundo; com isso, supus que estivesse bem
mais próximo.

Durante a noite, a Lua\footnote{ Não há mais nenhuma menção à Lua. A julgar
por esta descrição, resta evidente que nosso satélite ficou bem mais
distante da Terra. Possivelmente, em alguma era posterior, talvez tenha até se
desprendido de nossa gravidade. Lamento que não haja mais esclarecimentos sobre
isso.} ainda aparecia, mas pequena e remota; a luz que refletia era
tão baça e fraca que ela era pouco mais que o pequeno e turvo fantasma da antiga Lua.

Aos poucos, os dias e noites foram"-se estendendo até se igualarem a um período aproximado de uma das antigas horas
terrestres; o Sol erguia"-se e se punha como um enorme disco de bronze avermelhado, cruzado por faixas negras como
tinta. A essa altura, descobri que conseguia, mais uma vez, ver os jardins com clareza. Pois o mundo agora estava
bastante imóvel e imutável. Mas não seria correto dizer ``jardins'', pois não havia jardins --- nada que eu conhecesse ou
reconhecesse. No lugar deles, eu via uma vasta planície que se desenrolava até o horizonte. Um pouco para a esquerda,
havia uma cadeia baixa de morros. Por todos os lados, estava a cobertura branca da neve, uniforme, que em alguns pontos
formava colinas e cristas.

E só agora percebia quanto fora intensa a nevasca. Em alguns pontos, a neve era muitíssimo profunda, prova
disso era o enorme monte em forma de onda, mais a distância, a minha direita, mas talvez isso se devesse em parte a alguma
elevação na superfície do solo. Era estranho, mas a cadeia de pequenos morros à minha esquerda não estava completamente
coberta por essa neve onipresente; eu podia ver as laterais nuas e escuras dos morros, aparentes em diversos pontos. Em
tudo reinavam o incrível silêncio mortal e a sensação de desolação. O silêncio imutável e terrível de um mundo a
caminho da morte.

Durante todo esse tempo, os dias e as noites aumentaram perceptivelmente. Cada dia já ocupava, talvez, umas duas horas,
desde o amanhecer até o anoitecer. À noite, fiquei surpreso ao descobrir que havia bem poucas estrelas no céu e que
eram pequenas, embora fossem de um brilho extraordinário; atribuí tal brilho ao peculiar, mas límpido, negrume do
período noturno.

A distância, mais para o norte, podia discernir uma espécie de névoa nebulosa; não muito diferente, em aparência, de um
pedaço da Via Láctea. Talvez fosse um agrupamento de estrelas extremamente remoto ou --- pensei de repente --- talvez
fosse o Universo sideral que eu conhecera e que agora era deixado para trás, para todo o sempre --- uma pequena névoa de
estrelas que mal brilhava, bem distante, nos confins do espaço.

Ainda assim, os dias e as noites aumentavam devagar de tamanho. A cada vez, o Sol erguia"-se mais opaco do que quando
havia desaparecido e os cinturões negros aumentavam em largura.

Mais ou menos neste ponto algo novo aconteceu. O Sol, a Terra e o céu escureceram de repente e, aparentemente,
ficaram ocultos durante um breve período. Eu tinha uma sensação, certa consciência (não podia ver muita coisa) que a
Terra era fustigada por uma grande nevasca. Então, num instante, o véu que tudo obscurecera desapareceu e pude
enxergar mais uma vez e o que vi foi uma paisagem incrível. A depressão onde fica esta casa, com seus jardins,
estava completamente tomada pela neve.\footnote{ Provavelmente, o ar congelado.} 
A neve estava acima do parapeito da janela. Estava em todos os lugares, uma
enorme e branca planície nivelada que refletia, de maneira melancólica, o brilho cor de cobre e sombrio do Sol que
morria. O mundo inteiro transformara"-se numa planície sem sombras, de um horizonte ao outro.

Olhei para o Sol. Tinha um brilho opaco e extraordinário. Eu agora o via como alguém que, até então, só conseguira
enxergá"-lo através de um filtro que o obscurecia parcialmente. Ao redor dele, o céu estava escuro, uma escuridão
abissal, aterradora em sua proximidade, em profundidade sem limite, em sua total hostilidade. Durante bom tempo,
fiquei olhando para ele de maneira nova, abalado e temeroso. Parecia tão próximo. Fosse eu uma criança, talvez tivesse
expressado parte da sensação e aflição que sentia dizendo que o telhado do céu havia sumido.

Depois, virei e olhei ao redor do aposento. Tudo estava coberto com uma leve camada branca e difusa. Eu
só conseguia enxergar um pouco devido à luz sombria que agora iluminava o mundo. Essa camada parecia agarrar"-se às
paredes em ruínas; o pó espesso e leve de todos aqueles anos, que cobria o chão até os joelhos, havia sumido por
completo. A neve talvez tivesse entrado pelos caixilhos abertos das janelas. Todavia, em nenhum local ela fazia volume,
estava presente em todos os lugares do enorme e antigo aposento, lisa e nivelada. Além disso, não houvera vento algum
em todos esses milhares de anos. Mas havia a neve,\footnote{ Ver a nota
anterior. Isso explicaria a ``neve'' dentro do aposento.} como eu disse.

E toda a Terra estava em silêncio. E fazia um frio que nenhum homem vivo poderia ter imaginado.

A Terra agora era iluminada, durante o dia, por uma luz extremamente pesarosa, além da minha capacidade de descrição. Era
como se eu estivesse olhando para uma grande planície através de um mar de tons de bronze.

Era evidente que o movimento rotatório da Terra divergia de modo constante.

E o fim veio de uma vez. A noite fora a mais longa até então; quando o Sol que morria finalmente surgiu sobre a
extremidade do mundo, eu estava tão cansado da escuridão que o recebi como a um amigo. O Sol ergueu"-se num movimento
regular até uns vinte graus acima do horizonte. Então parou de repente e, depois de um estranho movimento retrógrado,
ficou pairando, imóvel --- um escudo enorme no céu.\footnote{ Acho curioso que,
nem aqui, e nem depois, o Eremita faça referência ao movimento contínuo para o
norte e para o sul (visível, é claro) do Sol, de solstício a solstício.} 
Somente a borda circular do Sol tinha brilho --- somente ela e uma
linha fina de luz perto do equador.

Aos poucos, até mesmo essa linha de luz morreu; agora, tudo que restava de nosso grande e glorioso Sol era um
gigantesco disco extinto, contornado por um fino círculo de luz vermelho"-bronze.


\clearpage

\chapter{A Estrela Verde}

\textsc{O mundo} estava tomado por uma intensa penumbra, fria e insuportável. Lá fora, tudo era silêncio --- um silêncio
enorme! Do aposento escuro atrás de mim, vinha o ruído baixo e surdo\footnote{ Neste ponto, a atmosfera capaz de transmitir 
sons devia estar extremamente atenuada ou --- o que é mais provável ---
ser inexistente. À luz disso, não podemos supor que esses ruídos, ou quaisquer outros, pudessem ser detectáveis pela
audição que nós, em corpos materiais, compreendemos.} da ocasional queda de fragmentos das pedras
que apodreciam. Assim passou"-se o tempo, e a noite apoderou"-se do mundo, envolvendo"-o num manto de escuridão
impenetrável.

Não havia o céu noturno como o conhecemos. Mesmo as poucas estrelas dispersas desapareceram de vez. Era como se eu
estivesse num aposento fechado e sem luz, se levasse em conta o que conseguia enxergar. Só queimava, na penumbra
impalpável, lá do outro lado, aquele fio circular e vasto de fogo opaco. Para além dele, a vastidão da noite que me
circundava não tinha fim; exceto que, mais ao norte, aquele brilho suave, semelhante a uma névoa, ainda refulgia.

Os anos avançavam silenciosos. Quanto tempo havia se passado, eu jamais saberei dizer. Enquanto aguardava,
eternidades pareciam ir e vir furtivas; eu continuava a observar. Só podia enxergar às vezes o brilho do anel do
Sol; pois agora, ele começava a sumir e a reaparecer --- ficando um pouco mais nítido, e depois se extinguindo de novo.

Num átimo, em uma dessas eras, uma labareda repentina cortou a noite --- um clarão rápido que iluminou a Terra morta por
um breve momento, dando"-me um vislumbre de sua solidão plana. A luz parecia vir do Sol --- saindo de algum ponto próximo
ao seu centro, na diagonal. Durante um segundo fiquei observando, assustado. Então a chama diminuiu e a penumbra
voltou. Mas agora não estava mais tão escuro e o Sol apresentava um cinturão que era uma linha fina de luz branca e
vívida. Observei"-o com atenção. Teria um vulcão surgido na superfície do Sol? Mas logo rechacei o pensamento. Achei que
a luz era intensa e branca demais, e grande demais, para ser esse o motivo.

Outra hipótese surgia. Era a de que um dos planetas mais internos do Sistema Solar havia caído contra o Sol --- e se
tornado incandescente com o impacto. Era uma teoria que me agradava, já que parecia mais plausível e explicava de
maneira mais satisfatória o extraordinário tamanho e o clarão que iluminara o mundo de modo tão inesperado.

Muito interessado e animado, fiquei observando atentamente, através da escuridão, aquela linha de fogo branco que
cortava a noite. Uma coisa ela sem dúvida indicava: o Sol ainda rotacionava
numa velocidade enorme.\footnote{ Minha única suposição é que o tempo de
jornada anual da Terra havia de ter a atual duração relativa ao período da rotação solar.} Assim, soube
que os anos ainda voavam com rapidez incalculável, mas, no que tange à Terra, a vida, a luz e o tempo eram coisas que
pertenciam a um período perdido em eras havia muito esquecidas.

Depois daquela explosão de fogo, a luz era somente uma faixa de fogo vivo que circundava o Sol. Agora, contudo,
enquanto eu a observava, ela passava a adotar lentamente um matiz avermelhado e, depois, uma cor acobreada escura;
assim como o próprio Sol o fizera. Logo depois, ficou num tom ainda mais escuro; depois de mais algum tempo, passou
a oscilar, com períodos de incandescência e opacidade. Assim, após um bom tempo, ela desapareceu.

Bem antes disso, o anel ardente do Sol já havia sido engolido pela escuridão. Naquele futuro extremo, desse modo continuava
sombrio o mundo, mergulhado no mais profundo silêncio, em sua melancólica órbita ao redor da gigantesca massa do Sol
extinto.

É quase impossível descrever meus pensamentos durante esse estágio. No começo, eram caóticos, incoerentes. Mas depois,
enquanto eras iam e vinham, minha alma parecia absorver a própria essência da opressiva solidão e melancolia que
dominavam a Terra. 

Com essa sensação, veio uma maravilhosa clareza de raciocínio, e percebi, desesperado, que o mundo poderia vagar
durante toda a eternidade naquela noite infinita. Durante breve momento, essa ideia terrível tomou conta de mim,
fazendo com que eu sentisse uma desolação insuportável; era tão intensa que quase chorei feito uma criança. Com o
tempo, contudo, essa sensação foi diminuindo quase sem que eu percebesse, e comecei a sentir uma esperança sem nenhuma
lógica. Pacientemente, esperei.

De tempos em tempos, o ruído de partículas que caíam atrás de mim, no aposento, chegava debilmente aos meus ouvidos.
Certa vez ouvi um grande estrondo e me virei, instintivamente, para olhar, esquecendo"-me, durante um instante, da noite
impenetrável em que cada detalhe estava mergulhado. Logo depois, meu olhar vasculhou os céus, voltando"-se,
inconsciente, para o norte. Sim, o brilho nebuloso ainda estava lá. De fato, era quase possível imaginar que estava
um pouco mais nítido. Durante um bom tempo, mantive o olhar fixo nele, sentindo, em minha alma solitária, que aquela
suave névoa era, de certa maneira, um elo com o passado. Curioso como encontramos consolo em tais pormenores! Mas se eu
soubesse\ldots{} Falarei disso quando for a hora.

Durante um longo período, continuei observando, sem sentir sono algum, o que logo teria acontecido comigo em meus
antigos dias terrestres. Como seria bom adormecer, mesmo que fosse para passar o tempo, ficar afastado da sensação de
perplexidade e de meus pensamentos.

Diversas vezes, o som desagradável da queda de algum pedaço grande de pedra perturbava minhas reflexões; certa vez,
achei que estivesse ouvindo sussurros atrás de mim. Todavia, era inútil tentar enxergar. Não é possível
conceber uma escuridão como aquela. Era palpável, extremamente brutal para os sentidos, como se algo morto se apertasse
contra mim --- leve e gelada.

Com tudo isso, cresceu em minha mente uma ansiedade opressiva, que foi embora deixando em seu lugar uma incômoda
melancolia. Sentia que deveria lutar contra isso e, logo depois, tentando distrair meus pensamentos, virei para a
janela e olhei para o norte, em busca da brancura nebulosa que ainda acreditava ser o distante brilho enevoado do
Universo que havia abandonado. Assim que ergui o olhar, fui tomado pelo assombro, pois, agora, a luz nebulosa
revelava"-se uma única e enorme estrela de tom verde vívido.

Enquanto olhava, espantado, veio"-me de chofre o pensamento: a Terra deveria estar viajando rumo à estrela, e não se
afastando dela, como eu imaginara. Logo depois, pensei que talvez eu não pertencesse ao Universo que a Terra deixara,
e sim, talvez, fosse uma estrela fora dele, pertencente a alguma vasta constelação oculta nas profundezas do espaço.
Com um misto de assombro e curiosidade, continuei observando, tentando imaginar que novo elemento era esse que se
revelava.

Durante algum tempo, pensamentos e especulações vagas ocuparam minha mente e, durante todo o tempo, meu olhar fitava sem
cessar aquele ponto de luz, tão diferente da escuridão absoluta. A esperança tomou conta de mim, eliminando a opressão
do desespero que antes parecia me sufocar. Para onde quer que a Terra estivesse rumando, ela estava, pelo menos, indo
mais uma vez para o reino da luz. A luz! É preciso passar uma eternidade na noite insondável para compreender o horror
da ausência de luz.

Aos poucos, mas de modo constante, a estrela foi crescendo em meu campo de visão até que, depois de algum tempo, estava
brilhando como brilhara Júpiter nos velhos tempos da Terra. Ao aumentar de tamanho, sua cor ficou mais impressionante,
parecia uma enorme esmeralda que espalhava raios de fogo pelo mundo.

Os anos corriam silenciosos e a estrela esverdeada aumentou e tornou"-se uma grande chama no céu. Pouco depois
vi algo que me deixou impressionado. Era o fantasma do contorno de uma enorme meia"-lua no meio da noite, uma nova lua,
gigantesca, que parecia emergir da penumbra que a cercava. Totalmente perplexo, continuei olhando para ela. Parecia
estar bem próxima, em termos relativos, e quebrei a cabeça tentando entender como a Terra havia se aproximado dela sem
que eu a notasse antes.

A luz emitida pela estrela ficou mais intensa e, naquele instante, percebi que era possível ver outra vez a paisagem
ao redor, embora sem muita clareza. Durante alguns instantes fiquei observando, tentando discernir detalhes na
superfície terrestre, mas não havia luz suficiente. Depois de algum tempo, desisti e de novo passei a observar a
estrela. Mesmo naquele breve intervalo que desviara minha atenção, ela havia aumentado tanto de tamanho que
parecia ter, diante de meu olhar perplexo, um quarto da dimensão da lua cheia. A luz que dela emanava era absurdamente
poderosa; contudo, sua cor era tão estranha que o pouco do mundo que eu conseguia enxergar parecia irreal, mais como se
eu estivesse olhando para uma planície feita de sombras do que outra coisa.

Durante todo esse tempo, a nitidez do brilho da enorme meia"-lua aumentou, e agora começava a ter um tom
perceptivelmente verde. Aos poucos, a estrela aumentava em tamanho e brilho até ficar da metade de uma lua cheia; à
medida que crescia em tamanho e brilho, a gigantesca meia"-lua emitia mais e mais luz, mas de um tom de verde ainda mais
profundo. Sob o brilho conjunto de suas luzes, o descampado que se estendia diante de mim foi ficando aos poucos
visível. Logo eu conseguia enxergar toda a extensão do mundo visível, que agora parecia, sob aquela estranha luz,
terrível em sua solidão, tão plana, fria e tenebrosa.

Foi um pouco mais tarde que um fato chamou minha atenção: a grande estrela de fogo verde estava lentamente saindo do
norte e rumando para leste. No começo, eu mal conseguia acreditar no que via, mas logo não restava dúvida. Devagar
ela foi caindo e, à medida que caía, a vasta meia"-lua esverdeada começou a ficar cada vez mais embotada, até se tornar
mero arco de luz contra o céu de cor lívida. Depois a meia"-lua desapareceu, no mesmíssimo ponto de onde eu a vira
emergir.

A essa altura, a estrela estava a cerca de trinta graus do horizonte oculto. Em tamanho, rivalizava com a lua cheia,
mas, mesmo então, eu não conseguia discernir seu contorno. Esse fato me levou a pensar que ela ainda estava a uma
distância extraordinária; sendo assim, concluí que seu tamanho devia ser gigantesco, além da compreensão e imaginação
humanas.

De repente, enquanto eu observava, a extremidade inferior da estrela desapareceu --- foi atravessada por uma linha reta e
escura. Um minuto --- ou um século --- passou"-se, e a estrela ficou ainda mais baixa, até que metade dela havia
desaparecido. A distância, na enorme planície, vi uma sombra monstruosa a sua frente que avançava rapidamente. Agora,
somente um terço da estrela era visível. Então, de repente, a explicação daquele fenômeno extraordinário revelou"-se.
A estrela estava descendo por trás da enorme massa do Sol morto. Ou melhor, o Sol --- obedecendo à sua atração ---
erguia"-se rumo a ela,\footnote{ Uma leitura cuidadosa do manuscrito sugere que
o Sol ou está apresentando uma órbita bastante excêntrica, ou está se
aproximando da estrela esverdeada em uma órbita cada vez mais fraca. E, neste
ínterim, imagino que esteja sendo finalmente arrancado de seu curso oblíquo
pela força gravitacional da imensa estrela.} e a Terra seguia junto.
Enquanto esses pensamentos tomavam forma em minha mente, a estrela
desapareceu, completamente encoberta por uma enorme porção do Sol, e sobre a Terra caiu, mais uma vez, a melancólica
noite.

Com a escuridão veio uma sensação intolerável de solidão e medo. Pela primeira vez, pensei no Fosso e em seus
habitantes. Depois, lembrei"-me da Coisa ainda mais terrível que rondava as orlas do Mar Adormecido e furtava"-se às
sombras desta velha casa. ``Onde estariam?'', pensei; e esses terríveis pensamentos fizeram"-me estremecer. Durante algum
tempo, o medo tomou conta de mim, e implorei fervorosamente aos céus, de modo incoerente, por um raio de luz que
pudesse dissipar a fria escuridão que envolvia o mundo.

É impossível dizer quanto tempo esperei --- sem dúvida um bom tempo. Então, num átimo, vi um vulto de luz brilhar
adiante. Aos poucos, foi ficando mais nítido. De repente, um raio verde e nítido surgiu na escuridão. No mesmo
instante, vi uma linha fina de chama lívida nos confins da noite. No que parecia ser um breve segundo, ela se
transformara em grande massa de fogo; abaixo dela, o mundo jazia banhado pelo resplendor da luz cor de esmeralda.
Ela foi crescendo até que, finalmente, toda a extensão da estrela esverdeada era de novo visível. Mas, agora, mal
poderia ser chamada de estrela, pois havia aumentado muito de tamanho, ficando incomparavelmente maior do que o
Sol em tempos pregressos.

Então, enquanto olhava, percebi que podia ver a extremidade do Sol sem vida, brilhando como uma grande lua crescente.
Lentamente, sua superfície iluminada expandiu"-se até que metade de seu diâmetro ficasse visível e a estrela começou a
descer para a minha direita. O tempo passou e a Terra continuou se movendo, atravessando aos poucos a imensa face do
Sol morto.\footnote{ Podemos notar aqui que a Terra estava “aos poucos atravessando a imensa face do Sol morto”. Nenhuma explicação é
dada para tal acontecimento, e devemos concluir que o tempo estava passando mais devagar ou que a Terra na verdade
rotacionava lentamente, em relação aos padrões atuais. Uma análise cuidadosa do manuscrito, no entanto, leva"-me a
concluir que a velocidade do tempo já estava diminuindo havia um tempo considerável.}

Gradualmente, enquanto a Terra seguia sua jornada, a estrela caiu ainda mais para a direita, até que, por fim,
brilhava sobre a parte de trás da casa, fazendo com que uma enxurrada de raios de luz atravessasse os esqueletos das
paredes. Olhando para cima, vi que grande parte do teto havia desaparecido, o que me permitia observar que os andares de
cima estavam ainda mais dilapidados. O telhado, claro, havia sumido por completo e eu podia notar os raios
verdes da estrela penetrando de modo oblíquo. 


\clearpage

\chapter{O fim do Sistema Solar}

\textsc{Da parede} onde antes se encontravam as janelas através das quais eu observara aquela primeira e fatal aurora, pude ver
que o Sol estava imensamente maior do que estivera quando a estrela iluminou pela primeira vez o mundo. Estava tão
grande que sua extremidade inferior quase parecia tocar o horizonte distante. Achei que estivesse se aproximando mesmo
enquanto eu observava. O verde fulgor que iluminava a Terra congelada era cada vez mais nítido.

E foi assim durante um bom tempo. De repente, vi que o Sol estava mudando de forma, ficando menor, exatamente como a
Lua já fizera. Depois de algum tempo, somente um terço da parte iluminada estava voltada para a Terra. A
Estrela afastava"-se para a esquerda.

Aos poucos, à medida que o mundo se movia, a Estrela mais uma vez passou a irradiar sua luz sobre a frente da casa;
enquanto isso, o Sol parecia apenas um enorme arco de fogo esverdeado. Depois do que pareceu ser um breve instante, ele
havia desaparecido. A estrela ainda era totalmente visível. Então a Terra penetrou a negra sombra do Sol e tudo virou
noite --- uma noite profunda, sem estrelas, intolerável.

Tomado por pensamentos alvoroçados, fiquei observando a noite --- aguardando. Talvez anos tenham se passado e, então, na
casa escura, atrás de mim, o pesado silêncio do mundo foi rompido. Eu parecia ouvir os passos abafados de muitos pés e
um sussurro distante e inarticulado. Olhei em volta, na escuridão, e vi uma multidão de olhos. Enquanto os fitava, eles
aumentaram em número e pareciam vir na minha direção. Fiquei ali durante alguns instantes, sem conseguir me
mover. Então um terrível grito suíno\footnote{ Ver a primeira nota do
capítulo \textsc{xviii}.} ergueu"-se na noite e, com isso, pulei da janela para o mundo congelado lá
fora. Tenho a vaga lembrança de ter corrido durante algum tempo, depois disso, só esperei e esperei. Diversas vezes,
ouvi gritos, mas sempre como se viessem de longe. Exceto por esses sons, não tinha nenhuma ideia de onde estava a casa.
O tempo continuou avançando. Não estava ciente de muita coisa, exceto da sensação de frio, desespero e medo.

Ao fim do que parecia ser uma era, veio um brilho que anunciava a luz. Ela aumentou aos poucos. Então --- assomando em
toda sua glória fantasmagórica --- o primeiro raio da Estrela Verde atingiu a beirada do Sol negro e iluminou o mundo.
Iluminou uma enorme estrutura em ruínas, a uns duzentos metros de distância. Era a casa. Enquanto olhava para ela, vi
algo apavorante: sobre suas paredes, rastejava uma legião de seres inomináveis que quase cobriam a antiga construção,
desde as torres trôpegas até a base. Eu podia vê"-los claramente, eram as criaturas suínas.

O mundo avançou até ficar sob a luz da Estrela e, agora, ela parecia ocupar um quarto do céu. A glória de sua cor
lívida era tão formidável que parecia encher o céu de labaredas trêmulas. Então vi o Sol. Estava tão próximo que
metade de seu diâmetro ficava abaixo do horizonte; à medida que o mundo circulava acima de sua face, ele parecia
avultar até o céu, uma monstruosa cúpula de fogo cor de esmeralda. De vez em quando, eu olhava para a casa, mas os
seres suínos pareciam ignorar minha presença.

Os anos passaram lentamente. A Terra pairava quase sobre o centro do disco que era o Sol. A luz do Sol Verde --- como
ele agora deveria ser chamado --- brilhava através dos interstícios das paredes desmoronadas da velha casa, dando"-lhe a
aparência de estarem envoltas em chamas verdes. Os seres suínos ainda rastejavam pelas paredes.

De repente, ouvi um urro de vozes suínas e, do meio da casa sem teto, surgiu uma gigantesca coluna de chamas cor de
sangue. Vi as pequenas e retorcidas torres maiores e menores pegarem fogo, ainda assim, elas continuavam de pé. Os
raios do Sol Verde refletiam sobre a casa e mesclavam"-se com as chamas fantasmagóricas de tal modo que ela parecia uma
fornalha de intensas labaredas, com fogo verde e vermelho.

Fiquei observando, fascinado, até ser tomado por uma sensação avassaladora de perigo iminente. Olhei para cima e, no
mesmo instante, percebi que o Sol estava mais próximo, tão próximo, de fato, que parecia pairar sobre todo o mundo.
Então --- não sei como --- eu estranhamente estava nas alturas, flutuando como uma bolha em meio ao brilho terrível do Sol.

Bem abaixo de mim, vi a Terra e a casa em chamas, que se transformava numa montanha de fogo cada vez maior. Ao redor da
casa, o chão parecia estar em brasa; em alguns pontos, anéis de fumaça amarela subiam da terra. Era como se o mundo
estivesse se incendiando a partir daquele ponto castigado pelo fogo. Eu conseguia enxergar as criaturas suínas lá
longe. Não pareciam estar feridas. Então o chão pareceu ceder de repente, e a casa, com seu monte de criaturas
abomináveis, desapareceu nas profundezas da Terra, lançando no ar uma estranha nuvem de fumaça cor de sangue. Pensei no
Fosso infernal que ficava embaixo da casa.

Depois de algum tempo, olhei ao redor. A massa gigantesca do Sol pairava acima de mim. A distância entre ele e a Terra
diminuía bem rápido. De repente, a Terra impeliu"-se rapidamente para frente. Percorreu num instante o espaço entre
ela e o Sol. Não ouvi som algum, mas, na superfície do Sol, brotou uma labareda de fogo estonteante, cada vez maior.
Ela parecia quase alcançar o Sol Verde a distância --- penetrando a luz esmeralda numa verdadeira catarata de fogo
ofuscante. Chegou ao seu limite, diminuiu e sumiu; no Sol, restava uma enorme mancha branca em combustão --- o túmulo
da Terra. 

Agora, o Sol estava bem próximo de mim. Logo depois, descobri que eu estava pairando ainda mais alto, até que,
finalmente, estava acima do Sol, no vazio. O Sol Verde agora estava tão grande que sua largura parecia preencher todo o
céu à minha frente. Olhei para baixo e vi que o Sol passava bem abaixo de mim.

Talvez um ano tenha se passado --- ou um século --- e fiquei ali, suspenso, sozinho. O Sol pairava longe, à minha frente ---
uma negra massa circular contra o esplendor líquido do gigantesco Astro Verde. Em uma extremidade, via o brilho
fantasmagórico que marcava o local onde a Terra tinha caído. Com isso, soube que o Sol, havia muito extinto, ainda estava
em rotação, muito embora lentamente.

A distância, à direita, eu às vezes captava o leve brilho de uma luz esbranquiçada. Durante um bom tempo, não sabia ao
certo se era fruto da minha imaginação. Assim, por algum tempo fiquei observando, imaginando coisas, até que,
finalmente, tive certeza de que não era minha imaginação, que era algo real. A luz ficou mais intensa; logo depois,
surgiu de todo aquele verde uma pálida esfera, do branco mais suave. Ela foi se aproximando e vi que parecia estar
rodeada por um manto de nuvens levemente incandescentes\ldots{}

Olhei rápido para o Sol que diminuía. Ele agora era só uma mancha negra sobre a face do Sol Verde. Enquanto
olhava, vi"-o diminuir progressivamente, como se rumasse apressado para o astro superior, com impressionante velocidade.
Fiquei prestando atenção. O que será que aconteceria? Percebi que sentia uma ansiedade extraordinária ao dar"-me
conta de que ele atingiria o Sol Verde. Ele agora parecia menor que uma ervilha, e concentrei toda a minha atenção para
testemunhar o final derradeiro de nosso sistema --- o sistema que carregara o mundo através de tantas eras, com suas
infinitas tristezas e alegrias, e agora\ldots{}

De repente, algo cortou meu campo de visão, impedindo que eu visse qualquer vestígio do espetáculo que observava com
profundo interesse. O que aconteceu com o Sol extinto não pude ver, mas tenho motivos --- à luz do que vi depois --- para
acreditar que ele de fato mergulhou no estranho fogo do Sol Verde e assim pereceu.

Então, de repente, uma questão extraordinária surgiu em minha mente: seria aquela assombrosa esfera de fogo verde um
enorme Sol Central --- o grande Sol ao redor do qual o nosso Universo e incontáveis outros giravam? Senti"-me confuso.
Pensei no provável fim do Sol extinto e outra tola ideia surgiu: será que o Sol Verde era o túmulo das estrelas
extintas? A ideia não me parecia nem um pouco absurda, e sim algo possível e provável.


\clearpage

\chapter{As esferas celestiais}

\textsc{Durante certo} tempo, diversos pensamentos ocuparam minha mente a ponto de eu nada fazer além de fitar cegamente o
espaço à minha frente. Sentia"-me submerso em dúvidas, assombro e pesarosas lembranças.

Foi só mais tarde que saí desse estado de perplexidade. Olhei ao redor, confuso, e foi aí que vi algo tão
extraordinário que, durante algum tempo, mal conseguia acreditar que não estava mais absorto no turbilhão de visões de
meus pensamentos. Do verde que reinava absoluto, havia surgido uma multidão infinita de esferas que brilhavam
suavemente --- cada uma delas envolta numa camada de nuvens imaculadas que parecia lã. Alcançavam, tanto acima quanto
abaixo de mim, as profundezas distantes e não apenas ofuscavam o brilho do Sol Verde como emitiam, no lugar de sua
luz, uma luz suave que permeava tudo ao meu redor. Era algo completamente diferente de tudo que já vira, antes ou
depois.

Logo percebi que havia nessas esferas algum tipo de transparência, quase como se fossem feitas de cristal fosco, dentro
do qual ardia um brilho suave e discreto. Elas passavam por mim sem parar, flutuando numa velocidade amena, como se
tivessem toda a eternidade a seu dispor. Fiquei um bom tempo observando e elas pareciam não ter fim. Às vezes, achava
que via rostos em meio às nuvens mas eram rostos estranhamente indistintos, em parte reais, em parte formados pela
névoa através da qual surgiam.

Durante um longo período aguardei, passivo, com uma sensação cada vez maior de contentamento. Eu não tinha mais
aquela sensação de solidão inexprimível; em vez disso, havia eras não me sentia tão pouco solitário. Essa sensação de
contentamento aumentou de tal maneira que me daria por satisfeito se pudesse flutuar na companhia daqueles pequenos
globos celestiais por toda a eternidade.

Eras passaram"-se e eu via os rostos indistintos com frequência cada vez maior, e também mais nítidos. Se isso se devia
ao fato de minha alma estar em sintonia maior com o mundo ao redor, não sei dizer --- talvez sim. Mas, qualquer que fosse
o motivo, agora tenho certeza apenas do fato de que ficava cada vez mais consciente de um novo mistério ao meu redor, o
qual indicava que eu havia, de fato, penetrado a fronteira de alguma região jamais imaginada --- algum lugar ou forma de
existência sutil e intangível.

O enorme fluxo de esferas luminosas continuou a passar por mim, em número constante --- incontáveis milhões, e elas
continuavam a surgir, sem sinal de que cessariam ou diminuiriam.

Então, assim como fui em silêncio lançado no éter, senti que estava sendo impelido, de modo repentino e
irresistível, para frente, rumo a uma das esferas que passavam. Depois de poucos instantes, já estava ao lado dela.
Penetrei na esfera, sem sentir nenhuma resistência. Durante breve período, não conseguia enxergar nada e
aguardei, curioso.

Num átimo, percebi um som que interrompia o silêncio inconcebível. Era como o murmúrio de um grande mar calmo --- um mar
que respirava em seu sono. Gradualmente, a névoa que obscurecia meu campo de visão começou a se dissipar; e assim,
depois de algum tempo, meus olhos de novo deparavam com a silenciosa superfície do Mar Adormecido.

Fiquei fitando a paisagem durante algum tempo e mal podia acreditar no que via. Olhei ao redor. Havia um grande globo
de fogo pálido, flutuando, como eu o vira, a uma pequena distância do horizonte indistinto. À minha esquerda, bem
distante no mar, descobri, logo depois, uma fina linha, como se fosse uma suave névoa, que imaginei ser a praia onde
meu Amor e eu nos encontramos durante o maravilhoso período em que minha alma perambulou sem rumo, que tive o
privilégio de viver em minha estada terrena.

Outra memória, esta perturbadora, surgiu em minha mente --- da Coisa Disforme que havia me assombrado às margens do Mar
Adormecido. O guardião daquele lugar silencioso e sem ecos. Recordei"-me desse e de outros detalhes, e soube, sem a
menor dúvida, que estava olhando para o mesmo mar. Com essa certeza, fui tomado de uma sensação avassaladora de
surpresa, alegria e trêmula esperança, imaginando que talvez fosse de novo ver o meu Amor. Olhei com atenção a
paisagem, mas não havia sinal dela. Com isso, durante breve instante, minhas esperanças arrefeceram. Implorei aos
céus, fervoroso, e continuei a perscrutar a paisagem, ansioso\ldots{} Como o mar estava calmo!

Lá embaixo, bem distante de mim, podia ver os traços de fogo que antes haviam chamado minha atenção. Tentei imaginar
vagamente qual seria sua causa e lembrei que tencionara perguntar à minha Querida sobre eles, e também sobre muitas
outras coisas --- e que fora forçado a abandoná"-la antes de dizer nem metade do que desejava falar.

Meus pensamentos retornaram de repente. Percebi que algo me tocara. Virei"-me rápido. Deus, sois mesmo
misericordioso --- era Ela! Ela me fitou dentro dos olhos, com grande anelo, e eu a fitei com toda minha alma. Quis
abraçá"-la; todavia, a pureza gloriosa de seu rosto me mantinha afastado. Mas então ela esticou seus adoráveis braços
para fora da névoa que a envolvia. Seu sussurro chegou até mim, suave como uma nuvem que passa. “Meu amor!”, exclamou.
E foi só isso, mas eu a ouvira e, no instante seguinte, apertava"-a contra mim --- desejando que fosse assim para sempre.

E logo ela falava de muitas coisas e eu ouvia. Se de mim dependesse, teria ficado ouvindo durante todas as eras
vindouras. Às vezes, eu respondia em sussurro, e meus sussurros faziam surgir em seu rosto espectral, mais uma vez, um
matiz indescritivelmente delicado --- o rubor do amor. Depois, falei mais à vontade, e ela ouvia cada palavra e respondia
de maneira adorável, e então me senti no Paraíso.

Ela e eu; e nada além do vazio silencioso e gigantesco era nossa testemunha; e somente as águas tranquilas do Mar
Adormecido nos escutavam.

Bem antes, a multidão de esferas flutuantes, envoltas em nuvens, havia desaparecido no nada. Assim, contemplávamos
sozinhos a superfície das profundezas plácidas. Sozinhos, meu Deus, e assim ficaríamos por toda a eternidade, sem
contudo jamais nos sentirmos solitários! Ela era minha e, melhor ainda, eu era dela. Sim, eu mesmo, envelhecido por
eternidades incontáveis, e é refletindo sobre isso, e outras coisas, que espero existir durante os poucos anos que me
restam e que ainda nos separam.


\clearpage

\chapter{O Sol Negro}

\textsc{Durante quanto} tempo nossas almas descansaram nos braços da felicidade, não sei dizer, mas, de repente, fui despertado
de meu júbilo por uma diminuição na pálida luz suave que iluminava o Mar Adormecido. Virei"-me para observar a enorme
esfera branca, com um mau pressentimento. Um de seus lados curvava"-se para dentro, como se uma sombra convexa estivesse
passando por ele. Então lembrei. Foi assim que a escuridão surgiu antes da nossa despedida anterior. Voltei"-me para o
meu Amor, com um olhar intrigado. De repente, ciente de meu infortúnio, percebi quanto sua imagem agora estava
diáfana e enfraquecida, mesmo naquele breve período. Sua voz parecia vir de muito distante. O toque de suas mãos era
pouco mais que a leve pressão de uma brisa de verão e ficou menos perceptível.

Quase metade da enorme esfera estava encoberta. Uma sensação de desespero tomou conta de mim. Será que ela estava
prestes a me abandonar? Teria ela de ir embora, como já o fizera? Perguntei"-lhe, ansioso, com medo; e ela,
aninhando"-se mais contra mim, explicou, naquela voz estranha e distante, que era imprescindível que ela se fosse antes
que O Sol da Escuridão --- como ela o chamava --- encobrisse a luz. Com a confirmação de meus temores, fui tomado pelo
desespero; só me restava fitar, mudo, a tranquila planície do mar silencioso.

Como era rápida a escuridão que se espalhava sobre a superfície da Esfera Branca! Todavia, na realidade, tudo
aquilo deve ter se passado durante um longo tempo, além da compreensão humana.

Finalmente, somente uma meia"-lua de fogo pálido iluminava o agora obscuro Mar Adormecido. Durante todo esse tempo, ela
ficou me abraçando, mas seu toque agora era tão suave que eu mal o registrava. Ficamos ali aguardando, juntos, ela e
eu, mudos de tanta tristeza. Na luz que diminuía, seu rosto tornava"-se indistinto --- mesclando"-se à névoa escura que nos
rodeava.

Então, quando uma linha fina e recurva de luz suave era tudo o que iluminava o mar, ela me soltou --- empurrando"-me
ternamente para longe dela. Sua voz ecoou em meus ouvidos: “Não posso ficar, meu amor”. A frase terminou num soluço.

Ela pareceu flutuar para longe de mim e ficou invisível. Sua voz chegou até mim, das sombras, débil, aparentemente de
uma grande distância:

``Só mais um pouco\ldots{}'' E morreu à distância. Num segundo, o Mar Adormecido escureceu e virou noite. Bem distante, à
minha esquerda, achei que vi, durante breve momento, um leve brilho. Ele sumiu e, no mesmo instante, percebi que eu
não estava mais sobre o mar tranquilo, e sim mais uma vez suspenso no espaço infinito, com o Sol Verde agora eclipsado
por uma gigantesca esfera escura à minha frente.

Totalmente perplexo, fiquei olhando, quase sem prestar atenção no que via, o círculo de labaredas verdes que saltava
para fora da borda escura. Mesmo no caos dos meus pensamentos, fiquei estupidamente maravilhado por sua forma
extraordinária. Uma infinidade de dúvidas tomou conta de mim. A dor que sentia e os pensamentos sobre o futuro ocuparam
minha mente. Estaria eu condenado a ficar separado dela para sempre? Mesmo em minha pregressa vida terrestre, ela só
fora minha por pouco tempo e me abandonou, como eu acreditava, para sempre. Desde então, eu só a vira nessas
ocasiões, no Mar Adormecido.

Uma aguda sensação de raiva apoderou"-se de mim, com tristes questionamentos. Por que eu não poderia ter ido com
meu Amor? O que nos mantinha separados? Por que eu precisava esperar sozinho, enquanto ela repousava anos a fio no seio
tranquilo do Mar Adormecido? O Mar Adormecido! Meus pensamentos passaram, em vão, do amargor para novos e desesperados
questionamentos. Onde estava o Mar Adormecido? Onde estava? Eu mal havia me separado de meu Amor
e ele desapareceu por completo! Não devia estar muito longe dali! E a Esfera Branca que eu vira ser ocultada
pelas sombras do Sol da Escuridão! Meus olhos pousaram sobre o Sol Verde, em eclipse. O que havia ocasionado o eclipse?
Será que uma enorme estrela circundava"-o? Seria o Sol Central --- que era como eu agora o chamava --- uma estrela dupla? O
pensamento veio quase sem querer, mas por que não haveria de ser?

Meus pensamentos voltaram"-se para a Esfera Branca sobre o Mar Adormecido. Estranho que ela tivesse\ldots{} Parei de repente.
Tive uma ideia. A Esfera Branca e o Sol Verde! Seriam eles a mesma estrela? Minha imaginação voltou no tempo e
lembrei"-me da esfera luminosa à qual fora inexplicavelmente atraído. Curioso que tivesse me esquecido dela, mesmo que
por um momento. Onde estavam as outras? Voltei a pensar na esfera que havia penetrado. Raciocinei durante algum tempo, 
e as coisas ficaram mais claras. Imaginei que, ao entrar naquele globo impalpável, eu havia passado, no mesmo instante,
para alguma outra dimensão que era, até o momento, invisível; nela, o Sol Verde ainda era visível, mas como uma
monstruosa esfera de luz branca --- quase como se fosse o seu fantasma ali, não sua parte material.

Durante um bom tempo refleti sobre isso. Agora lembrava que, ao entrar na esfera, eu havia imediatamente deixado de ver
as outras. Continuei a meditar durante mais algum tempo sobre outros detalhes.

Depois, passei a pensar em outras coisas. Concentrei"-me no presente e comecei a prestar atenção ao meu redor. Pela
primeira vez, percebia que inúmeros raios, de um sutil tom violeta, penetravam a estranha semiescuridão em todas as
direções. Saíam da borda ardente do Sol Verde. Pareciam crescer em número enquanto eu observava, de tal modo que, em
pouco tempo, eram incontáveis. A noite estava cheia deles --- irradiavam"-se do Sol Verde em leque. Concluí que podia vê"-los
porque toda a glória do Sol estava encoberta pelo eclipse. Estendiam"-se até os confins do espaço e desapareciam.

Aos poucos, enquanto olhava, percebi que pequenos pontos de uma luz intensamente brilhante atravessavam os raios.
Muitos pareciam sair do Sol Verde e viajar até a distância. Outros vinham do nada e rumavam para o Sol, mas cada um
deles mantinha"-se estritamente preso ao raio sobre o qual viajava. A velocidade deles era inconcebível, e era só quando
se aproximavam do Sol Verde, ou quando dele se afastavam, que eu podia vê"-los como pontos de luz individuais. Distantes
do Sol, eles se tornavam finas linhas de fogo vívido dentro do violeta.

A descoberta desses raios, e das partículas que se moviam, deixou"-me muitíssimo interessado. Para onde iam, naquela
profusão incontável? Pensei nos mundos no espaço\ldots{} E aquelas fagulhas! Mensageiros! Talvez fosse uma ideia absurda;
mas não me parecia. Mensageiros! Mensageiros do Sol Central!

Uma ideia aos poucos foi se formando em minha mente. Seria o Sol Verde o lar de alguma inteligência superior? Era um
pensamento desconcertante. Imagens do Inominável surgiam em minha mente, incertas. Será que eu havia de fato
deparado com o lar do Eterno? Durante algum tempo, tentei debilmente repelir esse pensamento. Era uma ideia absurda
demais. No entanto\ldots{}

Pensamentos vagos e grandiosos surgiram em minha mente. Senti"-me, de repente, totalmente nu. E a sensação terrível de
proximidade da estrela me abalou.

E o Paraíso!\ldots{} Seria ele uma ilusão?

Meus pensamentos iam e vinham, erráticos. O Mar Adormecido\ldots{} e Ela! O Paraíso\ldots{} Voltei, sobressaltado, para o
presente. Em algum lugar, oriundo do vazio atrás de mim, surgia rápido um enorme e obscuro corpo celestial --- imenso,
silencioso. Era uma estrela extinta, que corria para a campa mortuária das estrelas. Ela passou entre mim e os Sóis
Centrais --- ocultando"-os de meu campo de visão, fazendo com que eu mergulhasse numa noite impenetrável.

Depois de um longo período, pude de novo ver os raios violeta. E um bom tempo depois --- eras, talvez --- um brilho
circular surgiu no céu, lá em cima, e eu vi a borda da estrela longínqua surgir escura contra ela. Assim soube que
ela se aproximava dos Sóis Centrais. Logo depois, vi o anel luminoso do Sol Verde ficar bem visível na noite. A estrela
havia passado para as sombras do sol extinto. Depois disso, esperei. Os estranhos anos passaram"-se devagar, e eu
observava sempre, muito atento.

Aquilo que esperava por fim aconteceu --- repentinamente, de modo horrível. Um clarão enorme de luz ofuscante. Uma
explosão contínua de labaredas brancas que atravessava o negro vazio. Durante um período indeterminado, essa explosão
subiu às alturas --- um gigante cogumelo de fogo. Parou de crescer. Então, com o tempo, começou a diminuir aos poucos.
Agora eu via que a explosão saía de um enorme ponto em brasa próximo ao centro do Sol Negro. Poderosas labaredas
ainda vinham vertiginosas daquele ponto. Todavia, apesar de seu tamanho, o túmulo da estrela não passava do reflexo
de Júpiter na superfície de um oceano, se comparado com a massa inconcebível do sol extinto.

Aqui, devo ressaltar, mais uma vez, que não há palavras capazes de transmitir a vastidão daqueles Sóis Centrais.


\clearpage

\chapter{A nebulosa obscura}


\textsc{Os anos} mesclaram"-se ao passado, séculos, eras. A luz da estrela incandescente transformou"-se num vermelho furioso.

Foi só mais tarde que vi a nebulosa obscura --- a princípio, uma nuvem muito distante, à minha direita. Ela foi crescendo
aos poucos até transformar"-se numa massa escura na noite. Impossível dizer quanto tempo fiquei observando, pois o
tempo, como nós o contamos, era algo do passado. Ela se aproximou, uma monstruosidade disforme de escuridão --- imensa.
Parecia deslizar pela noite, sonolenta --- uma verdadeira bruma infernal. Devagar, veio se aproximando e passou para o
vazio entre mim e os Sóis Centrais. Era como se minha visão tivesse sido bloqueada por uma cortina. Um estranho temor
tomou conta de mim, e um novo senso de curiosidade.

O crepúsculo verde que reinara durante tantos milhões de anos agora era substituído por uma penumbra impenetrável.
Imóvel, olhei ao redor. Um século passou"-se, e eu parecia detectar ocasionais clarões vermelhos e opacos que passavam
por mim de tempos em tempos.

Olhei ao redor com muita atenção e vi massas circulares de um tom vermelho terroso dentro da escuridão enevoada.
Pareciam brilhar através das trevas nebulosas. Depois de algum tempo, ficaram mais nítidas para minha visão que se
acostumava. Eu agora podia vê"-las com bastante nitidez --- esferas de cor vermelho"-ferrugem, semelhantes, em tamanho, às
esferas luminosas que eu vira tanto tempo antes.

Passavam constantemente por mim, flutuando. Aos poucos, passei a sentir uma peculiar ansiedade. Comecei a ter uma
sensação cada vez maior de repugnância e terror. E essa sensação dirigia"-se a essas esferas que passavam, e parecia
mais fruto de minha intuição do que ter uma causa ou motivo real.

Algumas das esferas eram mais brilhantes que outras e foi de uma dessas que um rosto surgiu de repente, olhando
fixamente para mim. Um rosto humano, pelo que indicava seu contorno, mas com feições tão torturadas pela angústia que
eu não conseguia desviar o olhar, horrorizado. Não imaginei que pudesse haver tamanha infelicidade como a que via
naquele rosto. Senti uma dor adicional ao perceber que os olhos, que fitavam o nada com intensa raiva, eram cegos.
Fiquei mais um tempo olhando para aquele rosto, então ele passou, sumindo na penumbra ao redor. Depois disso, vi
outros --- todos com a mesma expressão de tristeza e desespero, e cegos. 

Um bom tempo se passou e eu percebi que estava mais próximo das esferas do que antes. Com isso, fiquei nervoso; embora
tivesse com mais medo daqueles estranhos globos antes de ver seus tristes habitantes, pois a pena que sentia deles
abrandava o meu medo.

Mais tarde, não tive dúvidas de que estava sendo levado para mais perto das rubras esferas e de que, logo depois,
flutuava entre elas. Depois de pouco tempo, percebi que uma pairava cada vez mais perto sobre mim. Eu não conseguia
desviar"-me de seu caminho. Depois do que pareceu ser um minuto, ela já estava sobre mim, e eu era mergulhado numa névoa
de um vermelho"-escuro. A névoa se dissipou e agora eu olhava, confuso, para a imensa extensão da Planície do Silêncio.
Estava igual a quando eu a vira pela primeira vez. Eu era impelido em velocidade constante sobre sua superfície,
flutuando. Bem lá na frente, brilhava o vasto anel cor de sangue\footnote{ Sem dúvida, a massa de bordas
flamejantes do Sol Central extinto, vista de outra dimensão.} que iluminava o lugar. Tudo ao redor estava imerso
na extraordinária solidão do silêncio que tanto havia me impressionado da primeira vez em que percorri sua superfície
desoladora.

Logo depois vi, erguendo"-se até a penumbra cor de ferrugem, os picos distantes do portentoso anfiteatro de montanhas
onde, incontáveis eras antes, eu vira pela primeira vez terrores ocultos e onde, enorme e silenciosa, guardada por mil
deuses mudos, está a réplica desta casa cheia de mistérios --- esta mesma casa que vi ser consumida por chamas infernais,
nesta mesma Terra que se chocou contra o Sol e desapareceu para sempre. 

Embora pudesse ver os picos do anfiteatro montanhoso, foi só depois de um bom tempo que as partes inferiores ficaram
visíveis. Talvez isso se devesse à estranha névoa vermelha que parecia impregnar a superfície da Planície. Seja como
for, eu as vi por último.

Depois de mais algum tempo, estava tão próximo das montanhas que elas pareciam pairar acima de mim. Logo
depois, vi a grande fenda abrir"-se a minha frente, e flutuei para dentro dela, sem escolha.

Mais tarde, eu pairava sobre a vastidão da enorme arena e lá, aparentemente à distância de uns oito quilômetros,
estava a Casa, enorme, monstruosa, silenciosa --- bem no centro do imenso anfiteatro. Pelo que podia perceber, estava
igual; era como se eu a tivesse visto ontem. Ao redor, as montanhas sinistras e sombrias fitavam"-me com ar
de desaprovação, do alto de seu eminente silêncio.

Bem distante, à minha direita, lá no alto, entre os picos inacessíveis, avultava a enorme massa do grande Deus"-Fera.
Mais para cima, vi a forma da terrível deusa avultando em meio à penumbra vermelha, milhares de metros acima de mim.
À esquerda, pude distinguir o monstruoso Ser Sem Olhos, cinzento e inescrutável. Mais adiante, reclinada sobre seu
peitoril nas alturas, surgia a Forma Fantasmagórica --- uma mancha de cor sinistra contra as montanhas escuras.

Devagar, fui levado através da grande arena --- flutuando. No meu trajeto, pude discernir as formas indistintas de
muitos outros Horrores que espreitavam das enormes alturas.

Gradualmente fui me aproximando da Casa, e meus pensamentos voltaram num abismo de anos e anos. Lembrei"-me do terrível
Espectro que rondava aquele lugar. Algum tempo se passou e vi que estava sendo impelido na direção do enorme bloco que
era a silenciosa casa.

Neste ponto, estava ciente, de maneira um tanto indiferente, de ser tomado por uma sensação cada vez maior de torpor
que eliminava o medo, que eu deveria ter sentido ao me aproximar da fantástica ruína. Na verdade, eu a fitava
calmo --- assim como um homem observa um desastre através da fumaça de seu cachimbo.

Após certo período, eu havia me aproximado tanto da Casa que conseguia enxergar vários detalhes. Quanto mais olhava,
mais se confirmavam minhas impressões: ela tinha semelhança perfeita com esta estranha casa. Com exceção de suas
dimensões gigantescas, não encontrava nada que fosse diferente.

De repente, enquanto eu olhava, uma sensação enorme de assombro tomou conta de mim. Eu havia chegado em frente à porta
externa que leva ao estúdio. Ali, deitada sobre a soleira, estava uma enorme pedra do parapeito, idêntica --- talvez não
em tamanho e cor --- ao pedaço que eu havia deslocado no meu embate com as criaturas do Fosso.

Flutuei para mais perto e meu assombro aumentou ao perceber que a porta estava parcialmente arrancada das dobradiças,
exatamente do mesmo modo que a porta do meu estúdio fora forçada para dentro pelo ataque das criaturas suínas. Ver isso
desencadeou em mim uma série de pensamentos e comecei a suspeitar, debilmente, que o ataque a esta casa talvez tivesse
um significado bem maior do que imaginara até então. Lembrei"-me de como, havia tanto tempo, em minha pregressa vida
terrestre, eu suspeitara que, de algum modo inexplicável, esta casa que habito estava em harmonia --- para usar uma
expressão de fácil entendimento --- com essa outra estrutura espantosa, nos confins daquela Planície incomparável.

Agora, contudo, percebia que só concebera vagamente o que aquela suspeita significava. Comecei a compreender, com uma
clareza mais que humana, que o ataque que repelira estava, de alguma maneira extraordinária, ligado ao ataque àquela
bizarra construção.

Com peculiar falta de foco, meus pensamentos abruptamente abandonaram esse assunto e concentraram"-se, é estranho,
sobre o peculiar material de que era feita a Casa. Ela tinha --- como já mencionei --- um profundo tom esverdeado.
Agora que me encontrava tão próximo dela, podia perceber que ela às vezes flutuava, muito discretamente --- com um brilho
incandescente que ia e vinha, assim como os vapores do fósforo quando o esfregamos nas mãos, no escuro.

Logo depois, minha atenção foi desviada, pois me aproximava da grande entrada. Ali, pela primeira vez, senti medo,
pois, de repente, as enormes portas escancararam"-se e eu flutuei por entre elas, impotente. Lá dentro, só havia a mais
sólida escuridão. Num instante, eu havia atravessado a soleira, e as grandes portas fecharam"-se atrás de mim,
em silêncio, trancando"-me naquele lugar sem luz. 

Durante algum tempo, pareci pairar no ar, imóvel, suspenso na escuridão. Então percebi que estava me movendo
de novo; para onde, não saberia dizer. De repente, bem abaixo de mim, achei que ouvi o ruído balbuciante da risada de
uma criatura suína. O som morreu e o silêncio que se sucedeu parecia impregnado de horror.

Depois, uma porta abriu"-se em algum ponto à minha frente; uma névoa branca de luz penetrou o lugar e eu flutuei
devagar para dentro do aposento, que me parecia estranhamente familiar. De repente, ouvi gritos atordoantes,
ensurdecedores. Um panorama de visões turvas ardia diante de meus olhos. Meus sentidos ficaram abalados durante um momento
eterno. Então minha visão voltou ao normal. A sensação nebulosa de vertigem passou e agora eu podia ver com clareza.


\clearpage

\chapter{Pepper}

\textsc{Estava sentado} em minha cadeira, de volta a este velho estúdio. Meu olhar passeou pelo aposento. Durante cerca de um
minuto, ele pareceu ter uma aparência estranha, trêmula, irreal, sem substância. Logo isso sumiu e eu não via nada de
diferente. Olhei para a janela no fim do estúdio --- as cortinas estavam erguidas.

Fiquei de pé, cambaleante. Quando o fiz, um pequeno ruído, vindo da porta, chamou minha atenção. Olhei para lá. Durante
um breve instante, tive a impressão de que a porta estava sendo fechada com delicadeza. Fitei com atenção e vi que
certamente havia me enganado --- a porta estava bem fechada.

Fazendo grande esforço, repetidas vezes, consegui andar até a janela e olhar para fora. O Sol estava nascendo,
iluminando o mato emaranhado dos jardins. Durante um minuto, talvez, fiquei ali, olhando. Passei a mão na testa,
confuso.

Num instante, em meio ao caos dos meus sentidos, um pensamento repentino: virei"-me, rápido, e chamei Pepper. Não
tive resposta e saí andando pelo aposento, trôpego, sentindo medo de repente. Enquanto andava, tentava pronunciar seu
nome, mas meus lábios não obedeciam. Cheguei à mesa e abaixei"-me na direção onde ele estava, sentindo uma pontada no
coração. Ele estava deitado à sombra da mesa, e eu não havia conseguido enxergá"-lo de onde estava na janela. Agora, ali
inclinado, voltei a respirar brevemente. Não havia Pepper algum, em vez disso, eu me inclinava sobre um montinho
alongado de pó cinzento, semelhante a cinzas.

Devo ter permanecido naquela posição meio inclinada durante alguns minutos. Fiquei aturdido --- estupefato. Pepper havia
de fato passado para a dimensão das sombras.


\clearpage

\chapter{O som de passos no jardim}

\textsc{Pepper está} morto! Mesmo agora, mal consigo acreditar que isso seja verdade. Já se passaram várias semanas desde
que voltei daquela estranha e terrível jornada através do tempo e do espaço. Às vezes, quando durmo, sonho com ela, e
reproduzo, na minha imaginação, todos os terríveis acontecimentos. Quando acordo, fico um bom tempo refletindo sobre
ela. Aquele sol\ldots{} Seriam aqueles sóis de fato os grandes Sóis Centrais, ao redor dos quais gira todo o Universo e
espaços desconhecidos? Quem poderá dizer? E as esferas luminosas, flutuando para sempre à luz do Sol Verde! E o Mar
Adormecido sobre o qual flutuam! Como tudo isso é inacreditável. Não fosse por Pepper, eu estaria inclinado, mesmo
depois das muitas coisas extraordinárias que testemunhei, a pensar que tudo não passava de um enorme sonho. E também
havia aquela terrível nebulosa escura (com sua multidão de esferas vermelhas) movimentando"-se sempre à sombra do Sol
Negro, deslizando ao longo de sua impressionante órbita, envolta pela penumbra para toda a eternidade. E os rostos que
me fitavam! Meu Deus, será que eles, será que algo assim de fato existe? Ainda há um pequeno montinho de cinzas no
assoalho do estúdio. Não ouso tocar nele.

Às vezes, quando me sinto mais calmo, penso no que aconteceu com os planetas mais distantes em nosso Sistema Solar. Já
me ocorreu que pudessem ter se desprendido da atração solar e saído girando pelo espaço. Isso, é claro, é somente uma
hipótese. Há muitas outras coisas sobre as quais penso.

Agora que estou escrevendo, preciso registrar que tenho certeza de que há algo terrível prestes a acontecer. Na noite
passada, ocorreu algo que me deixou ainda mais aterrorizado que o Fosso. Agora descreverei o acontecimento e, se
algo mais acontecer, tentarei transcrevê"-lo assim que possível. Tenho a sensação de que há mais significado neste
último acontecimento do que em todos os outros. Mesmo agora, enquanto escrevo, sinto"-me abalado e nervoso. De algum
modo, desconfio que a morte está próxima. Não que eu tema a morte --- não a morte como a conhecemos. Todavia, há algo no
ar que me aterroriza --- um horror intangível, gélido. Eu me senti assim na noite passada. Foi assim:

Na noite passada, eu estava sentado aqui no meu estúdio, escrevendo. A porta que leva ao jardim estava entreaberta. Às
vezes, eu detectava de leve o barulho metálico da corrente do cachorro. É a corrente do cão que comprei depois da morte de
Pepper. Não permito que fique dentro de casa --- não depois de Pepper. Achei que seria melhor voltar a ter um cão. São criaturas
maravilhosas.

Estava absorto em meu trabalho e o tempo passou rápido. De repente, ouvi um ruído no jardim, no caminho que leva ao
estúdio --- um som estranho, furtivo, de passos abafados. Sentei ereto na cadeira, num movimento rápido, e olhei pela
porta aberta. Outra vez o ruído --- \textit{paf}, \textit{paf}, \textit{paf}. Parecia estar se aproximando. Sentindo
certo nervosismo, fiquei olhando para o jardim, mas a noite ocultava tudo. 

Então o cão uivou longamente e tive um sobressalto. Durante um minuto, talvez, fiquei olhando atento lá fora,
mas não ouvia nada. Depois de algum tempo, peguei a caneta"-tinteiro que havia deitado sobre a mesa e voltei a
trabalhar. A sensação de nervosismo sumiu; supus que o som que ouvira nada mais era do que o cão andando no canil até a
distância permitida pela corrente.

Uns quinze minutos devem ter se passado, então, de repente, o cão uivou outra vez, com um tom tão pesaroso, tão
clemente, que na hora fiquei de pé, deixando cair a caneta, manchando de tinta a página na qual escrevia. 

``Maldito cão!'', murmurei, ao perceber o que eu havia feito. Então, no mesmo instante em que pronunciava essas palavras,
ouvi de novo aquele estranho som --- \textit{paf}, \textit{paf}, \textit{paf}. Parecia terrivelmente próximo --- quase
perto da porta, imaginei. Agora sabia que não poderia ser o cão; a corrente não permitia que ele chegasse tão perto.

Ouvi de novo o rangido do cão e percebi nele, subconscientemente, uma ponta de medo.

Lá fora, sobre o parapeito da janela, vi Tip, o gato de minha irmã. Enquanto eu olhava, ele se pôs de pé e seu rabo
ficou bem grosso e eriçado. Ficou alguns instantes assim, parado, parecia olhar fixamente para algo na direção
da porta. Depois, rápido, começou a andar para trás sobre o parapeito; até que, chegando à parede, não podia fugir
mais. E lá ficou, imóvel, congelado numa pose do mais profundo terror.

Assustado e perplexo, peguei um bastão no canto e fui até a porta, em silêncio, levei uma das velas comigo. Estava a
poucos passos dela quando, de repente, uma sensação peculiar de medo tomou conta de mim --- um medo intenso, real; não
sabia sua origem ou motivo. Tão grande era a sensação de terror que não perdi tempo: somente recuei --- andei para
trás, temeroso, mantendo o olhar fixo na porta. O que mais queria era correr até ela, empurrá"-la e trancá"-la com os
ferrolhos, pois eu a havia consertado e reforçado e, portanto, ela estava mais forte do que nunca. Assim como Tip,
continuei a andar quase inconscientemente para trás, até atingir a parede. Com isso, tive um sobressalto e olhei em
volta, ansioso. Quando o fiz, meu olhar caiu, por um momento, sobre o suporte de armas, e dei um passo em sua direção,
mas logo parei, com a estranha sensação de que elas de nada adiantariam. Lá fora, nos jardins, o cão soltou um estranho
gemido.

De repente, o gato soltou um berro comprido e selvagem. Olhei, sobressaltado, para a direção dele --- algo luminoso e
fantasmagórico o circundava, crescendo no meu campo de visão. Distingui uma mão luminosa e transparente, com uma chama
esverdeada e tremeluzente sobre ela. O gato soltou um derradeiro e terrível miado estridente e eu o vi entrar em
combustão e virar fumaça. Minha respiração voltou com dificuldade e pressionei o corpo contra a parede. Na parte da janela
onde ele estivera, havia uma mancha verde, sobrenatural. A mancha ocultava a coisa de mim, embora o brilho do fogo
transparecesse de leve. O fedor de queimado invadiu o estúdio. 

\textit{Paf}, \textit{paf}, \textit{paf }--- algo desceu pelo caminho do jardim e um leve odor de mofo parecia entrar
pela porta e misturar"-se ao cheiro de queimado.

O cão estivera em silêncio durante algum tempo. Agora eu o ouvia dar um ganido agudo, como se sentisse
dor. Depois, ele ficou em silêncio, exceto por um gemido baixinho de medo.

Um minuto se passou; então o portão do lado oeste dos jardins bateu a distância. Depois disso, mais nada; nem mesmo o
choro do cão.

Devo ter ficado ali parado durante alguns minutos. Então senti um resquício de coragem e fui correndo até a porta,
fechei"-a e a tranquei. Depois disso, durante meia hora, fiquei ali sentado, inerme --- olhando para o nada, rígido.

Lentamente voltei à vida e consegui subir, trêmulo, as escadas, rumo ao meu quarto. E isso é tudo.


\clearpage

\chapter{\textsc{xxv}:  A coisa da arena}

\textsc{“Hoje de manhã cedo,} passei pelos jardins, tudo estava normal. Perto da porta, examinei o caminho em busca de
pegadas, mas de novo não havia nada que me dissesse com certeza que eu havia sonhado na noite anterior.

Foi só quando fui falar com o cão que descobri uma prova tangível de que algo acontecera. Quando fui ao canil, ele
continuou lá dentro, encolhido num canto, e precisei insistir para que ele saísse. Quando, finalmente, ele consentiu em
vir até mim, veio desanimado, cabisbaixo. Enquanto eu o acariciava, percebi uma mancha esverdeada no seu lombo
esquerdo. Ao examiná"-la, descobri que o pelo e a pele aparentemente tinham sido queimados, pois a carne aparecia,
ferida e queimada. O formato da mancha era estranho, semelhante à marca de uma grande garra ou mão.

Fiquei ali em pé, pensativo. Desviei o olhar para a janela do estúdio. Os raios do Sol que se levantava brilharam sobre
o pedaço esfumaçado no canto inferior da janela, fazendo com que sua cor estranhamente passasse do verde para o
vermelho. Ah! Sem dúvida aquilo era outra prova; de repente, lembrei"-me da Coisa horrível que vira na noite
anterior. Olhei de novo para o cão. Agora eu sabia a causa daquela terrível ferida em seu flanco --- e sabia, também, que
o que eu vira na noite anterior era real. E que grande tristeza senti. Pepper! Tip! E agora aquele pobre animal!\ldots{} Olhei
outra vez para o cão e vi que ele lambia a ferida.

‘Coitadinho!’, murmurei, abaixando"-me para lhe acariciar a cabeça. Com isso, ele se apoiou nas patas traseiras,
enfiando o focinho em minha mão e lambendo, triste.

Logo depois, saí dali, tinha outros assuntos a tratar.

Depois do jantar, fui vê"-lo de novo. Ele parecia quieto, sem vontade de deixar o canil. Minha irmã informou"-me que
ele não quis comer o dia inteiro. Ela me pareceu um tanto perplexa quando me disse isso, mas não parecia desconfiar de
nada que pudesse dar"-lhe medo.

O dia passou, sem grandes acontecimentos. Depois do chá da tarde, fui mais uma vez olhar o cão. Ele parecia triste e um
tanto inquieto, ainda assim, não queria sair do canil. Antes de trancar a casa à noite, mudei seu canil de lugar,
afastando"-o da parede, para que eu pudesse vê"-lo da pequena janela. Pensei em levá"-lo para passar a noite dentro de
casa, mas pensei bem e achei melhor deixá"-lo lá fora. Não é possível afirmar com certeza que a casa é, sob qualquer
aspecto, menos perigosa que os jardins. Pepper estava dentro de casa e, mesmo assim\ldots{}

Agora são duas da manhã. Desde as oito da noite, observo o canil a partir da pequena janela lateral de meu estúdio.
Todavia, nada aconteceu, e sinto"-me cansado demais para continuar vigiando. Vou para a cama\ldots{}

À noite, fiquei inquieto. Isso me é incomum, mas, mais perto da manhã, consegui dormir algumas horas.

Levantei"-me cedo e, depois do café da manhã, fui visitar o cão. Ele estava quieto, mas mal"-humorado, e se recusava a
sair do canil. Oxalá houvesse algum médico de cavalos nas redondezas, pediria que olhasse o pobre animal. Ele não comeu
o dia inteiro, mas estava sedento --- bebeu água com grande vontade. Fiquei aliviado ao ver isso.

Já é noite e estou em meu estúdio. Tenciono seguir o mesmo plano da noite anterior, observar o canil. A porta que leva
ao jardim está bem trancada. Sinto"-me grato por haver grades nas janelas\ldots{}

Já passa de meia"-noite. O cão, até o momento, ficou em silêncio. Da janela lateral à minha esquerda, posso
ver, um tanto indistintos, os contornos do canil. Pela primeira vez, o cão se move e ouço o barulho de sua corrente.
Olho para fora, rápido. Enquanto olho, o cão se movimenta, inquieto, e vejo uma pequena mancha luminosa brilhar no
interior do canil. Ela desaparece; o cão movimenta"-se de novo e, mais uma vez, o brilho surge. Fico intrigado. O cão
não faz nenhum ruído, mas posso ver a coisa luminosa claramente. É uma luminosidade bem nítida. Há algo familiar em seu
formato. Durante alguns instantes, tento adivinhar o que é então percebo que é semelhante ao formato dos dedos de
uma mão. Como uma mão! Então me lembro do contorno da terrível ferida no flanco do cachorro. Talvez o que eu esteja
vendo seja o ferimento. Ele fica luminoso à noite --- mas por quê? Os minutos se passam. Só consigo pensar nessa nova
descoberta\ldots{}

De repente, ouço um ruído nos jardins. É como se um choque atravessasse meu corpo. Está mais próximo. \textit{Paf},
\textit{paf}, \textit{paf}. Uma sensação de arrepio sobe"-me pela espinha e parece chegar até meu couro cabeludo. O cão
move"-se no canil e chora, com medo. Ele deve ter dado meia"-volta, pois agora não consigo mais ver o contorno do
ferimento que brilha.

Lá fora, os jardins ficam mais uma vez em silêncio e fico atento, com medo. Passa"-se um minuto, e mais outro, então
ouço de novo o som de passos. Bem próximos e parecem vir do caminho de cascalho. Curiosamente, é um som pausado e
prudente. Cessa bem perto da porta; fico em pé, imóvel. Na porta, ouço um som baixo --- é o trinco da porta que está
sendo erguido. Sinto um som agudo nos ouvidos, uma pressão na cabeça\ldots{}

O trinco sai, com um clique agudo. O ruído faz com que eu tenha um novo sobressalto; um choque terrível para os meus
nervos já tensos. Depois disso, fico em pé durante um bom tempo, ouvindo o silêncio cada vez maior. De repente, sinto
os joelhos começando a tremer e preciso sentar"-me depressa.

Depois de um intervalo indeterminado, começo aos poucos a me livrar da sensação de terror que se apoderara de mim.
Mesmo assim, continuo imóvel na cadeira. Parece que perdi o poder de me movimentar. Sinto"-me estranhamente cansado, com
vontade de cochilar. Meus olhos pesados fecham"-se e abrem"-se e, então, percebo que estou adormecendo e acordando,
repetidas vezes, sobressaltado.

Só alguns momentos depois é que percebo, sonolento, que uma das velas está quase acabando. Quando acordo de novo, ela
já se extinguiu, e o estúdio está muito escuro, iluminado pela única vela que resta. A semiescuridão não me perturba
tanto. Já não sinto mais a terrível sensação de medo, e meu único desejo é dormir\ldots{} só dormir.

Num sobressalto, embora não ouça nenhum ruído, estou desperto --- totalmente desperto. Estou bem consciente da
proximidade de algum mistério, de alguma Presença avassaladora. O próprio ar parece tomado pelo terror. Fico sentado,
encolhido, apenas ouvindo, atento. Mas não há nenhum ruído. A própria Natureza parece morta. Então, esse silêncio
opressivo é quebrado por uma assustadora e ruidosa ventania que varre a casa e morre em algum lugar remoto.

Deixo meu olhar vagar pelo aposento mal iluminado. Perto do grande relógio no canto, lá longe, há um vulto alto e
escuro. Fiquei alguns instantes olhando para ele, com medo. Mas percebo que não é nada e sinto um alívio momentâneo.

Nos instantes que se seguem, surge um pensamento em minha mente: por que não sair desta casa --- esta casa cheia de
mistérios e terror? Então, como resposta, surge, diante de mim, uma visão do fantástico Mar Adormecido --- o Mar
Adormecido onde ela e eu pudemos nos encontrar, depois de anos de separação e pesar e sei que preciso ficar aqui,
aconteça o que acontecer.

Pela janela lateral, noto a escuridão sinistra da noite. Meu olhar passeia pelo aposento, pousa num objeto mal
iluminado, depois em outro. De repente, viro e olho pela janela à minha direita; quando o faço, minha respiração
fica rápida e me inclino para frente, olhando assustado para algo do lado de fora, próximo à grade. Estou fitando um
rosto suíno, enorme, enevoado, sobre o qual flutua uma exuberante chama de tom esverdeado. É a Coisa que eu vira na arena.
De sua boca trêmula parece pingar uma baba contínua e fosforescente. Os olhos estão voltados diretamente para o
aposento, com uma expressão enigmática. Continuo sentado, paralisado.

A Coisa começa a se mover. Ela se vira, lenta, na minha direção. O rosto está se voltando para mim. Ela me vê. Dois
olhos imensos, ao mesmo tempo humanos e bestiais, estão olhando para mim através da penumbra. Estou gelado de medo;
mesmo agora, estou muito consciente e percebo, de maneira completamente irrelevante, que as estrelas distantes ficaram
ocultas pela cara gigantesca.

Um novo horror toma conta de mim. Estou levantando da cadeira sem a menor intenção de fazê"-lo. Estou de pé, e algo me
força a ir à porta que leva até o jardim. Quero parar, mas não consigo. Alguma força constante se opõe ao meu desejo, e
continuo lentamente a andar, relutante, tentando resistir. Olho desesperado para todos os lados, desamparado, e meu
olhar chega à janela. O enorme rosto de feições suínas desapareceu e de novo ouço aquele som furtivo, \textit{paf},
\textit{paf}, \textit{paf}. O som cessa bem em frente à porta --- a porta à qual estou sendo levado\ldots{}

Logo depois, há um breve e intenso silêncio; então ouço um ruído. É o som do trinco sendo levantado. Com isso, sou
tomado pelo desespero. Não quero dar mais nenhum passo. Faço um esforço enorme para voltar, mas é como se estivesse de
costas contra uma parede invisível. Solto um gemido alto, na agonia do meu medo, e o som de minha voz é assustador.
Outra vez o ruído e eu estremeço, suando frio. Tento --- não, luto, ferozmente, para parar, voltar, mas não adianta\ldots{}

Estou perto da porta e, de modo mecânico, vejo minha mão esticar"-se e destravar o ferrolho de cima. Ela o faz
totalmente contra a minha vontade. Mesmo enquanto estico a mão para o ferrolho, a porta é sacudida com violência, e
sinto um cheiro repulsivo de mofo que parece penetrar pelas reentrâncias da porta. Puxo o ferrolho devagar, lutando,
débil, o tempo todo. Ele sai do encaixe com um clique e começo a tremer, agoniado. Há mais dois ferrolhos; um na
parte inferior da porta e o outro, enorme, no meio.

Durante talvez um minuto, fico ali parado, os braços caídos. A influência para que eu mexesse nos ferrolhos parece ter
cessado. No mesmo instante, ouço um ruído metálico, alto e repentino, aos meus pés. Olho rápido para baixo e percebo,
com terror indescritível, que meu pé está empurrando o ferrolho inferior. Uma sensação terrível de impotência toma
conta de mim\ldots{} O ferrolho sai do encaixe com um leve ruído metálico e eu fico tonto, e me agarro ao grande ferrolho
central em busca de apoio. Passa"-se um minuto, uma eternidade, depois mais outro. Meu Deus, ajude"-me! Estou sendo
forçado a puxar o último ferrolho. Não quero! Melhor morrer do que abrir a porta para o Terror que se encontra do outro
lado. Será que não há escapatória?\ldots{} Deus do céu, puxei o ferrolho do trinco! Meus lábios deixam escapar um grito
rouco de terror, o ferrolho já está três quartos para fora, e ainda assim minha mão inconsciente opera para minha
ruína. Só há um pedaço de metal entre minha alma e Aquela Coisa. Duas vezes, grito na agonia suprema do meu medo;
então, fazendo um esforço incrível, retraio as mãos. Meus olhos parecem cegos. Uma grande escuridão cai sobre mim. A
Natureza veio em meu socorro. Sinto os joelhos vacilar. Ouço o barulho surdo e alto de algo atingindo o chão, e sinto
que estou caindo, caindo\ldots{}

Devo ter ficado ali no chão pelo menos umas duas horas. Quando recobro os sentidos, percebo que a outra vela apagou"-se
e o aposento está quase que totalmente escuro. Não consigo ficar de pé, pois sinto frio e uma câimbra terrível. Mas
minha mente está alerta e não sinto mais a força daquela influência diabólica.

Cauteloso, fico de joelhos e tateio em busca do ferrolho central. Eu o encontro e o coloco totalmente dentro do
trinco, depois, faço o mesmo com o trinco da parte inferior da porta. A essa altura, já estou apto a ficar de pé, então
consigo fechar o ferrolho da parte de cima. Depois, ponho"-me de novo de joelhos e saio engatinhando por entre os
móveis, rumo às escadas. Ao fazer isso, fico a salvo de ser observado através da janela.

Chego à porta oposta e, enquanto saio do estúdio, lanço um breve olhar nervoso para trás, para a janela. Lá fora, na
noite, acredito ver o vulto de algo impalpável, mas talvez seja só minha imaginação. Então chego ao corredor e à
escada.

Chegando ao quarto, subo na cama, ainda vestido, e puxo as cobertas. Ali, depois de alguns momentos, começo a me sentir
um pouco mais confiante. É impossível dormir, mas sinto"-me grato pelo calor das cobertas. Logo depois, tento analisar
os acontecimentos da noite anterior, mas, por mais que não consiga dormir, descubro que é inútil tentar raciocinar de
maneira coerente. Minha mente parece vazia.

Perto da manhã, começo a me revirar na cama, inquieto. Não consigo descansar e, após algum tempo, levanto e fico
andando pelo quarto. A alvorada de inverno começa a entrar sorrateira pelas janelas, evidenciando o desconforto daquele
quarto sem móveis. Estranho que, durante todos esses anos, jamais me ocorrera quanto o lugar era lúgubre. E, assim, o
tempo passa. 

Ouço um ruído em algum lugar lá embaixo. Vou até a porta do quarto e presto atenção. É Mary, mexendo na enorme e
velha cozinha, preparando o café da manhã. Não sinto muito interesse. Não tenho fome. Todavia, meus pensamentos
concentram"-se nela. Os estranhos acontecimentos naquela casa parecem perturbá"-la tão pouco! Com exceção do incidente
das criaturas do Fosso, ela nunca pareceu ter consciência de que coisas estranhas estivessem acontecendo. Ela é idosa,
assim como eu; mesmo assim, como somos diferentes. Será porque não temos nada em comum, ou simplesmente porque, sendo
velhos, preferimos a solidão à convivência? Esses e outros pensamentos atravessam minha mente e ajudam a me distrair,
por alguns instantes, dos pensamentos opressivos da noite.

Depois de algum tempo, vou até a janela e, ao abri"-la, olho para fora. O Sol agora já está acima do horizonte e o ar,
embora frio, é puro e agradável. Aos poucos, minha mente fica menos confusa, e provo, pelo menos por enquanto, uma
sensação de segurança. Um pouco mais feliz, desço as escadas e vou até o jardim, para ver como está o cão.

Quando me aproximo do canil, sou recebido pelo mesmo cheiro de mofo que me assaltou à porta, na noite anterior.
Desvencilhando"-me da momentânea sensação de medo, chamo o cão, mas ele não atende e, depois de chamá"-lo mais uma
vez, jogo uma pequena pedra dentro do canil. Com isso, ele se remexe lá dentro, sem vontade, e de novo chamo alto o
seu nome, mas não me aproximo. Logo depois, minha irmã sai de casa e se junta a mim para tentar persuadi"-lo a sair do
canil.

Passado certo tempo, o pobre animal fica de pé e sai com passos morosos, sacudindo"-se de modo estranho. Sob a luz do
dia, ele fica parado, incerto, balançando"-se um pouco, piscando tolamente. Olho para ele e percebo que a terrível ferida
está maior, bem maior, e parece ter uma aparência esbranquiçada como a de um fungo. Minha irmã move"-se para
acariciá"-lo, mas eu a impeço, explicando que acho melhor não chegar perto dele durante alguns dias, já que não é
possível saber qual é o problema, melhor ter cautela.

Um minuto depois, ela se vai, volta com uma bacia cheia de restos de comida. Coloca a bacia no chão, perto do cachorro,
e eu a empurro, para que fique ao alcance dele, com o auxílio de um galho que arranquei de um dos arbustos. Por mais
que a carne seja tentadora, ele não se interessa, em vez disso, entra de novo no canil. Ainda há água na vasilha,
então, depois de conversarmos um pouco, voltamos para dentro de casa. Percebo que minha irmã está bastante intrigada
com a situação do animal; todavia, seria loucura mencionar algo próximo da verdade.

O dia vai embora, sem grandes acontecimentos; surge a noite. Estou determinado a repetir o experimento da noite
anterior. Não saberia dizer se é uma decisão sábia; todavia, estou decidido. Tomei algumas precauções: bati
pregos grandes atrás de cada um dos três ferrolhos da porta do estúdio que leva aos jardins. Talvez pelo menos isso
impeça a reincidência do perigo por que passei na noite anterior.

Das dez até aproximadamente duas e meia da manhã, fico alerta, mas nada acontece; finalmente, deixo"-me cair na cama,
onde logo adormeço.


\clearpage

\chapter{\textsc{xxvi}:  O ponto luminoso}

\textsc{“Acordo de repente.} Ainda está escuro. Viro"-me na cama, uma ou duas vezes, tentando voltar a dormir, mas não consigo.
Minha cabeça dói um pouco; sinto alternadamente frio e calor. Depois de algum tempo, desisto e estico a mão em busca
dos fósforos. Pretendo acender a vela e ler um pouco; talvez consiga dormir depois. Durante alguns momentos, fico
tateando; minha mão sente a caixa de fósforos, mas, quando a abro, levo um susto ao ver um ponto de fogo fosforescente
brilhando na escuridão. Estico a outra mão para tocá"-lo. O ponto está no meu pulso. Com certo medo, acendo rapidamente
um fósforo para olhar, mas não consigo ver nada além de um pequeno arranhão.

‘Estou imaginando coisas!’, murmuro, com um leve suspiro de alívio. Então sinto o fósforo queimar meu dedo e o solto
imediatamente. Estou tateando no escuro, tentando acender outro, e a coisa brilha de novo. Agora sei que não é minha
imaginação. Desta vez, acendo a vela e examino com cuidado o ponto em meu pulso. Há uma leve descoloração esverdeada
ao redor do arranhão. Fico intrigado e assustado. Então um pensamento me ocorre. Lembro a manhã depois que a Coisa
apareceu. Lembro que o cão lambeu minha mão. E foi esta a mão com o arranhão; embora eu não tivesse percebido a
queimadura até agora. Um medo horrível apodera"-se de mim. O pensamento insinua"-se em minha mente: o ferimento do cão
brilha à noite. Atordoado, fico sentado na cama, tentando raciocinar, mas não consigo. Meu cérebro parece entorpecido
pelo mais absoluto horror. 

O tempo passa sem que eu perceba. Tento me desvencilhar daquilo e me convencer de que estou enganado, mas não adianta.
Lá no fundo, não tenho dúvida.

Hora após hora, permaneço sentado na escuridão, em silêncio, estremecendo, sentindo"-me perdido\ldots{}

O dia vem e vai, e já é noite de novo.

Hoje cedo, de manhã, atirei no cão e o enterrei longe, entre os arbustos. Minha irmã ficou alarmada e assustada, mas
estou desesperado. Além disso, foi melhor assim. A asquerosa ferida já cobria todo o lado esquerdo do animal. Quanto a
mim, o ponto em meu pulso está perceptivelmente maior. Diversas vezes peguei"-me murmurando orações --- pequenas orações
que aprendi na infância. Deus misericordioso, ajude"-me! Sinto que vou ficar louco.

Seis dias se passaram e não comi nada. É noite. Estou sentado em minha cadeira. Ah, meu Deus! Teria alguém sentido
horror semelhante ao que provo? Estou tomado pelo terror. O tempo todo sinto a queimação desse fungo tenebroso. Já
cobriu todo o meu braço direito e a lateral de meu corpo, e ameaça subir"-me pelo pescoço. Amanhã cobrirá o meu rosto.
Tornar"-me"-ei uma grande e terrível massa de podridão ambulante. Não há escapatória. Mas um pensamento me ocorreu quando
olhei para o suporte das armas, do outro lado do aposento. Olhei mais uma vez, tomado por um estranho sentimento. O
pensamento se apodera de mim. Deus meu, sabeis, deveis saber que a morte é melhor, sim!, mil vezes melhor do que Isto.
Do que Isto! Jesus, perdoe"-me, mas não posso mais viver, não posso, não posso! Não ouso! Não me resta qualquer
esperança --- nada mais me resta. Ao menos poderei me livrar deste derradeiro horror. “Acho que cochilei. Sinto"-me fraco
e, ah!, tão desesperado, tão desesperado e cansado\ldots{} Cansado. O ruído dos papéis é uma tortura para meu cérebro. Minha
audição parece afiada de modo sobrenatural. Vou continuar aqui sentado, pensando\ldots{}

Silêncio! Ouço algo lá embaixo ---
lá no porão. O som de madeira rangendo. Meu Deus, é o grande alçapão de carvalho que se abre. O que poderia ser isso? O
ruído da minha caneta é ensurdecedor\ldots{} Preciso continuar ouvindo\ldots{} Ouço passos na escada; estranhos passos abafados
que se aproximam cada vez mais\ldots{} Senhor Jesus, tenha piedade de mim, um velho. Algo está mexendo na maçaneta. Meu
Deus, ajude"-me! Jesus\ldots{} A porta está se abrindo lentamente\ldots{} Alg\ldots{}”

\smallskip

E isso é tudo.

\smallskip

\textsc{nota}: A partir da palavra interrompida, é possível perceber, no manuscrito, uma fina linha de tinta que sugere o rastro
da caneta no papel, talvez causado por medo e fraqueza.


\clearpage

\chapter{\textsc{xxvii}: Conclusão}

\textsc{Fecho o manuscrito} e olho de soslaio para Tonnison; ele estava sentado, mirando o nada na escuridão. Esperei um
minuto e então falei.

--- E então? --- eu quis saber.

Ele se virou, devagar, e olhou para mim. Seus pensamentos pareciam estar longe. --- Ele era louco? --- perguntei, indicando o
manuscrito com um leve movimento de cabeça.

Tonnison olhava para mim, mas não me via, e ficou assim durante alguns instantes; depois, parece que voltou à realidade e
entendeu, de repente, a minha pergunta.

--- Não! --- disse ele.

Abri os lábios para emitir opinião contrária, pois, devido à sanidade que a tudo eu imprimia, eu não me permitia
entender aquela história literalmente, mas fechei a boca, sem dizer nada. De algum modo, a certeza na voz de Tonnison
me fez duvidar. Senti"-me, na hora, inseguro em minhas opiniões, muito embora ainda não estivesse convencido.

Depois de alguns instantes de silêncio, Tonnison pôs"-se de pé, com o corpo meio enrijecido, e começou a tirar a roupa.
Não parecia ter vontade de conversar, então eu não disse nada; em vez disso, segui seu exemplo. Sentia"-me cansado, no
entanto, ainda absorto na história que havia lido.

De algum modo, enquanto me enrolava nas cobertas, surgiu"-me na mente a memória do antigo jardim, tal e qual o vimos.
Lembrei"-me do estranho medo que sentimos naquele lugar e pensei, cada vez mais convicto, que Tonnison estava
certo.

Era bem tarde quando acordamos --- perto de meio"-dia; ficamos lendo o manuscrito durante grande parte da noite.

Tonnison estava irritadiço e eu não me sentia muito bem. O dia estava um tanto lúgubre, o ar um pouco frio. Nenhum de
nós sugeriu ir pescar. Fizemos a refeição principal e, depois disso, ficamos só sentados, fumando em silêncio.

Logo depois, Tonnison pediu o manuscrito, eu o entreguei a ele, que passou a maior parte da tarde lendo"-o sozinho.

Foi enquanto ele lia que uma ideia me veio:

--- O que você diz sobre darmos mais uma olhada no\ldots{}? --- perguntei, indicando com a cabeça o riacho lá embaixo.

Tonnison olhou para mim. 

--- Não! --- respondeu ele, de modo ríspido; de algum modo, fiquei mais
aliviado do que irritado com sua resposta.

Depois disso, deixei"-o em paz.

Um pouco antes da hora do chá, ele olhou para mim, com ar curioso.

--- Desculpe, amigo, se fui ríspido com você agora há pouco --- (agora há pouco! ele não falava nada havia três
horas). --- Mas não ouso voltar lá --- revelou ele, indicando a direção do lugar com a cabeça --- por nada nesse mundo. Que
horror! --- e fechou o manuscrito com a história que relatava o terror, a esperança e o desespero daquele homem.

Na manhã seguinte, levantamos cedo e fomos nadar, como era costumeiro; não sentíamos mais tanto a melancolia do dia
anterior; assim, quando terminamos o café da manhã, pegamos nossas varas de pescar e passamos o dia em nossa
atividade favorita.

Depois daquele dia, aproveitamos nossos dias de folga ao máximo; no entanto, nós dois esperávamos ansiosos o momento
em que o cocheiro viria, pois queríamos muito perguntar a ele e, por meio dele, às pessoas do pequeno vilarejo, se
alguém poderia nos dar alguma informação sobre aquele estranho jardim que vicejava solitário no coração de uma parte
quase desconhecida daquela área.

Finalmente, chegou o dia em que aguardávamos a visita do cocheiro para nos apanhar. Ele veio cedo, enquanto ainda
estávamos deitados; quando percebemos, ele já estava na abertura da barraca, querendo saber se havíamos nos
divertido. Respondemos que sim, então, os dois juntos, quase em uníssono, fizemos a pergunta que mais ocupava nossos
pensamentos: saberia ele algo a respeito de um velho jardim, e de um enorme fosso, e de um lago, situados a alguns
quilômetros dali, descendo o rio? E saberia também algo sobre uma enorme casa que ficava ali perto?

Não, ele não sabia, mas, vejam só, ele ouvira boatos, havia muito tempo, sobre uma grande e antiga casa, solitária no meio
do mato, mas, pelo que se lembrava, era um lugar lendário, ou, caso isso não fosse verdade, ele tinha certeza (e aqui
usou uma palavra típica do linguajar irlandês) que havia algo de “estranho” ali; de todo modo, não ouvia falar do lugar
fazia muito tempo, pelo menos não desde que era jovem. Não, ele não se lembrava de nada específico sobre o lugar; na
verdade, disse que não se lembrava “de nada, de absolutamente nada” até ser interrogado.

--- Escute --- pediu Tonnison, percebendo que aquilo era tudo o que ele podia nos dizer. --- Dê uma volta pelo vilarejo,
enquanto nos aprontamos, e descubra algo, por favor.

Com um gesto de saudação genérico, o homem saiu em sua missão; enquanto isso, apressamo"-nos para nos vestir; depois,
começamos a preparar o café da manhã.

Estávamos prestes a sentar para comer quando ele voltou.

--- Esses preguiçosos ainda estão dormindo, senhor --- contou ele, com seu forte sotaque irlandês, repetindo a saudação,
lançando um olhar de desejo para a boa comida espalhada sobre o nosso baú de mantimentos, que utilizávamos como mesa.

--- Ora, sente"-se --- respondeu meu amigo. --- Coma conosco. --- O homem obedeceu, sem demora.

Depois do café da manhã, Tonnison enviou de novo o homem para a aldeia, com a mesma missão e, enquanto isso, ficamos
esperando sentados, fumando. Ele se foi durante aproximadamente quarenta e cinco minutos e, quando voltou, era evidente
que havia descoberto algo. Parece que conversou com um homem muito velho do vilarejo, que talvez soubesse mais
do que qualquer outra pessoa viva --- embora fosse muito pouco --- sobre aquela estranha casa.

A essência do que ele sabia era isto: na juventude do “velho decrépito” --- e sabe Deus há quanto tempo foi isso --- havia
uma enorme casa no centro dos jardins, onde agora só existiam fragmentos de uma ruína. Essa casa ficou vazia durante um
bom tempo; anos antes de o velho nascer. Era um lugar que as pessoas do vilarejo evitavam, como seus pais antes delas.
Diziam"-se muitas coisas a respeito dela, todas ruins. Ninguém chegava perto dela, fosse de dia ou de noite. No
vilarejo, a casa era sinônimo de tudo que havia de terrível e diabólico no mundo.

Então, um dia, um desconhecido passou montado a cavalo pelo vilarejo, desceu pelo rio, indo rumo à Casa, como
assim a denominavam os habitantes. Algumas horas depois, ele retornava a cavalo pelo mesmo caminho que fizera, rumo ao
vilarejo de Ardrahan. Durante uns três meses, nada se ouviu sobre ele. No fim desse período, ele reapareceu;
agora, acompanhado de uma mulher mais velha e vários jumentos, carregados de coisas. Passaram pelo vilarejo sem parar e
desceram direto para o rio, na direção da Casa.

Desde aquela época, ninguém, com exceção do homem contratado para levar víveres de Ardrahan até o vilarejo, vira
qualquer um dos dois; quanto ao mercador, ninguém tentara conversar com ele; evidentemente, alguém lhe pagara para que
permanecesse em silêncio.

Os anos passaram"-se sem grandes acontecimentos no pequeno vilarejo; o homem fazendo suas regulares viagens mensais.

Um dia, ele apareceu como sempre. Passou pelo vilarejo sem fazer mais do cumprimentar os aldeões com um mal"-humorado
gesto de cabeça e seguiu rumo à Casa. Em geral, estava de volta à noitinha. Desta vez, contudo, reapareceu no
vilarejo algumas horas depois, em grande estado de excitação, com a fantástica informação de que a Casa havia
desaparecido e que em seu lugar havia agora um fosso monstruoso.

Essa novidade, aparentemente, excitou tanto a curiosidade dos aldeões que eles superaram o medo e foram em massa até o
lugar. Lá, descobriram tudo exatamente como descrito pelo caixeiro.

E foi só isso que ele nos contou. Quanto ao autor do manuscrito, quem era e de onde vinha, jamais saberemos.

Sua identidade, como parece que era seu desejo, está enterrada para sempre.

Naquele mesmo dia, deixamos o solitário vilarejo de Kraighten. Não voltamos lá desde então.

Às vezes, em sonhos, vejo aquele fosso enorme, cercado por todos os lados, como o é de fato, pela floresta de arbustos e
árvores. O ruído da água sobe das profundezas e se mistura --- em meus sonhos --- com outros ruídos menores e, sobre
tudo, paira a eterna nuvem de gotículas.


\cleardoublepage

\chapter[Sofrimento]{Sofrimento\footnoteInSection{ Encontrei estas estrofes escritas a
lápis em um pedaço de papel colado atrás da folha de guarda do manuscrito.
Parecem ter sido escritas em data anterior à do manuscrito.}}

\begin{verse}
A saudade abarca meu peito,\\*
Jamais imaginei que este mundo\\*
Tão aprisionado na mão de Deus\\*
Lançar"-me"-ia no silencioso leito\\*
Da dor de um Pesar profundo\\*
Que atormenta os sonhos meus!

Cada suspiro são mil ais,\\*
O coração flameja em agonia,\\*
E resta um único pensamento:\\*
Que não poderei jamais\\*
(A dor da memória não sacia)\\*
Tocar"-te, tu que és mero vento!

À noite saio em tua busca,\\*
Tolamente clamando por ti;\\*
Na noite, o trono da noite eterna\\*
É uma imensa igreja que ofusca\\*
Cujos sinos batem por mim,\\*
Solitário, preso nesta caverna!

Vou até a orla, faminto,\\*
Talvez haja consolo adiante\\*
No âmago eterno do Mar;\\*
Mas das profundezas pressinto\\*
Vozes misteriosas e distantes\\*
Que nossa dor parecem questionar!

Aonde quer que vá, estou sozinho,\\*
Eu, que contigo, tinha o mundo.\\*
Sinto só as dores ancestrais\\*
Por ordem do destino mesquinho,\\*
Sou lançado no vazio profundo\\*
Onde tudo não é nada, jamais!”
\end{verse}

