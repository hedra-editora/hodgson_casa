Quando li este livro pela primeira vez na infância, há quase cinquenta anos, eu já tinha uma prateleira cheia de
histórias de horror --- de Stoker a Derleth --- e um gosto exigente: adorava todas. Ratos dentro de paredes, mãos decepadas
assassinas, \textit{coisas} inomináveis oriundas de lugares inimagináveis --- tudo isso alimentava a minha imaginação,
faminta por subsistir da despensa da vida real no centro"-oeste americano. Tudo bem que Jesus me amasse --- não era um
fato interessante, mas tudo bem ---, mas certamente o Yog"-Sothoth de Lovecraft não me amava, e me parecia mais do que
plausível que, se o universo tinha espaço para deuses, eles deveriam ser do tipo maligno, como os de Lovecraft.

Quando me deparei com \textit{A casa do fim do mundo}, já conhecia o aterrorizante conto do autor, \textit{“The Voice
in the Night”}. Desde então, a imagem de um pequeno e maligno ponto fungiforme na mão que cresce cada vez mais assombra
minhas noites de maneiras bem interessantes. Que o desventurado amante tenha remado num barquinho até um navio parado
no mar para implorar por comida para si mesmo e sua amada, depois que toda a esperança de cura se esvai, parecia me
dizer algo sobre o papel do \textit{páthos} no amor que eu, ainda jovem, até então não suspeitava.

Em 1962, a Ace Books publicou uma versão em papel jornal de \textit{A casa do fim do mundo}, e foi provavelmente aí
que, pela primeira vez, a maioria dos leitores do pós"-guerra foram expostos a um texto mais extenso de William Hope
Hodgson. No começo da década de 1970, a Ballantine Books publicou duas outras fantasias de Hodgson, \textit{The Boats
of the “Glen Carrig”} e \textit{The Night Land}. As duas têm seus pontos fortes --- se você está buscando a sensação de
uma mão úmida e pegajosa na nuca ---, mas nenhuma delas transmite ao leitor o arrepiante desespero de \textit{A casa do
fim do mundo}. Não há um único feixe de luz esmaecida neste livro que não tenha um quê de terror. Se algum escritor
moderno ultrapassou Hodgson nesse aspecto, eu o desconheço.

\textit{A casa entre dois mundos}, publicado pela primeira vez em 1908, é mais do que uma história de horror. Quando o
narrador não está combatendo as criaturas suínas que habitam as cavernas sob a casa isolada, ele está sendo
transportado para um futuro desolador, cada vez mais imerso na escuridão. A Terra está morrendo. O Sol está morrendo. Dias e
séculos passam voando como as imagens projetadas nas paredes por uma luminária acesa. O que nos aguarda é uma escuridão
vazia sem fim; Hodgson compreendia que o verdadeiro horror é ter um vislumbre da dimensão cósmica.

Se algumas das descrições neste livro devem parte de seu mérito a \textit{A máquina do tempo}, de H. G. Wells (publicado
originalmente em 1895), muitos escritores posteriores devem muito mais a Hodgson. Os deuses monstruosos que observam a
casa do fim do mundo são antecedentes diretos dos deuses de Lovecraft. Para ambos os escritores, o universo, em seu
âmago, é um lugar apavorante e totalmente avesso a nós.

Uma década depois do lançamento deste livro, Hodgson morreu em combate, aos quarenta anos de idade, durante a Primeira Guerra
Mundial. 

MAX GERSH
